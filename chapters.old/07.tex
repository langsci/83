% CHAPTER 7 SYNTAX
\chapter{Syntax}\label{ch:syntax}
This chapter covers the syntax\index[sub]{syntax} of Southern Yauyos Quechua. The chapter counts fourteen sections covering constituent order, sentences, coordination, comparison, negation, interrogation, reflexives and reciprocals, equatives, possession, topic, emphasis, complementization, relativization and subordination. 

\section{Constituent order}\label{sec:conord}
The unmarked constituent order\index[sub]{constituent order} in \SYQ, as in other Quechuan languages, is SOV (\phono{Mila-qa} \phono{vikuña-n-kuna-ta} \phono{riku-ra} ‘Melanie saw her vicuñas’). Having said that, as constituents are obligatorily marked for case, they may appear in any order without necessarily varying the sense of the utterance (\phono{Mila-qa} \phono{riku-ra} \phono{vikuña-n-kuna-ta} ‘Melanie saw her vicuñas’). Change in constituent order does not necessarily change the interpretation of topic or focus. Topic is generally signaled by \phono{-qa}, while the evidentials \phono{-mI}, \phono{-shI}, and \phono{-trI} signal focus (\phono{Carmen-qa} \phono{llama-n-kuna-ta\pb{-sh}} \phono{wañu-chi-nqa} ‘Carmen will butcher her llamas, they say’ \phononb{Carmen-qa} \phononb{llama-n-kuna-ta} \phononb{wañu-chi-nqa\pb{-sh}} ‘Carmen will \pb{butcher} her \pb{llamas}, they say’). In the first case, the focus is on the direct object: she will butcher her llamas and not, say, her goats; in the second case, it is the verb that is marked as the focus: she will butcher her llamas and not, say, pet them. Nevertheless, the verb and the object cannot commute in subordinate clauses, where only the order OV is grammatical (\phono{fruta-cha-y-kuna} \phono{apa-sa-y-ta} ‘the fruit I bring’ \phono{*apa-sa-y-ta} \phono{fruta-cha-y-kuna-ta}).

Modifiers generally precede the elements they modify: adjectives precede the nouns they modify (\phono{yuraq} \phono{wayta} ‘white flower’), possessors precede the thing possessed (\phono{pay-pa} \phono{pupu-n} ‘her navel’), and relative clauses precede their heads (\phono{trabaha-sa-yki} \phono{wasi-pa} ‘in the house where you worked’). In case an NP includes multiple modifiers, these appear in the order:

\begin{center}
\begin{tabular}{c}
\lsptoprule
\lsc{dem}-\lsc{quant}-\lsc{num}-\lsc{neg}-\lsc{pre}\lsc{adj}-\lsc{adj}-\lsc{atr}-\lsc{nucleus}\\
\lspbottomrule
\end{tabular}
\end{center}

\section{Sentences}\label{sec:sentence}\index[sub]{sentence}
With the exceptions of (a)~abbreviated questions and responses to questions (\phono{¿May-pi?} ‘Where?’ \phono{Chay-pi-(m)} ‘There’), and (b)~exclamations (\phono{¡Atatayáw!} ‘How disgusting!’) no \SYQ{} sentence is grammatical without a verb (\phono{*Sasa}. ‘Hard’). As it is unnecessary in \SYQ{} to specify either the subject or the object, a verb alone inflected for person is sufficient for grammaticality (\phono{Apa-n} ‘[She] brings [it]’). First- and second-person objects are indicated in verbal inflection: \phono{-wa/-ma} indicates a first-person object, and \phono{-yki}, \phono{-sHQayki} and \phono{-shunki} indicate second-person objects (\phono{suya-wa-nki} ‘you wait for me’ \phono{suya-shunki} ‘She’ll wait for you’) (see~§~\ref{ssec:actorobjref} on actor-object reference).

\section{Coordination}\label{sec:coord}\index[sub]{sentence!coordination}
The enclitics \phono{-pis}, \phono{-taq}, and \phono{-raq} can all be used to coordinate NPs~(\ref{Glo7:Walmi}--\ref{Glo7:Uyqapaq}), AdvPs and VPs~(\ref{Glo7:Ishpani}); the case suffix \phono{-wan} can be used with the first two of these three~(\ref{Glo7:Mila}). \phono{-pis}, \phono{-taq}, and \phono{-raq} generally imply relations of inclusion, contrast, or contradiction, respectively. Thus, \phono{-pis} (inclusion) can generally be translated as ‘and’ or ‘also’~(\ref{Glo7:Walmi}),~(\ref{Glo7:Uyqapaq}).\\

\bgroup\renewcommand{\thefootnote}{\alph{footnote}}
% 1
\gloexe{Glo7:Walmi}{}{ch}%
{Walmi\pb{pis} qali\pb{pis}.}%ch que first line
{\morglo{walmi-pis}{woman-\lsc{add}}\morglo{qali-pis}{man-\lsc{add}}}%morpheme+gloss
\glotran{Women \pb{and} men.\protect\footnotemark[1]}{}%eng+spa trans
{}{}%rec - time
\footnotetext[1]{An anonymous reviewer suggests that a better gloss here would be ‘not only women, but men, too.’ This gloss would be consistent with an analysis of \phono{-pis} as generally indicating contrast. In this case, I am directly translating the Spanish gloss suggested to me by my consultant.}\egroup

% 2
\gloexe{Glo7:Uyqapaq}{}{ach}%
{Uyqapaq\pb{pis} kanmi alpakapaq\pb{pis} kanmi llamapaq\pb{pis} kanmi.}%ach que first line
{\morglo{uyqa-paq-pis}{sheep-\lsc{abl}-\lsc{add}}\morglo{ka-n-mi}{be-\lsc{3}-\lsc{evd}}\morglo{alpaka-paq-pis}{alpaca-\lsc{abl}-\lsc{add}}\morglo{ka-n-mi}{be-\lsc{3}-\lsc{evd}}\morglo{llama-paq-pis}{llama-\lsc{abl}-\lsc{add}}\morglo{ka-n-mi}{be--\lsc{3}--\lsc{evd}}}%morpheme+gloss
\glotran{There are [some] out of sheep [wool] \pb{and} there are [some] out of alpaca [wool] \pb{and} there are [some] out of llama [wool].}{}%eng+spa trans
{}{}%rec - time

% 3
\gloexe{Glo7:Ishpani}{}{amv}%
{Ishpani\pb{pis}chu puquchini\pb{pis}chu.}%amv que first line
{\morglo{ishpa-ni-pis-chu}{urinate-\lsc{1}-\lsc{add}-\lsc{neg}}\morglo{puqu-chi-ni-pis-chu}{ferment-\lsc{caus}-\lsc{1}-\lsc{add}-\lsc{neg}}}%morpheme+gloss
\glotran{I \pb{neither} urinate \pb{nor} ferment [urine].}{}%eng+spa trans
{}{}%rec - time

\noindent
\phono{-wan} is unmarked and can generally be translated as ‘and’~(\ref{Glo7:Mila}).\\

% 4
\gloexe{Glo7:Mila}{}{amv}%
{Mila\pb{wan} Alicia\pb{wan} Hilda trayaramun.~\updag}%amv que first line
{\morglo{Mila-wan}{Mila-\lsc{instr}}\morglo{Alicia-wan}{Alicia-\lsc{instr}}\morglo{Hilda}{Hilda-\lsc{instr}}\morglo{traya-ra-mu-n}{arrive-\lsc{urgt}-\lsc{cisl}-\lsc{3}}}%morpheme+gloss
\glotran{Hilda arrived with Mila \pb{and} Alicia.}{}%eng+spa trans
{}{}%rec - time

\noindent
\phono{-taq} and \phono{-raq} (contrast and contradiction) can both be translated ‘but’, ‘while’, ‘whereas’ and so on~(\ref{Glo7:Wawanchikta}).\\

% 5
\gloexe{Glo7:Wawanchikta}{}{ach}%
{Wawanchikta idukanchik qillakunaqa mana\pb{taq}mi.}%ach que first line
{\morglo{wawa-nchik-ta}{baby-\lsc{1pl}-\lsc{acc}}\morglo{iduka-nchik}{educate-\lsc{1pl}}\morglo{qilla-kuna-qa}{lazy-\lsc{pl}-\lsc{top}}\morglo{mana-taq-mi}{no-\lsc{seq}-\lsc{evd}}}%morpheme+gloss
\glotran{We’re educating our children; \pb{whereas} the lazy ones aren’t.}{}%eng+spa trans
{}{}%rec - time

\noindent
Additional strategies employed for coordination in \SYQ{} include (a)~the employment of the indigenous coordinating particle \phono{icha} ‘or’~(\ref{Glo7:Mikuramanmantri}) or any of the borrowed Spanish coordinators \phono{i} ‘and’~(\ref{Glo7:Tushunchik}), \phono{u} ‘or’~(\ref{Glo7:Kaytaq}), \phono{piru} ‘but’~(\ref{Glo7:turiypis}), or \phono{ni} ‘nor’~(\ref{Glo7:alpaka}) (\Sp~\spanish{y}, \spanish{o}, \spanish{pero}, and \spanish{ni}) and (b)~juxtaposition.\\

% 6
\gloexe{Glo7:Mikuramanmantri}{}{ach}%
{Mikuramanmantri kara \pb{icha} aparamanmantri.}%ach que first line
{\morglo{miku-ra-ma-n-man-tri}{eat-\lsc{urgt}-\lsc{1.obj}-\lsc{3}-\lsc{cond}-\lsc{evc}}\morglo{ka-ra}{be-\lsc{pst}}\morglo{icha}{or}\morglo{apa-ra-ma-n-man-tri}{bring-\lsc{urgt}-\lsc{1.obj}-\lsc{3}-\lsc{cond}-\lsc{evc}}}%morpheme+gloss
\glotran{It would have eaten me \pb{or} it would have taken me away.}{}%eng+spa trans
{}{}%rec - time

% 7
\gloexe{Glo7:Tushunchik}{}{ch}%
{Tushunchik i imahintam kriyinchik ñuqakunaqa \pb{piru} chay ivanhilyukuna sabadistakunaqa mana kriyinchu.}%ch que first line
{\morglo{tushu-nchik}{dance-\lsc{1pl}}\morglo{i}{and}\morglo{imahin-ta-m}{image-\lsc{acc}-\lsc{evd}}\morglo{kriyi-nchik}{believe-\lsc{1pl}}\morglo{ñuqa-kuna-qa}{1-\lsc{pl}-\lsc{top}}\morglo{piru}{but}\morglo{chay}{\lsc{dem.d}}\morglo{ivanhilyu-kuna}{Evangelical-\lsc{pl}}\morglo{sabadista-kuna-qa}{Seventh.Day.Adventist-\lsc{pl}-\lsc{top}}\morglo{mana}{no}\morglo{kriyi-n-chu}{believe-\lsc{3}-\lsc{neg}}}%morpheme+gloss
\glotran{We dance and believe in the saints \pb{but} those Evangelists and Seventh Day Adventists don’t believe.}{}%eng+spa trans
{}{}%rec - time

% 8
\gloexe{Glo7:Kaytaq}{}{ach}%
{Kaytaq ishkay puntraw \pb{u} huk puntrawllam ruwa:.}%ach que first line
{\morglo{kay-taq}{\lsc{dem.p}-\lsc{seq}}\morglo{ishkay}{two}\morglo{puntraw}{day}\morglo{u}{or}\morglo{huk}{one}\morglo{puntraw-lla-m}{day-\lsc{rstr}-\lsc{evd}}\morglo{ruwa-:}{make-\lsc{1}}}%morpheme+gloss
\glotran{I make this one in two days \pb{or} just one day.}{}%eng+spa trans
{}{}%rec - time

% 9
\gloexe{Glo7:turiypis}{}{amv}%
{“Ñañaypis, turiypis karqam \pb{piru} wañukunña,” nishpa, ¡rimay!}%amv que first line
{\morglo{ñaña-y-pis,}{sister-\lsc{1}-\lsc{add}}\morglo{turi-y-pis}{brother-\lsc{1}-\lsc{add}}\morglo{ka-rqa-m}{be-\lsc{pst}-\lsc{evd}}\morglo{\pb{piru}}{but}\morglo{wañu-ku-n-ña}{die-\lsc{refl}-\lsc{3}-\lsc{disc}}\morglo{ni-shpa}{say-\lsc{subis}}\morglo{rima-y}{talk-\lsc{imp}}}%morpheme+gloss
\glotran{Say, “I had a sister and a brother, \pb{but} they died.” Talk!}{}%eng+spa trans
{}{}%rec - time

% 10
\gloexe{Glo7:alpaka}{}{ach}%
{\pb{Ni} alpaka \pb{ni} llama. Kanan manam trayamun\pb{chu}.}%ach que first line
{\morglo{ni}{nor}\morglo{alpaka}{alpaca}\morglo{ni}{nor}\morglo{llama}{llama}\morglo{kanan}{now}\morglo{mana-m}{no-\lsc{evd}}\morglo{traya-mu-n-chu}{arrive-\lsc{cisl}-\lsc{3}-\lsc{neg}}}%morpheme+gloss
\glotran{\pb{Neither} alpacas \pb{nor} llamas. They don’t come here now.}{}%eng+spa trans
{}{}%rec - time

\noindent
Juxtaposition consists of the placement of the coordinated elements in sequence~(\ref{Glo7:Sibadakunata}), (\ref{Glo7:Walmiqa}).\\

% 11
\gloexe{Glo7:Sibadakunata}{}{ach}%
{Sibadakunata kargashpa, triguta rantishpa, sarata rantishpam purira.}%ach que first line
{\morglo{sibada-kuna-ta}{barley-\lsc{pl}-\lsc{acc}}\morglo{karga-shpa}{carry-\lsc{subis}}\morglo{trigu-ta}{wheat-\lsc{acc}}\morglo{ranti-shpa}{buy-\lsc{subis}}\morglo{sara-ta}{corn-\lsc{acc}}\morglo{ranti-shpa-m}{buy-\lsc{subis}-\lsc{evd}}\morglo{puri-ra}{walk-\lsc{pst}}}%morpheme+gloss
\glotran{They walked about, carrying barley and selling wheat \pb{and} selling corn.}{}%eng+spa trans
{}{}%rec - time

% 12
\gloexe{Glo7:Walmiqa}{}{ch}%
{Walmiqa talpunchik, allichanchikmi.}%ch que first line
{\morglo{walmi-qa}{woman-\lsc{top}}\morglo{talpu-nchik}{plant-\lsc{1pl}}\morglo{alli-cha-nchik-mi}{good-\lsc{fact}-\lsc{1pl}-\lsc{evd}}}%morpheme+gloss
\glotran{We women plant \pb{and} fix up [the soil].}{}%eng+spa trans
{}{}%rec - time

\noindent
When \phono{-kuna} signals inclusion, it can be used to coordinate NP’s~(\ref{Glo7:Chayman}) (see~§~\ref{sssc:nonex}).\\

% 13
\gloexe{Glo7:Chayman}{}{amv}%
{Chayman risa Marleni, Ayde, Vilma, Norma\pb{kuna}.}%amv que first line
{\morglo{chay-man}{\lsc{dem.d}-\lsc{all}}\morglo{ri-sa}{go-\lsc{npst}}\morglo{Marleni}{Marleni}\morglo{Ayde}{Ayde}\morglo{Vilma}{Vilma}\morglo{Norma-kuna}{Norma-\lsc{pl}}}%morpheme+gloss
\glotran{Marleni went there with Ayde, Vilma \pb{and} Norma.}{}%eng+spa trans
{}{}%rec - time

\noindent
The Spanish coordinators are widely employed. Coordinators indigenous to \SYQ{} generally attach to both coordinated elements~(\ref{Glo7:Ullqush}). The coordinators are not necessarily mutually exclusive.\\

% 14
\gloexe{Glo7:Ullqush}{}{amv}%
{Ullqush\pb{pis} kayan, ¿aw? Chuqlluqupa\pb{pis} yuraq\pb{pis} puka\pb{pis}.}%amv que first line
{\morglo{ullqush-pis}{ullqush.flowers-\lsc{add}}\morglo{ka-ya-n}{be-\lsc{prog}-\lsc{3}}\morglo{aw}{yes}\morglo{chuqlluqupa-pis}{chuqlluqupa.flowers-\lsc{add}}\morglo{yuraq-pis}{white-\lsc{add}}\morglo{puka-pis}{red-\lsc{add}}}%morpheme+gloss
\glotran{There are \phono{ullqush} flowers, \pb{too}, no? \phono{Chuqlluqupa} flowers, \pb{too} -- white and red.}{}%eng+spa trans
{}{}%rec - time

\section{Comparison}\index[sub]{sentence!comparison}
Comparisons of inequality are formed in \SYQ{} with the borrowed particle \phono{mas} (‘more’) in construction with the indigenous ablative case suffix, \phono{-paq}, which attaches to the base of comparison~(\ref{Glo7:Huancayopaqa}), (\ref{Glo7:Qaynapuntraw}).\\

% 1
\gloexe{Glo7:Huancayopaqa}{}{amv}%
{Huancayopaqa wak mashwaqa papa\pb{paq}pis \pb{mas}mi kwistan.}%amv que first line
{\morglo{Huancayo-pa-qa}{Huancayo-\lsc{loc}-\lsc{top}}\morglo{wak}{\lsc{dem.d}}\morglo{mashwa-qa}{mashua-\lsc{top}}\morglo{papa-paq-pis}{potato-\lsc{abl}-\lsc{add}}\morglo{mas-mi}{more-\lsc{evd}}\morglo{kwista-n}{cost-\lsc{3}}}%morpheme+gloss
\glotran{In Huancayo, mashua costs \pb{more} than potatoes.}{}%eng+spa trans
{}{}%rec - time

% 2
\gloexe{Glo7:Qaynapuntraw}{}{amv}%
{Qayna puntraw\pb{paq} \pb{mas}mi.}%amv que first line
{\morglo{qayna}{previous}\morglo{puntraw-paq}{day-\lsc{abl}}\morglo{mas-mi}{more-\lsc{evd}}}%morpheme+gloss
\glotran{It’s \pb{more} than yesterday.}{}%eng+spa trans
{}{}%rec - time

\noindent
\phono{mas} and \phono{minus} ‘less’, also borrowed from Spanish, may function as pronouns~(\ref{Glo7:Granadakunaktapis}) and adjectives~(\ref{Glo7:Qaynawata}), and, when inflected with accusative \phono{-ta}, as adverbs~(\ref{Glo7:Masta}) as well.\\

% 3
\gloexe{Glo7:Granadakunaktapis}{}{ch}%
{Granadakunaktapis, armamintukunaktapis lantiyan \pb{mas}ta.}%ch que first line
{\morglo{granada-kuna-kta-pis}{grenade-\lsc{pl}-\lsc{acc}-\lsc{add}}\morglo{armamintu-kuna-kta-pis}{armaments-\lsc{pl}-\lsc{acc}-\lsc{add}}\morglo{lanti-ya-n}{buy-\lsc{prog}-\lsc{3}}\morglo{mas-ta}{more-\lsc{acc}}}%morpheme+gloss
\glotran{Grenades and weapons and all, too -- they’re buying \pb{more}.}{}%eng+spa trans
{}{}%rec - time

% 4 (3*)
\gloexe{Glo7:Qaynawata}{}{amv}%
{Qayna wata pukum karqa. Chaymi \pb{minus} pastupis karqa.}%amv que first line
{\morglo{qayna}{previous}\morglo{wata}{year}\morglo{puku-m}{little-\lsc{evd}}\morglo{ka-rqa}{be-\lsc{pst}}\morglo{chay-mi}{\lsc{dem.d}-\lsc{evd}}\morglo{minus}{less}\morglo{pastu-pis}{pasture.grass-\lsc{add}}\morglo{ka-rqa}{be-\lsc{pst}}}%morpheme+gloss
\glotran{Last year there was little [rain]. So there was \pb{less} pasture grass.}{}%eng+spa trans
{}{}%rec - time

% 5
\gloexe{Glo7:Masta}{}{lt}%
{\pb{Masta}qa mashtakuyanmi.}%lt que first line
{\morglo{mas-ta-qa}{more-\lsc{acc}-\lsc{top}}\morglo{mashta-ku-ya-n-mi}{spread-\lsc{refl}-\lsc{prog}-\lsc{3}-\lsc{evd}}}%morpheme+gloss
\glotran{It’s spreading out \pb{more}.}{}%eng+spa trans
{}{}%rec - time

\noindent
Also borrowed from Spanish are the irregular \phono{mihur} ‘better’~(\ref{Glo7:Pular}) and \phono{piyur} ‘worse’ (\ref{Glo7:Unayqamana}), (\ref{Glo7:Sapa}).\\

% 6
\gloexe{Glo7:Pular}{}{ach}%
{Pular\pb{paq}pis \pb{mas} \pb{mihur}tam chayqa ayllukun.}%ach que first line
{\morglo{pular-paq-pis}{fleece-\lsc{abl}-\lsc{add}}\morglo{mas}{more}\morglo{mihur-ta-m}{better-\lsc{acc}-\lsc{evd}}\morglo{chay-qa}{\lsc{dem.d}-\lsc{top}}\morglo{ayllu-ku-n}{wrap-\lsc{refl}-\lsc{3}}}%morpheme+gloss
\glotran{It’s \pb{much better} than fleece -- this wraps [you] up.}{}%eng+spa trans
{}{}%rec - time

% 7
\gloexe{Glo7:Unayqamana}{}{amv}%
{Unayqa manayá iskwilaqa kasa. Unayqa analfabitullaya kayaq. Warmiqa \pb{piyur}.}%amv que first line
{\morglo{unay-qa}{before-\lsc{top}}\morglo{mana-yá}{no-\lsc{emph}}\morglo{iskwila-qa}{school-\lsc{top}}\morglo{ka-sa}{be-\lsc{npst}}\morglo{unay-qa}{before-\lsc{top}}\morglo{analfabitu-lla-ya}{illiterate-\lsc{rstr}-\lsc{emo}}\morglo{ka-ya-q}{be-\lsc{prog}-\lsc{ag}}\morglo{warmi-qa}{woman-\lsc{top}}\morglo{piyur}{worse}}%morpheme+gloss
\glotran{Ah, before, they didn’t have schools. Before, they were just illiterate. \pb{Worse} [for the] women.}{}%eng+spa trans
{}{}%rec - time

% 8
\gloexe{Glo7:Sapa}{}{amv}%
{Sapa putraw \pb{piyur piyur}ñam kayani. Mastaña qayna puntraw mana puriyta wakchawta qatiyta atipanichu.}%amv que first line
{\morglo{sapa}{every}\morglo{putraw}{day}\morglo{piyur}{worse}\morglo{piyur-ña-m}{worse-\lsc{disc}-\lsc{evd}}\morglo{ka-ya-ni}{be-\lsc{prog}-\lsc{1}}\morglo{mas-ta-ña}{more-\lsc{acc}-\lsc{disc}}\morglo{qayna}{previous}\morglo{puntraw}{day}\morglo{mana}{no}\morglo{puri-y-ta}{walk-\lsc{inf}-\lsc{acc}}\morglo{wakchaw-ta}{sheep-\lsc{acc}}\morglo{qati-y-ta}{follow-\lsc{inf}-\lsc{acc}}\morglo{atipa-ni-chu}{be.able-\lsc{1}-\lsc{neg}}}%morpheme+gloss
\glotran{Every day it’s worse, I’m worse. More yesterday. I couldn’t walk or take out my sheep.}{}%eng+spa trans
{}{}%rec - time

\noindent
Comparisons of equality are formed with the borrowed particle \phono{igwal} ‘equal’, ‘same’ in construction with the indigenous instrumental/comitative case suffix, \phono{-wan}, which attaches to the base of comparison~(\ref{Glo7:Runa}).\\

% 9
\gloexe{Glo7:Runa}{}{amv}%
{Runa\pb{wan} \pb{igwal}triki vakaqa: nuybi mis.}%amv que first line
{\morglo{runa-wan}{person-\lsc{instr}}\morglo{igwal-tr-iki}{equal-\lsc{evc}-\lsc{iki}}\morglo{vaka-qa:}{cow-\lsc{top}}\morglo{nuybi}{nine}\morglo{mis}{month}}%morpheme+gloss
\glotran{Cows are the \pb{same} as people: [they gestate for] nine months.}{}%eng+spa trans
{}{}%rec - time

\section{Negation}\label{sec:negation}\index[sub]{sentence!negation}
This section partially repeats §~\ref{ssec:innedi} on \phono{-chu}. Please consult that section for further discussion and glossed examples. In \SYQ, negation\index[sub]{negation} is indicated by the enclitic \phono{-chu} in combination with any of the particles \phono{mana}, \phono{ama}, or \phono{ni} or with the enclitic suffix \phono{-pis}. \phono{-chu} attaches to the sentence fragment that is the focus of negation. In negative sentences, \phono{-chu} generally co-occurs with \phono{mana} ‘not’~(\ref{Glo7:Chaytri}),~(\ref{Glo7:Aa}). \phono{-chu} is also licensed by additive \phono{-pis}~(\ref{Glo7:Kaspin}),~(\ref{Glo7:Manchakushpa}) as well as by \phono{ni} ‘nor’~(\ref{Glo7:Apuraw}),~(\ref{Glo7:Manamwayta}).\\

% 1
\gloexe{Glo7:Chaytri}{}{amv}%
{Chaytri \pb{mana} suyawarqa\pb{chu}.}%amv que first line
{\morglo{chay-tri}{\lsc{dem.d}-\lsc{evc}}\morglo{mana}{no}\morglo{suya-wa-rqa-chu}{wait-\lsc{1.obj}-\lsc{pst}-\lsc{neg}}}%morpheme+gloss
\glotran{That’s why she would\pb{n’t} have waited for me.}{}%eng+spa trans
{}{}%rec - time

% 2
\gloexe{Glo7:Aa}{}{lt}%
{Aa, \pb{mana}ya kan\pb{chu}. \pb{Mana}ya bulayuq kan\pb{chu}.}%lt que first line
{\morglo{aa}{ah}\morglo{mana-ya}{no-\lsc{emo}}\morglo{ka-n-chu}{be-\lsc{3}-\lsc{neg}}\morglo{mana-ya}{no-\lsc{emo}}\morglo{bula-yuq}{ball-\lsc{poss}}\morglo{ka-n-chu}{be-\lsc{3}-\lsc{neg}}}%morpheme+gloss
\glotran{Ah, there are\pb{n’t} any. \pb{No one} has any balls.}{}%eng+spa trans
{}{}%rec - time

% 3
\gloexe{Glo7:Kaspin}{}{amv}%
{Kaspin\pb{pis} kan\pb{chu}.}%amv que first line
{\morglo{kaspi-n-pis}{stick-\lsc{3}-\lsc{add}}\morglo{ka-n-chu}{be-\lsc{3}-\lsc{neg}}}%morpheme+gloss
\glotran{She \pb{doesn’t} have a stick.}{}%eng+spa trans
{}{}%rec - time

% 4
\gloexe{Glo7:Manchakushpa}{}{ach}%
{Manchakushpa tuta\pb{s} puñu:\pb{chu}.}%ach que first line
{\morglo{mancha-ku-shpa}{scare-\lsc{refl}-\lsc{subis}}\morglo{tuta-s}{night-\lsc{add}}\morglo{puñu-:-chu}{sleep-\lsc{1}-\lsc{neg}}}%morpheme+gloss
\glotran{Being scared, I \pb{didn’t} sleep at night.}{}%eng+spa trans
{}{}%rec - time

% 5
\gloexe{Glo7:Apuraw}{}{amv}%
{Apuraw wañururqariki. \pb{Ni} apanña\pb{chu}.}%amv que first line
{\morglo{apuraw}{quick}\morglo{wañu-ru-rqa-r-iki}{die-\lsc{urgt}-\lsc{pst}-\lsc{r}-\lsc{iki}}\morglo{ni}{nor}\morglo{apa-n-ña-chu}{bring-\lsc{3}-\lsc{disc}-\lsc{neg}}}%morpheme+gloss
\glotran{He died quickly. They \pb{didn’t even} bring him [to the hospital].}{}%eng+spa trans
{}{}%rec - time

% 6
\gloexe{Glo7:Manamwayta}{}{amv}%
{\pb{Manam} wayta\pb{chu} \pb{ni} pishqu\pb{chu}.}%amv que first line
{\morglo{manam}{no-\lsc{evd}}\morglo{wayta-chu}{flower-\lsc{neg}}\morglo{ni}{nor}\morglo{pishqu-chu}{bird-\lsc{neg}}}%morpheme+gloss
\glotran{\pb{Neither} a flower \pb{nor} a bird.}{}%eng+spa trans
{}{}%rec - time

\noindent
\phono{-chu} co-occurs with \phono{ama} in prohibitions~(\ref{Glo7:Ama}), imperatives~(\ref{Glo7:kutimu}), (\ref{Glo7:nunka}), and injunctions~(\ref{Glo7:chun}).\\

% 7
\gloexe{Glo7:Ama}{}{amv}%
{¡\pb{Ama} manchariy\pb{chu}! ¡\pb{Ama} qaway\pb{chu}!}%amv que first line
{\morglo{ama}{\lsc{proh}}\morglo{mancha-ri-y-chu}{scare-\lsc{incep}-\lsc{imp}-\lsc{neg}}\morglo{ama}{\lsc{ama}}\morglo{qawa-y-chu}{look-\lsc{imp}-\lsc{chu}}}%morpheme+gloss
\glotran{\pb{Don’t} be scared! \pb{Don’t} look!}{}%eng+spa trans
{}{}%rec - time

% 8
\gloexe{Glo7:kutimu}{}{amv}%
{¡\pb{Ama} kutimu\pb{nki}\pb{chu}! Qamqa isturbum kayanki.}% que first line
{\morglo{ama}{\lsc{proh}}\morglo{kuti-mu-nki-chu}{return-\lsc{cisl}-\lsc{2}-\lsc{neg}}\morglo{qam-qa}{you-\lsc{top}}\morglo{isturbu-m}{nuisance-\lsc{evd}}\morglo{ka-ya-nki}{be-\lsc{prog}-\lsc{2}}}%morpheme+gloss
\glotran{\pb{Don’t} you come back! You’re a hinderance.}{}%eng+spa trans
{}{}%rec - time

% 9
\gloexe{Glo7:nunka}{}{lt}%
{¡\pb{Ama}m nunka katraykanaku\pb{shun}\pb{chu}!}%lt que first line
{\morglo{ama-m}{\lsc{proh}-\lsc{evd}}\morglo{nunka}{never}\morglo{katra-yka-naku-shun-chu}{release-\lsc{excep}-\lsc{recip}-\lsc{1pl.fut}-\lsc{neg}}}%morpheme+gloss
\glotran{\pb{Let’s never} leave each other!}{}%eng+spa trans
{}{}%rec - time

% 10
\gloexe{Glo7:chun}{}{amv}%
{¡\pb{Ama} wañu\pb{chun}\pb{chu}!~\updag}%amv que first line
{\morglo{ama}{\lsc{proh}}\morglo{wañu-chun-chu}{die-\lsc{injunc}-\lsc{neg}}}%morpheme+gloss
\glotran{\pb{Don’t let} her die!}{}%eng+spa trans
{}{}%rec - time

\noindent
\phono{-chu} does not appear in subordinate clauses. In subordinate clauses negation is indicated with a negative particle alone~(\ref{Glo7:kaptinqa}--\ref{Glo7:qatrachakunanpaq}).\\

% 11
\gloexe{Glo7:kaptinqa}{}{ch}%
{\pb{Mana} qali kaptinqa ñuqanchikpis taqllakta hapishpa qaluwanchik.}%ch que first line
{\morglo{mana}{no}\morglo{qali}{man}\morglo{ka-pti-n-qa}{be-\lsc{subds}-\lsc{3}-\lsc{top}}\morglo{ñuqanchik-pis}{we-\lsc{add}}\morglo{taqlla-kta}{plow-\lsc{acc}}\morglo{hapi-shpa}{grab-\lsc{subis}}\morglo{qaluwa-nchik}{turn.earth-\lsc{1pl}}}%morpheme+gloss
\glotran{When there are \pb{no} men, we grab the plow and turn the earth.}{}%eng+spa trans
{}{}%rec - time

% 12
\gloexe{Glo7:qatrachakunanpaq}{}{amv}%
{\pb{Mana} qatrachakunanpaq mandilchanta watachakun.}%amv que first line
{\morglo{mana}{no}\morglo{qatra-cha-ku-na-n-paq}{dirty-\lsc{fact}-\lsc{refl}-\lsc{nmlz}-\lsc{3}-\lsc{purp}}\morglo{mandil-cha-n-ta}{apron-\lsc{dim}-\lsc{3}-\lsc{acc}}\morglo{wata-cha-ku-n}{tie-\lsc{dim}-\lsc{refl}-\lsc{3}}}%morpheme+gloss
\glotran{She’s tying on her apron \pb{so} she does\pb{n’t} get dirty.}{}%eng+spa trans
{}{}%rec - time

\section{Interrogation}\label{sec:interr}\index[sub]{sentence!interrogation}
This section partially repeats §~\ref{sec:IntInd} and §~\ref{ssec:innedi} on interrogative indefinites and \phono{-chu} Please consult those sections for further discussion and glossed examples.

Absolute~(\ref{Glo7:Chuqamunkiman}) and disjunctive~(\ref{Glo7:Maytaq}), ~(\ref{Glo7:Maniyayan}) questions are formed with the enclitic \phono{-chu}. When it functions to indicate interrogation\index[sub]{interrogation}, \phono{-chu} attaches to the sentence fragment that is the focus of the interrogation~(\ref{Glo7:Chaypa}).\\

% 1
\gloexe{Glo7:Chuqamunkiman}{}{amv}%
{¿Chuqamunkiman\pb{chu}?}%amv que first line
{\morglo{chuqa-mu-nki-man-chu}{throw-\lsc{cisl}-\lsc{2}-\lsc{cond}-\lsc{q}}}%morpheme+gloss
\glotran{Can you throw?}{}%eng+spa trans
{}{}%rec - time

% 2
\gloexe{Glo7:Maytaq}{}{ch}%
{¿Maytaq chayqa? ¿Apurí\pb{chu} Viñac\pb{chu}?}%ch que first line
{\morglo{may-taq}{where-\lsc{seq}}\morglo{chay-qa}{\lsc{dem.d}-\lsc{top}}\morglo{Apurí-chu}{Apurí-\lsc{q}}\morglo{Viñac-chu}{Viñac-\lsc{q}}}%morpheme+gloss
\glotran{Where is that? Apurí \pb{or} Viñac?}{}%eng+spa trans
{}{}%rec - time

% 3
\gloexe{Glo7:Maniyayan}{}{amv}%
{¿Maniyayan \pb{icha} katrariyan\pb{chu}?}%amv que first line
{\morglo{maniya-ya-n}{tie.limbs-\lsc{prog}-\lsc{3}}\morglo{icha}{or}\morglo{katra-ri-ya-n-chu}{release-\lsc{incep}-\lsc{prog}-\lsc{3}-\lsc{neg}}}%morpheme+gloss
\glotran{Is she tying its feet \pb{or} is she setting it loose?}{}%eng+spa trans
{}{}%rec - time

% 4
\gloexe{Glo7:Chaypa}{}{amv}%
{¿Chaypa\pb{chu} tumarqanki?}%amv que first line
{\morglo{chay-pa-chu}{\lsc{dem.d}-\lsc{loc}-\lsc{q}}\morglo{tuma-rqa-nki}{take-\lsc{pst}-\lsc{2}}}%morpheme+gloss
\glotran{Did you take [pictures] there?}{}%eng+spa trans
{}{}%rec - time

\noindent
In disjunctive questions, it generally attaches to each of the disjuncts~(\ref{Glo7:Kanastapi}).\\

% 5
\gloexe{Glo7:Kanastapi}{}{amv}%
{¿Kanastapi\pb{chu} baldipi\pb{chu}?}%amv que first line
{\morglo{kanasta-pi-chu}{basket-\lsc{loc}-\lsc{q}}\morglo{baldi-pi-chu}{bucket-\lsc{loc}-\lsc{q}}}%morpheme+gloss
\glotran{In the basket \pb{or} in the bucket?}{}%eng+spa trans
{}{}%rec - time

\noindent
Questions that anticipate a negative answer are indicated by \phono{manachu}~(\ref{Glo7:Manachu}).\\

% 6
\gloexe{Glo7:Manachu}{}{amv}%
{¿\pb{Manachu} friqulniki? ¿Puchukarun\pb{chu}?}%amv que first line
{\morglo{mana-chu}{no-\lsc{q}}\morglo{friqul-ni-ki}{bean-\lsc{euph}-\lsc{2}}\morglo{puchuka-ru-n-chu}{finish-\lsc{urgt}-\lsc{3}-\lsc{q}}}%morpheme+gloss
\glotran{Do\pb{n’t} you have any beans? They’re finished?}{}%eng+spa trans
{}{}%rec - time

\noindent
\phono{Manachu} may also “soften” questions~(\ref{Glo7:wankuchata}).\\

% 7
\gloexe{Glo7:wankuchata}{}{amv}%
{¿\pb{Manachu} chay wankuchata qawanki?}%amv que first line
{\morglo{mana-chu}{no-\lsc{q}}\morglo{chay}{\lsc{dem.d}}\morglo{wanku-cha-ta}{mold-\lsc{dim}-\lsc{acc}}\morglo{qawa-nki}{see-\lsc{2}}}%morpheme+gloss
\glotran{You have\pb{n’t} seen the little [cheese] mold?}{}%eng+spa trans
{}{}%rec - time

\noindent
\phono{Manachu}, like \phono{aw} ‘yes’, may also be used in the formation of tag questions~(\ref{Glo7:chimpapaqa}).\\

% 8
\gloexe{Glo7:chimpapaqa}{}{ach}%
{Wak chimpapaqa yuraqyayan, ¿\pb{manachu}?}%ach que first line
{\morglo{wak}{\lsc{dem.d}}\morglo{chimpa-pa-qa}{front-\lsc{loc}-\lsc{top}}\morglo{yuraq-ya-ya-n}{white-\lsc{inch}-\lsc{prog}-\lsc{3}}\morglo{mana-chu}{no-\lsc{q}}}%morpheme+gloss
\glotran{There in front they’re turning white, \pb{aren’t they}?}{}%eng+spa trans
{}{}%rec - time

\noindent
Interrogative \phono{-chu} does not appear in questions using interrogative pronouns~(\ref{Glo7:haqtrirqa}), (\ref{Glo7:Pitaq}).\\

% 9
\gloexe{Glo7:haqtrirqa}{}{amv}%
{*¿Pi haqtrirqa\pb{chu?}}% que first line
{\morglo{pi}{who}\morglo{haqtri-rqa-chu}{sneeze-\lsc{pst}-\lsc{q}}}%morpheme+gloss
\glotran{Who sneezed?}{}%eng+spa trans
{}{}%rec - time

% 10
\gloexe{Glo7:Pitaq}{}{amv}%
{*¿Pitaq qurquryara\pb{chu}? *¿Pitaq\pb{chu} qurquryara?}% que first line
{\morglo{pi-taq}{who-\lsc{seq}}\morglo{qurqurya-ra-chu}{snore-\lsc{pst}-\lsc{q}}\morglo{pi-taq-chu}{who-\lsc{seq}-\lsc{q}}\morglo{qurqurya-ra}{snore-\lsc{pst}}}%morpheme+gloss
\glotran{Who snored?}{}%eng+spa trans
{}{}%rec - time

Constituent questions are formed with the interrogative-indefinite stems \phono{pi} ‘who’, \phono{ima} ‘what’, \phono{imay} ‘when’, \phono{may} ‘where’, \phono{imayna} ‘how’, \phono{mayqin} ‘which’, \phono{imapaq} ‘why’, and \phono{ayka} ‘how much/many’ (see~Table~\ref{Tab9}). Interrogative pronouns are formed by suffixing the stem --~generally but not obligatorily~-- with one of the enclitics \phono{-taq}, \phono{-raq}, \phono{-mI}, \phono{-shI} or \phono{-trI}~(\ref{Glo7:uraraq}--\ref{Glo7:Tapun}).\\

% 11 (1*)
\gloexe{Glo7:uraraq}{}{sp}%
{¿\pb{Imay uraraq} chay kunihuqa kutimunqa yanapamananpaq?}%sp que first line
{\morglo{imay}{when}\morglo{ura-raq}{hour-\lsc{cont}}\morglo{chay}{\lsc{dem.d}}\morglo{kunihu-qa}{rabbit-\lsc{top}}\morglo{kuti-mu-nqa}{return-\lsc{cisl}-\lsc{3.fut}}\morglo{yanapa-ma-na-n-paq}{help-\lsc{1.obj}-\lsc{nmlz}-\lsc{3}-\lsc{purp}}}%morpheme+gloss
\glotran{\pb{What time} is that rabbit going to come back so he can help me?}{}%eng+spa trans
{}{}%rec - time

% 12 (2*)
\gloexe{Glo7:Imatr}{}{lt}%
{¿\pb{Imatr} kakun?}%lt que first line
{\morglo{ima-tr}{what-\lsc{evc}}\morglo{ka-ku-n}{be-\lsc{refl}-\lsc{3}}}%morpheme+gloss
\glotran{\pb{What} could it be?}{}%eng+spa trans
{}{}%rec - time

% 13 (3*)
\gloexe{Glo7:Tapun}{}{ach}%
{Tapun, “¿\pb{Imapaq} waqakunki, paluma?”}%ach que first line
{\morglo{tapu-n}{ask-\lsc{3}}\morglo{ima-paq}{what-\lsc{purp}}\morglo{waqa-ku-nki}{cry-\lsc{refl}-\lsc{2}}\morglo{paluma}{dove}}%morpheme+gloss
\glotran{He asked, “\pb{Why} are you crying, dove?”}{}%eng+spa trans
{}{}%rec - time

\noindent
Interrogative pronouns are suffixed with the case markers corresponding to the questioned element~(\ref{Glo7:pasaruptin}), (\ref{Glo7:Traklamanchu}).\\

% 14 (4*)
\gloexe{Glo7:pasaruptin}{}{amv}%
{¿Inti pasaruptin \pb{imay urata} munayan?}%amv que first line
{\morglo{inti}{sun}\morglo{pasa-ru-pti-n}{pass-\lsc{urgt}-\lsc{subds}-\lsc{3}}\morglo{imay}{when}\morglo{ura-ta}{hour-\lsc{acc}}\morglo{muna-ya-n}{want-\lsc{prog}-\lsc{3}}}%morpheme+gloss
\glotran{\pb{What time} will it be when the sun sets?}{}%eng+spa trans
{}{}%rec - time

% 15 (5*)
\gloexe{Glo7:Traklamanchu}{}{ch}%
{¿Traklamanchu liyan? ¿\pb{Piwan}yá?}%ch que first line
{\morglo{trakla-man-chu}{field-\lsc{all}-\lsc{q}}\morglo{li-ya-n}{go-\lsc{prog}-\lsc{3}}\morglo{pi-wan-yá}{who-\lsc{instr}-\lsc{emph}}}%morpheme+gloss
\glotran{Is he going to the field? \pb{With whom}?}{}%eng+spa trans
{}{}%rec - time

\noindent
The enclitic generally attaches to the final word in the interrogative phrase: when the interrogative pronoun completes the phrase, it attaches directly to the interrogative; in contrast, when the phrase includes an NP, the enclitic attaches to the NP (\phono{pi-\pb{paq}-\pb{taq}} ‘for whom’ \phono{ima} \phono{qullqi-\pb{tr}} ‘what money’)~(\ref{Glo7:Chaypaqa}).\\

% 16 (6*)
\gloexe{Glo7:Chaypaqa}{}{amv}%
{Chaypaqa wiñaraptinqa, ¿\pb{ayka} \pb{puntrawnintataq} riganchik?}%amv que first line
{\morglo{chay-pa-qa}{\lsc{dem.d}-\lsc{loc}-\lsc{top}}\morglo{wiña-ra-pti-n-qa}{grow-\lsc{unint}-\lsc{subds}-\lsc{3}-\lsc{top}}\morglo{ayka}{how.many}\morglo{puntraw-ni-n-ta-taq}{day-\lsc{euph}-\lsc{3}-\lsc{acc}-\lsc{seq}}\morglo{riga-nchik}{irrigate-\lsc{1pl}}}%morpheme+gloss
\glotran{When it grows, at \pb{how many} days do you water it?}{}%eng+spa trans
{}{}%rec - time

Enclitics are not employed in the interior of a subordinate clause but may attach to the final word in the clause (¿\phono{\pb{Pi}} \phono{mishi-ta} \phono{saru-ri-sa-n-ta-\pb{taq}} \phono{qawa-rqa-nki}? ‘Who did you see trample the cat?’).

\section{Reflexives and reciprocals}\index[sub]{sentence!reciprocals}
This section partially repeats §~\ref{par:reflexive} and §~\ref{par:reciprocal} on \phono{-ku}, and \phono{-na} Please consult those sections for further discussion and examples. \SYQ{} employs the verb-verb derivational suffixes \phono{-kU} and \phono{-nakU} to indicate reflexive and reciprocal action, respectively.\\

\noindent
\phono{-kU} may indicate that the subject acts on himself/herself or that the subject of the verb is the object of the event referred to, i.e., \phono{-kU} derives verbs with the meanings ‘V one’s self’~(\ref{Glo7:Kikinpis}), (\ref{Glo7:Kundina}), and ‘be Ved’~(\ref{Glo7:nchik}), (\ref{Glo7:Pampa}). Note that \phono{-kU} is not restricted to forming reflexives and may also indicate pseudo-reflexives, middles, medio-passives and passives.\\

% 1
\gloexe{Glo7:Kikinpis}{}{amv}%
{Kikinpis Campiona\pb{ku}run.}%amv que first line
{\morglo{kiki-n-pis}{self-\lsc{3}-\lsc{add}}\morglo{Campiona-ku-ru-n}{poison.with.Campión-\lsc{refl}-\lsc{urgt}-\lsc{3}}}%morpheme+gloss
\glotran{They themselves \pb{Campioned themselves} [took Campion rat poison].}{}%eng+spa trans
{}{}%rec - time

% 2
\gloexe{Glo7:Kundina}{}{amv}%
{Kundina\pb{ku}rushpa chay pashña kaqta trayaramun.}%amv que first line
{\morglo{kundina-ku-ru-shpa}{condemn-\lsc{refl}-\lsc{urgt}-\lsc{subis}}\morglo{chay}{\lsc{dem.d}}\morglo{pashña}{girl}\morglo{ka-q-ta}{be-\lsc{ag}-\lsc{acc}}\morglo{traya-ra-mu-n}{arrive-\lsc{urgt}-\lsc{cisl}-\lsc{3}}}%morpheme+gloss
\glotran{Condemning himself [turning into a zombie], he arrived at the girl’s place.}{}%eng+spa trans
{}{}%rec - time

% 3
\gloexe{Glo7:nchik}{}{amv}%
{Mancha\pb{ku}nchik runa wañuypaq kaptin.}%amv que first line
{\morglo{mancha-ku-nchik}{scare-\lsc{refl}-\lsc{1pl}}\morglo{runa}{person}\morglo{wañu-y-paq}{die-\lsc{inf}-\lsc{purp}}\morglo{ka-pti-n}{be-\lsc{subds}-\lsc{3}}}%morpheme+gloss
\glotran{We \pb{get scared} when people are about to die.}{}%eng+spa trans
{}{}%rec - time

% 4
\gloexe{Glo7:Pampa}{}{amv}%
{Pampa\pb{ku}run chayshi.}%amv que first line
{\morglo{pampa-ku-ru-n}{bury-\lsc{refl}-\lsc{urgt}-\lsc{3}}\morglo{chay-shi}{\lsc{dem.d}-\lsc{evr}}}%morpheme+gloss
\glotran{He \pb{was buried}, they say.}{}%eng+spa trans
{}{}%rec - time

\noindent
\phono{-na} indicates that two or more actors act reflexively on each other, i.e.,  \phono{-na} derives verbs with the meaning ‘V each other’~(\ref{Glo7:Unayqa}), (\ref{Glo7:ValiPaqarin}).\\

% 5
\gloexe{Glo7:Unayqa}{}{amv}%
{Unayqa chay nishpa willa\pb{naku}n.}%amv que first line
{\morglo{unay-qa}{before-\lsc{top}}\morglo{chay}{\lsc{dem.d}}\morglo{ni-shpa}{say-\lsc{subis}}\morglo{willa-naku-n}{tell-\lsc{recip}-\lsc{3}}}%morpheme+gloss
\glotran{Formerly, saying that, we told \pb{each other}.}{}%eng+spa trans
{}{}%rec - time

% 6
\gloexe{Glo7:ValiPaqarin}{}{ch}%
{Vali\pb{naku}:. ‘Paqarin yanapamay u paqarin ñuqakta chaypaq talpashun qampaktañataq’, ni\pb{naku}:mi.}%ch que first line
{\morglo{vali-naku-:}{solicit-\lsc{recip}-\lsc{1}}\morglo{paqarin}{tomorrow}\morglo{yanapa-ma-y}{help-\lsc{1.obj}-\lsc{imp}}\morglo{u}{or}\morglo{paqarin}{tomorrow}\morglo{ñuqa-kta}{I-\lsc{acc}}\morglo{chay-paq}{\lsc{dem.d}-\lsc{abl}}\morglo{talpa-shun}{plow-\lsc{1pl.fut}}\morglo{qam-pa-kta-ña-taq}{you-\lsc{gen}-\lsc{acc}-\lsc{disc}-\lsc{seq}}\morglo{ni-naku-:-mi}{say-\lsc{recip}-\lsc{1}-\lsc{evd}}}%morpheme+gloss
\glotran{We solicit \pb{each other}, “Help me tomorrow,” or, “Tomorrow me and then we’ll plant yours,” we say to \pb{each other}.}{}%eng+spa trans
{}{}%rec - time

\noindent
\phono{-na} is dependent and never appears independent of \phono{-kU}. \phono{-chinakU} derives verbs with the meaning ‘cause each other to V’~(\ref{Glo7:chinaku}), (\ref{Glo7:Kukankunata}).\\

% 7
\gloexe{Glo7:chinaku}{}{amv}%
{Yuyari\pb{chinaku}yan.}%amv que first line
{\morglo{yuya-ri-chi-naku-ya-n}{remember-\lsc{incep}-\lsc{caus}-\lsc{recip}-\lsc{prog}-\lsc{3}}}%morpheme+gloss
\glotran{They’re \pb{making} \pb{each other} remember.}{}%eng+spa trans
{}{}%rec - time

% 8
\gloexe{Glo7:Kukankunata}{}{amv}%
{Kukankunata tragunkunata muyuyka\pb{chinaku}shpa.}%amv que first line
{\morglo{kuka-n-kuna-ta}{coca-\lsc{3}-\lsc{pl}-\lsc{acc}}\morglo{tragu-n-kuna-ta}{drink-\lsc{3}-\lsc{pl}-\lsc{acc}}\morglo{muyu-yka-chi-naku-shpa}{circle-\lsc{excep}-\lsc{caus}-\lsc{recip}-\lsc{subis}}}%morpheme+gloss
\glotran{\pb{Making} their coca and liquor circulate \pb{among themselves}.}{}%eng+spa trans
{}{}%rec - time

Preceding any of the derivational suffixes \phono{-mu}, \phono{-ykU}, or \phono{-chi} or the inflectional suffix \phono{-ma}, \phono{-(chi-na)-kU} is realized as \phono{-(chi-na)-ka}.

\section{Equatives}\label{sec:equative}\index[sub]{sentence!equatives}
This section partially repeats §~\ref{ssec:copu} on equative verbs Please consult that section for further discussion and examples. \SYQ{} counts a single copulative verb, \phono{ka-}. Like the English verb \phono{be}, \phono{ka-} has both copulative~(\ref{Glo7:fwirti}), (\ref{Glo7:Qammi}) and existential~(\ref{Glo7:Kanna}), (\ref{Glo7:Rantiqpis}) interpretations. \phono{ka-} is irregular: its third person singular present tense form, \phono{ka-n}, never appears in equational statements, but only in existential statements. ‘This is a llama’ would be translated \phono{Kay-qa} \phono{llama-m}, while ‘There are llamas’ would be translated \phono{Llama-qa ka-n-mi}.\\

% 1
\gloexe{Glo7:fwirti}{}{amv}%
{Ñuqa-nchik fwirti \pb{ka}nchik patachita, matrkata, trakranchik lluqsiqta mikushpam.}%amv que first line
{\morglo{ñuqa-nchik}{I-\lsc{1pl}}\morglo{fwirti}{strong}\morglo{ka-nchik}{be-\lsc{1pl}}\morglo{patachi-ta}{wheat.soup-\lsc{acc}}\morglo{matrka-ta}{ground.cereal.meal-\lsc{acc}}\morglo{trakra-nchik}{field-\lsc{1pl}}\morglo{lluqsi-q-ta}{come.out-\lsc{ag}-\lsc{acc}}\morglo{miku-shpa-m}{eat-\lsc{subis}-\lsc{evd}}}%morpheme+gloss
\glotran{We \pb{are} strong because we eat what comes out of our fields -- wheat soup and toasted grain.}{}%eng+spa trans
{}{}%rec - time

% 2
\gloexe{Glo7:Qammi}{}{amv}%
{Qammi salvasyunniy \pb{ka}nki.}%amv que first line
{\morglo{qam-mi}{you-\lsc{evd}}\morglo{salvasyun-ni-y}{salvation-\lsc{euph}-\lsc{1}}\morglo{ka-nki}{be-\lsc{2}}}%morpheme+gloss
\glotran{You \pb{are} my salvation.}{}%eng+spa trans
{}{}%rec - time

% 3
\gloexe{Glo7:Kanna}{}{amv}%
{\pb{Kan}ña piña turu.}%amv que first line
{\morglo{ka-n-ña}{be-\lsc{3}-\lsc{disc}}\morglo{piña}{angry}\morglo{turu}{bull}}%morpheme+gloss
\glotran{\pb{There are} mean bulls.}{}%eng+spa trans
{}{}%rec - time

% 4
\gloexe{Glo7:Rantiqpis}{}{amv}%
{Rantiqpis \pb{kan}taqmi.}%amv que first line
{\morglo{ranti-q-pis}{buy-\lsc{ag}-\lsc{add}}\morglo{ka-n-taq-mi}{be-\lsc{3}-\lsc{seq}-\lsc{evd}}}%morpheme+gloss
\glotran{\pb{There are} also buyers.}{}%eng+spa trans
{}{}%rec - time

\noindent
Evidentials (\phono{-mI}, \phono{-shI} and \phono{-trI}) often attach to the predicate in equational statements without \phono{ka-n}~(\ref{Glo7:Vakay}), (\ref{Glo7:Llutan}).\\

% 5
\gloexe{Glo7:Vakay}{}{amv}%
{Vakay wira wira\pb{m} matraypi puñushpa, allin pastuta mikushpam.}%amv que first line
{\morglo{vaka-y}{cow-\lsc{1}}\morglo{wira}{fat}\morglo{wira-m}{fat-\lsc{evd}}\morglo{matray-pi}{cave-\lsc{loc}}\morglo{puñu-shpa}{sleep-\lsc{subis}}\morglo{allin}{good}\morglo{pastu-ta}{pasture.grass-\lsc{acc}}\morglo{miku-shpam}{eat-\lsc{subis}}}%morpheme+gloss
\glotran{Sleeping in a cave and eating good pasture, my cow \pb{is} really fat.}{}%eng+spa trans
{}{}%rec - time

% 6
\gloexe{Glo7:Llutan}{}{lt}%
{Llutan\pb{shi}ki.}%lt que first line
{\morglo{llutan-sh-iki}{deformed-\lsc{evr}-\lsc{iki}}}%morpheme+gloss
\glotran{\pb{They are} deformed, they say.}{}%eng+spa trans
{}{}%rec - time

\noindent
The principal strategy in \SYQ{} for constructing equational statements is to employ the continuous form \phono{ka\pb{-ya}-n}~(\ref{Glo7:Alpakachu}).\\

% 7
\gloexe{Glo7:Alpakachu}{}{amv}%
{¿Alpakachu wak \pb{kaya}n?}%amv que first line
{\morglo{alpaka-chu}{alpaca-\lsc{q}}\morglo{wak}{\lsc{dem.d}}\morglo{ka-ya-n}{be-\lsc{prog}-\lsc{3}}}%morpheme+gloss
\glotran{\pb{Is that} alpaca [wool]?}{}%eng+spa trans
{}{}%rec - time

\section{Possession}\index[sub]{sentence!possession}
This section partially repeats §~\ref{ssec:alloP} on possession Please consult that section for further discussion and glossed examples. \SYQ{} employs the suffixes of the nominal paradigm to indicate possession. These are the same in all dialects for all persons except the first person singular. Two of the five dialects --~\AMV{} and \LT~-- follow the \QII{} pattern, marking the first person singular with \phono{-y}; three dialects --~\ACH, \CH, and \SP~-- follow the \QI{} pattern marking it with \phono{-:} (vowel length). The \SYQ{} nominal suffixes, then, are: \phono{-y} or \phono{-:} (1\lsc{p}), \phono{-Yki} (2\lsc{p}), \phono{-n} (3\lsc{p}), \phono{-nchik} (1\lsc{pl})~(\ref{Glo7:Wiqawni}--\ref{Glo7:Chayna}). Table~\ref{Tab10} displays this paradigm.\\

% 1
\gloexe{Glo7:Wiqawni}{}{amv}%
{Wiqawni\pb{y}mi nanan.}%amv que first line
{\morglo{wiqaw-ni-y-mi}{waist-\lsc{euph}-\lsc{1}-\lsc{evd}}\morglo{nana-n}{hurt-\lsc{3}}}%morpheme+gloss
\glotran{\pb{My} lower back hurts.}{}%eng+spa trans
{}{}%rec - time

% 2
\gloexe{Glo7:Qusa}{}{ach}%
{Qusa\pb{:}ta listaman trurarusa.}%ach que first line
{\morglo{qusa-:-ta}{husband-\lsc{1}-\lsc{acc}}\morglo{lista-man}{list-\lsc{all}}\morglo{trura-ru-sa}{put-\lsc{urgt}-\lsc{npst}}}%morpheme+gloss
\glotran{They put \pb{my} husband on the list.}{}%eng+spa trans
{}{}%rec - time

% 3
\gloexe{Glo7:Kimsan}{}{amv}%
{Kimsan wambra\pb{yki}kuna takikuyan.}%amv que first line
{\morglo{kimsa-n}{three-\lsc{3}}\morglo{wambra-yki-kuna}{child-\lsc{2}-\lsc{pl}}\morglo{taki-ku-ya-n}{sing-\lsc{refl}-\lsc{prog}-\lsc{3}}}%morpheme+gloss
\glotran{The three of \pb{your} children are singing.}{}%eng+spa trans
{}{}%rec - time

% 4
\gloexe{Glo7:Maypish}{}{ch}%
{¿Maypish wasi\pb{n}ta lulayan?}%ch que first line
{\morglo{may-pi-sh}{where-\lsc{loc}-\lsc{evr}}\morglo{wasi-n-ta}{house-\lsc{3}-\lsc{acc}}\morglo{lula-ya-n}{make-\lsc{prog}-\lsc{3}}}%morpheme+gloss
\glotran{Where [did she say she] is making \pb{her} house?}{}%eng+spa trans
{}{}%rec - time

% 5
\gloexe{Glo7:Chayna}{}{ach}%
{Chayna achka wambra\pb{nchik}ta familya\pb{nchik}kunata aparun.}%ach que first line
{\morglo{chayna}{thus}\morglo{achka}{a.lot}\morglo{wambra-nchik-ta}{child-\lsc{1pl}-\lsc{acc}}\morglo{familya-nchik-kuna-ta}{family-\lsc{1pl}-\lsc{pl}-\lsc{acc}}\morglo{apa-ru-n}{bring-\lsc{urgt}-\lsc{3}}}%morpheme+gloss
\glotran{So they took away lots of \pb{our} children, our relatives.}{}%eng+spa trans
{}{}%rec - time

\noindent
In the case of words ending in a consonant, \phono{-ni} --~semantically vacuous~-- precedes the person suffix~(\ref{Glo7:Gana}).\\

% 6
\gloexe{Glo7:Gana}{}{ach}%
{Gana\pb{wnin}ta qatikura qalay qalay.}%ach que first line
{\morglo{ganaw-ni-n-ta}{cattle-\lsc{euph}-\lsc{3}-\lsc{acc}}\morglo{qati-ku-ra}{follow-\lsc{refl}-\lsc{pst}}\morglo{qalay}{all}\morglo{qalay}{all}}%morpheme+gloss
\glotran{They herded \pb{their} cattle, absolutely all.}{}%eng+spa trans
{}{}%rec - time

\noindent
\SYQ{} “have” constructions are formed \lsc{substantive-poss} \phono{ka-}~(\ref{Glo7:wambrayki}).\\

% 7
\gloexe{Glo7:wambrayki}{}{ach}%
{Mana \pb{wambrayki kan}chu mana \pb{qariyki kan}chu.}%ach que first line
{\morglo{mana}{no}\morglo{wambra-yki}{child-\lsc{2}}\morglo{ka-n-chu}{be-\lsc{3}-\lsc{neg}}\morglo{mana}{no}\morglo{qari-yki}{man-\lsc{2}}\morglo{ka-n-chu}{be-\lsc{3}-\lsc{neg}}}%morpheme+gloss
\glotran{\pb{You} don’t \pb{have children}, \pb{you} don’t \pb{have a husband}.}{}%eng+spa trans
{}{}%rec - time

\noindent
In case a noun or pronoun referring to the possessor appears in the same clause, the noun or pronoun is case-marked genitive with either \phono{-pa}, \phono{-pi}, or \phono{-paq}~(\ref{Glo7:wallqa}), (\ref{Glo7:Asnuqa}).\footnote{An anonymous reviewer points out that possessive constructions are formed differently in \QI{}: “The possessed item takes a possessive suffix and the copula takes -pU followed by an object suffix that agrees with the person of the possessor. In other words, the verbal object suffix and the possessive suffix refer to the same person.” The reviewer offers the following examples:\\
\phono{Ishkay wa:ka-: ka-pa-ma-n.} ‘I have two cows.’\\
\phono{Ishkay wa:ka-yki ka-pu-shu-nki.} ‘You have two cows.’\\
\phono{Ishkay wa:ka-n ka-pu-n (or ka-n).} ‘She has two cows.’}\\

% 8
\gloexe{Glo7:wallqa}{}{amv}%
{Duyñupa wallqa\pb{n}ta ruwan.}%amv que first line
{\morglo{duyñu-pa}{owner-\lsc{gen}}\morglo{wallqa-n-ta}{garland-\lsc{3}-\lsc{acc}}\morglo{ruwa-n}{make-\lsc{3}}}%morpheme+gloss
\glotran{They make the owner \pb{his} \phono{wallqa} (garland).}{}%eng+spa trans
{}{}%rec - time

% 9
\gloexe{Glo7:Asnuqa}{}{sp}%
{Asnuqa hatarishpash ripukun chay runa\pb{pa} wasi\pb{n}man.}%sp que first line
{\morglo{asnu-qa}{donkey-\lsc{top}}\morglo{hatari-shpa-sh}{get.up-\lsc{subis}-\lsc{evr}}\morglo{ripu-ku-n}{go-\lsc{refl}-\lsc{3}}\morglo{chay}{\lsc{dem.d}}\morglo{runa-pa}{person-\lsc{gen}}\morglo{wasi-n-man}{house-\lsc{3}-\lsc{all}}}%morpheme+gloss
\glotran{Geting up, the donkey went to the man\pb{’s} house.}{}%eng+spa trans
{}{}%rec - time

\section{Topic}\index[sub]{sentence!topicalization}
This section partially repeats §~\ref{ssec:topic} on \phono{-qa}. Please consult that section for further discussion and glossed examples. \SYQ{} uses the enclitic \phono{-qa} to mark topic.\\

% 1
\gloexe{Glo7:Ganawniyki}{}{lt}%
{Ganawniyki\pb{qa} achkam miranqa.}%lt que first line
{\morglo{qanaw-ni-yki-qa}{cattle-\lsc{euph}-\lsc{2}-\lsc{top}}\morglo{achka-m}{a.lot-\lsc{evd}}\morglo{mira-nqa}{increase-\lsc{3.fut}}}%morpheme+gloss
\glotran{Your \pb{cattle} are going to multiply a lot.}{}%eng+spa trans
{}{}%rec - time

% 2
\gloexe{Glo7:Chaynam}{}{amv}%
{Chaynam unay\pb{qa} manam imapis kaptinqa.}%amv que first line
{\morglo{chayna-m}{thus-\lsc{evd}}\morglo{unay-qa}{before-\lsc{top}}\morglo{mana-m}{no-\lsc{evd}}\morglo{ima-pis}{what-\lsc{add}}\morglo{ka-pti-n-qa}{be-\lsc{subds}-\lsc{3}-\lsc{top}}}%morpheme+gloss
\glotran{That’s how it was \pb{before} when there wasn’t anything.}{}%eng+spa trans
{}{}%rec - time

% 3
\gloexe{Glo7:munasanchik}{}{amv}%
{Kanan\pb{qa} mikun munasanchik qullqi kaptinqa.}%amv que first line
{\morglo{kanan-qa}{now-\lsc{top}}\morglo{miku-n}{eat-\lsc{3}}\morglo{muna-sa-nchik}{want-\lsc{prf}-\lsc{1pl}}\morglo{qullqi}{money}\morglo{ka-pti-n-qa}{be-\lsc{subds}-\lsc{3}-\lsc{top}}}%morpheme+gloss
\glotran{\pb{Now} we eat whatever we want when there’s money.}{}%eng+spa trans
{}{}%rec - time

% 4
\gloexe{Glo7:Llaqtaykipa}{}{amv}%
{Llaqtaykipa\pb{qa} ¿tarpunkichu sibadata?}%amv que first line
{\morglo{llaqta-yki-pa-qa}{town-\lsc{2}-\lsc{loc}-\lsc{top}}\morglo{tarpu-nki-chu}{plant-\lsc{2}-\lsc{q}}\morglo{sibada-ta}{barley-\lsc{acc}}}%morpheme+gloss
\glotran{In \pb{your town}, do you plant barley?}{}%eng+spa trans
{}{}%rec - time

\section{Focus}\label{sec:emphasis}\index[sub]{sentence!emphasis}
In \SYQ, it is the evidentials, \phono{-mI}, \phono{-shI}, and \phono{-trI}, that, by virtue of their placement, indicate focus or comment. For example, in~(\ref{Glo7:Paqarin}), the evidential attaches to the direct object, \phono{shakash} ‘guinea pig’, and it is that element that is stressed: it is a \pb{guinea pig} that you are going to butcher tomorrow. In~(\ref{Glo7:qamqa}) the evidential attaches to the temporal noun \phono{paqarin} ‘tomorrow’, with the resulting interpretation: it is \pb{tomorrow} that you are going to butcher a guinea pig. Evidentials never attach to the topic or subject. Topic and subject are, rather, marked with \phono{-qa}, as is \phono{qam} in~(\ref{Glo7:Paqarin}) and~(\ref{Glo7:qamqa}).\\

% 1
\gloexe{Glo7:Paqarin}{}{amv}%
{Paqarin qamqa shakashta\pb{tr} wañuchinki.~\updag}% que first line
{\morglo{paqarin}{tomorrow}\morglo{qam-qa}{you-\lsc{top}}\morglo{shakash-ta-tr}{guinea.pig-\lsc{acc}-\lsc{evc}}\morglo{wañu-chi-nki}{die-\lsc{caus}-\lsc{2}}}%morpheme+gloss
\glotran{Tomorrow you’ll kill a \pb{guinea pig}\tss{F}.}{}%eng+spa trans
{}{}%rec - time

% 2
\gloexe{Glo7:qamqa}{}{amv}%
{Paqarin\pb{tri} qamqa shakashta wañuchinki.~\updag}% que first line
{\morglo{paqarin-tri}{tomorrow-\lsc{evc}}\morglo{qam-qa}{you-\lsc{top}}\morglo{shakash-ta}{guinea.pig-\lsc{acc}}\morglo{wañu-chi-nki}{die-\lsc{caus}-\lsc{2}}}%morpheme+gloss
\glotran{\pb{Tomorrow}\tss{F} you’ll kill a guinea pig.}{}%eng+spa trans
{}{}%rec - time

\section{Complementization (infinitive, agentive, indicative and subjunctive clauses)}\label{sec:comple}\index[sub]{sentence!complementation}
This section partially repeats §~\ref{ssec:sdfv} on substantives derived from verbs Please consult that section for further discussion and glossed examples. \SYQ{} forms infinitive complements with \phono{-y}~(\ref{Glo7:Munankichu}--\ref{Glo7:Algunus}), purposive complements with \phono{-q}~(\ref{Glo7:Misa}), (\ref{Glo7:Pasaruptin}), indicative complements with \phono{-sHa}~(\ref{Glo7:Atipa}--\ref{Glo7:wambran}), and subjunctive complements with \phono{-na}~(\ref{Glo7:Puchukananta}). Infinitive complements often figure as the object of the verbs ~\phono{muna-} ‘want’~(\ref{Glo7:Munankichu}), \phono{atipa-} ‘be able’~(\ref{Glo7:Lukuyarun}), and \phono{gusta-} ‘like’~(\ref{Glo7:Algunus}). Indicative complements are common with the verbs \phono{yatra-} ‘know’~(\ref{Glo7:maypa}), (\ref{Glo7:Kwirpu}), \phono{qunqa-} ‘forget’, \phono{qawa} ‘see’~(\ref{Glo7:wambran}), and \phono{uyari-} ‘hear’. Note that infinitive complements are case-marked with accusative \phono{-ta} and that \phono{-q} purposive complements only occur with verbs of movement (\phono{-na-}(\lsc{poss})\phono{-paq}, being used for other verb types~(\ref{Glo7:pulluyki}) (see~§~\ref{sssc:con})).\\

% 1
\gloexe{Glo7:Munankichu}{}{amv}%
{¿Munankichu sintachi\pb{y}\pb{ta}qa?}%amv que first line
{\morglo{muna-nki-chu}{want-\lsc{2}-\lsc{q}}\morglo{sintachi-y-ta-qa}{put.ribbons-\lsc{inf}-\lsc{acc}-\lsc{top}}}%morpheme+gloss
\glotran{Do you \pb{want to}? To piece their ears with ribbons?}{}%eng+spa trans
{}{}%rec - time

% 2
\gloexe{Glo7:Lukuyarun}{}{ach}%
{Lukuyarun runalla. Manam puñu\pb{yta} \pb{atipa}rachu.}%ach que first line
{\morglo{luku-ya-ru-n}{crazy-\lsc{inch}-\lsc{urgt}-\lsc{3}}\morglo{runa-lla}{person-\lsc{rstr}}\morglo{mana-m}{no-\lsc{evd}}\morglo{puñu-y-ta}{sleep-\lsc{inf}-\lsc{acc}}\morglo{atipa-ra-chu}{be.able-\lsc{pst}-\lsc{neg}}}%morpheme+gloss
\glotran{My husband was going crazy. He \pb{couldn’t} sleep.}{}%eng+spa trans
{}{}%rec - time

% 3
\gloexe{Glo7:Algunus}{}{amv}%
{Algunus turuqa runa waqra\pb{yta} \pb{gusta}n.}%amv que first line
{\morglo{algunus}{some}\morglo{turu-qa}{bull-\lsc{top}}\morglo{runa}{person}\morglo{waqra-y-ta}{horn-\lsc{inf}-\lsc{acc}}\morglo{gusta-n}{like-\lsc{3}}}%morpheme+gloss
\glotran{Some bulls \pb{like to} gore people.}{}%eng+spa trans
{}{}%rec - time

% 4
\gloexe{Glo7:Misa}{}{ch}%
{Misa lulaq \pb{shamu}n.}%ch que first line
{\morglo{misa}{mass}\morglo{lula-q}{make-\lsc{ag}}\morglo{shamu-n}{come-\lsc{3}}}%morpheme+gloss
\glotran{They \pb{come} to hold mass.}{}%eng+spa trans
{}{}%rec - time

% 5
\gloexe{Glo7:Pasaruptin}{}{amv}%
{Pasaruptin qawa\pb{q} \pb{hamu}ni.}%amv que first line
{\morglo{pasa-ru-pti-n}{pass-\lsc{urgt}-\lsc{subds}-\lsc{3}}\morglo{qawa-q}{see-\lsc{ag}}\morglo{hamu-ni}{come-\lsc{1}}}%morpheme+gloss
\glotran{When that happened, I \pb{came to see}.}{}%eng+spa trans
{}{}%rec - time

% 6
\gloexe{Glo7:Atipa}{}{ach}%
{Atipa\pb{sa}ntatriki ruwan.}%ach que first line
{\morglo{atipa-sa-n-ta-tr-iki}{be.able-\lsc{prf}-\lsc{3}-\lsc{acc}-\lsc{evc}-\lsc{iki}}\morglo{ruwa-n}{make-\lsc{3}}}%morpheme+gloss
\glotran{They do \pb{what they can}.}{}%eng+spa trans
{}{}%rec - time

% 7
\gloexe{Glo7:maypa}{}{ach}%
{Ni maypa ka\pb{sa}ntapis yatra:chu. Waqaku:.}%ach que first line
{\morglo{ni}{nor}\morglo{may-pa}{where-\lsc{loc}}\morglo{ka-sa-n-ta-pis}{be-\lsc{prf}-\lsc{3}-\lsc{acc}-\lsc{add}}\morglo{yatra-:-chu}{know-\lsc{1}-\lsc{neg}}\morglo{waqa-ku-:}{cry-\lsc{refl}-\lsc{1}}}%morpheme+gloss
\glotran{I don’t even know where \pb{he is}. I cry.}{}%eng+spa trans
{}{}%rec - time

% 8
\gloexe{Glo7:Kwirpu}{}{ch}%
{Kwirpu: yatran imapaq kayna puli\pb{sha}:tapis.}%ch que first line
{\morglo{kwirpu-:}{body-\lsc{1}}\morglo{yatra-n}{know-\lsc{3}}\morglo{ima-paq}{what-\lsc{purp}}\morglo{kayna}{thus}\morglo{puli-sha-:-ta-pis}{walk-\lsc{prf}-\lsc{1}-\lsc{acc}-\lsc{add}}}%morpheme+gloss
\glotran{My body knows \pb{why I walk around} like this.}{}%eng+spa trans
{}{}%rec - time

% 9
\gloexe{Glo7:wambran}{}{amv}%
{Ñuqaqa wambran \pb{qipikusan}ta qawarqanichu.}%amv que first line
{\morglo{ñuqa-qa}{I-\lsc{top}}\morglo{wambra-n}{child-\lsc{3}}\morglo{qipi-ku-sa-n-ta}{carry-\lsc{refl}-\lsc{prf}-\lsc{3}-\lsc{acc}}\morglo{qawa-rqa-ni-chu}{see-\lsc{pst}-\lsc{1}-\lsc{neg}}}%morpheme+gloss
\glotran{I didn’t see \pb{that she carried} her baby.}{}%eng+spa trans
{}{}%rec - time

% 10
\gloexe{Glo7:Puchukananta}{}{amv}%
{Puchuka\pb{na}nta munani.}%amv que first line
{\morglo{puchuka-na-n-ta}{finish-\lsc{nmlz}-\lsc{3}-\lsc{acc}}\morglo{muna-ni}{want-\lsc{1}}}%morpheme+gloss
\glotran{I want \pb{them to finish}.}{}%eng+spa trans
{}{}%rec - time

% 11
\gloexe{Glo7:pulluyki}{}{amv}%
{¡Uqi pulluyki qawachi\pb{naypaq} kaynam ruwasay!}%amv que first line
{\morglo{uqi}{grey}\morglo{pullu-yki}{shawl-\lsc{2}}\morglo{qawa-chi-na-y-paq}{see-\lsc{caus}-\lsc{nmlz}-\lsc{1}-\lsc{purp}}\morglo{kayna-m}{thus-\lsc{evd}}\morglo{ruwa-sa-y}{make-\lsc{prf}-\lsc{1}}}%morpheme+gloss
\glotran{[Bring] your grey manta \pb{so I can} show it to her. What I make is like this.}{}%eng+spa trans
{}{}%rec - time

\section{Relativization}\index[sub]{sentence!relativization}
This section partially repeats §~\ref{ssec:sdfv} on substantives derived from verbs Please consult that section for further discussion and glossed examples. \SYQ{} forms relative clauses with the four deverbalizing suffixes: concretizing \phono{-na}~(\ref{Glo7:puntraw}), agentive \phono{-q}~(\ref{Glo7:Rigakuq}), perfective \phono{-sHa}~(\ref{Glo7:Nuqaqa}), and infinitive \phono{-y}~(\ref{Glo7:vilakuy}). As these structures are formally nouns, they are inflected with substantive suffixes, not verbal suffixes (\phono{ranti-sa-\pb{yki}} \phono{*ranti-sa-\pb{nki}} ‘that you sold’)~(\ref{Glo7:Rigalakullaq}). The inflected forms may be reinforced with possessive pronouns~(\ref{Glo7:rantikurasa}). \phono{-sHa} may additionally form nouns referring to the location where~(\ref{Glo7:fwirapi} or time at which~(\ref{Glo7:Urqupa}) an event \lsc{E} occurred. \phono{-sHa} is realized as \phono{-sa} in \ACH{}~(\ref{Glo7:Rigalakullaq}), \AMV{}~(\ref{Glo7:Pampaykuni}) and \SP{}~(\ref{Glo7:mayman}) and as \phono{-sha} in \LT{}~(\ref{Glo7:Kalamina}) and \CH. Any substantive constituent --~subject~(\ref{Glo7:Rigakuq}), object~(\ref{Glo7:Pampaykuni}), or complement~(\ref{Glo7:puntraw})~-- can be relativized. Nominalizing suffixes attach directly to the verb stem, with the exception that the person suffixes \phono{-wa/-ma} (first person object) and \phono{-sHu} (second person object) may intercede~(\ref{Glo7:Ampullakta}),~(\ref{Glo7:Filupa}).\\

% 1
\gloexe{Glo7:puntraw}{}{lt}%
{Asta \pb{wañukunay puntraw}kamatriki chayna purishaq.}%lt que first line
{\morglo{asta}{until}\morglo{wañu-ku-na-y}{die-\lsc{refl}-\lsc{nmlz}-\lsc{1}}\morglo{puntraw-kama-tr-iki}{day-\lsc{lim}-\lsc{evc}-\lsc{iki}}\morglo{chay-na}{thus}\morglo{puri-shaq}{walk-\lsc{1.fut}}}%morpheme+gloss
\glotran{Until \pb{the day I die}, I’m going to walk around like that.}{}%eng+spa trans
{}{}%rec - time

% 2
\gloexe{Glo7:Rigakuq}{}{ch}%
{\pb{Rigakuq luna} trabahaya:.}%ch que first line
{\morglo{riga-ku-q}{irrigate-\lsc{refl}-\lsc{ag}}\morglo{luna}{person}\morglo{trabaha-ya-:}{work-\lsc{prog}-\lsc{1}}}%morpheme+gloss
\glotran{\pb{The people who water}, we’re working.}{}%eng+spa trans
{}{}%rec - time

% 3
\gloexe{Glo7:Nuqaqa}{}{sp}%
{Ñuqaqa manam rimayta yatrara:chu prufusurni: nima\pb{sa}nta.}%sp que first line
{\morglo{ñuqa-qa}{I-\lsc{top}}\morglo{mana-m}{no-\lsc{evd}}\morglo{rima-y-ta}{talk-\lsc{inf}-\lsc{acc}}\morglo{yatra-ra-:-chu}{know-\lsc{pst}-\lsc{1}-\lsc{neg}}\morglo{prufusur-ni-:}{teacher-\lsc{euph}-\lsc{1}}\morglo{ni-ma-sa-n-ta}{say-\lsc{1.obj}-\lsc{prf}-\lsc{3}-\lsc{acc}}}%morpheme+gloss
\glotran{I didn’t know how to say \pb{what my teacher said to me}.}{}%eng+spa trans
{}{}%rec - time

% 4
\gloexe{Glo7:vilakuy}{}{amv}%
{Chay \pb{vilakuy puntraw} simintiryupa.}%amv que first line
{\morglo{chay}{\lsc{dem.d}}\morglo{vila-ku-y}{candle-\lsc{refl}-\lsc{inf}}\morglo{puntraw}{day}\morglo{simintiryu-pa}{cemetery-\lsc{loc}}}%morpheme+gloss
\glotran{The \pb{day we lit candles} in the cemetery.}{}%eng+spa trans
{}{}%rec - time

% 5
\gloexe{Glo7:Rigalakullaq}{}{ach}%
{Rigalakullaq ka: \pb{mana rantikusa:taqa}.}%ach que first line
{\morglo{rigala-ku-lla-q}{give.as.a.gift-\lsc{refl}-\lsc{rstr}-\lsc{ag}}\morglo{ka-:}{be-\lsc{1}}\morglo{mana}{no}\morglo{ranti-ku-sa-:-ta-qa}{buy-\lsc{refl}-\lsc{prf}-\lsc{1}-\lsc{acc}-\lsc{top}}}%morpheme+gloss
\glotran{I used to give away \pb{what I didn’t sell}.}{}%eng+spa trans
{}{}%rec - time

% 6
\gloexe{Glo7:rantikurasa}{}{amv}%
{Qampa \pb{rantikurasa}ykiyá chay shakash.}%amv que first line
{\morglo{qam-pa}{you-\lsc{gen}}\morglo{ranti-ku-ra-sa-yki-yá}{buy-\lsc{refl}-\lsc{unint}-\lsc{prf}-\lsc{2}-\lsc{emph}}\morglo{chay}{\lsc{dem.d}}\morglo{shakash}{guinea.pig}}%morpheme+gloss
\glotran{That guinea pig \pb{that \emph{you} sold}.}{}%eng+spa trans
{}{}%rec - time

% 7
\gloexe{Glo7:fwirapi}{}{amv}%
{Chay fwirapi chay vilakuna rantiku\pb{sa}n.}%amv que first line
{\morglo{chay}{\lsc{dem.d}}\morglo{fwira-pi}{outside-\lsc{loc}}\morglo{chay}{\lsc{dem.d}}\morglo{vila-kuna}{candle-\lsc{pl}}\morglo{ranti-ku-sa-n}{buy-\lsc{refl}-\lsc{prf}-\lsc{3}}}%morpheme+gloss
\glotran{That’s outside \pb{where they sell} candles.}{}%eng+spa trans
{}{}%rec - time

% 8
\gloexe{Glo7:Urqupa}{}{amv}%
{Urqupa kaya\pb{sa}nchikpis.}%amv que first line
{\morglo{urqu-pa}{hill-\lsc{loc}}\morglo{ka-ya-sa-nchik-pis}{be-\lsc{prog}-\lsc{prf}-\lsc{1pl}-\lsc{add}}}%morpheme+gloss
\glotran{\pb{When} we were in the mountains.}{}%eng+spa trans
{}{}%rec - time

% 9
\gloexe{Glo7:Pampaykuni}{}{amv}%
{Pampaykuni \pb{frutachaykuna} \pb{apasa}yta.}%amv que first line
{\morglo{pampa-yku-ni}{bury-\lsc{excep}-\lsc{1}}\morglo{fruta-cha-y-kuna}{	fruit-\lsc{dim}-\lsc{1}-\lsc{pl}}\morglo{apa-sa-y-ta}{bring-\lsc{prf}-\lsc{1}-\lsc{acc}}}%morpheme+gloss
\glotran{I bury the fruit \pb{that I bring}.}{}%eng+spa trans
{}{}%rec - time

% 10
\gloexe{Glo7:Kalamina}{}{lt}%
{Kalamina \pb{rantishanchikkuna}.}%lt que first line
{\morglo{kalamina}{corrugated.iron}\morglo{ranti-sha-nchik-kuna}{buy-\lsc{prf}-\lsc{1pl}-\lsc{pl}}}%morpheme+gloss
\glotran{The tin roofing \pb{that we bought}.}{}%eng+spa trans
{}{}%rec - time

% 11
\gloexe{Glo7:mayman}{}{sp}%
{Ni mayman yayku\pb{sa}y yatrakunchu.}%sp que first line
{\morglo{ni}{nor}\morglo{may-man}{where-\lsc{all}}\morglo{yayku-sa-y}{enter-\lsc{prf}-\lsc{1}}\morglo{yatra-ku-n-chu}{know-\lsc{refl}-\lsc{3}-\lsc{neg}}}%morpheme+gloss
\glotran{They didn’t know even \pb{where I had gone in}.}{}%eng+spa trans
{}{}%rec - time

% 12
\gloexe{Glo7:Ampullakta}{}{ch}%
{Ampullakta inyikta\pb{mana}nchikpaq.}%ch que first line
{\morglo{ampulla-kta}{ampoule-\lsc{acc}}\morglo{inyikta-ma-na-nchik-paq}{inject-\lsc{1.obj}-\lsc{nmlz}-\lsc{1pl}-\lsc{purp}}}%morpheme+gloss
\glotran{Ampoules \pb{to} inject us / \pb{for} injecting us.}{}%eng+spa trans
{}{}%rec - time

% 13
\gloexe{Glo7:Filupa}{}{amv}%
{Filupa paninqa nin, “Maqa\pb{way}tam ñuqata pinsayan”.}%amv que first line
{\morglo{Filu-pa}{Filu-\lsc{gen}}\morglo{pani-n-qa}{sister-\lsc{3}-\lsc{top}}\morglo{ni-n}{say-\lsc{3}}\morglo{maqa-wa-y-ta-m}{hit-\lsc{1.obj}-\lsc{inf}-\lsc{acc}-\lsc{evd}}\morglo{ñuqa-ta}{I-\lsc{acc}}\morglo{pinsa-ya-n}{think-\lsc{prog}-\lsc{3}}}%morpheme+gloss
\glotran{Filomena’s sister said, “He’s thinking about hitting [wants to hit] \pb{me}.”}{}%eng+spa trans
{}{}%rec - time

\section{Subordination}\index[sub]{sentence!subordination}
This section partially repeats §~\ref{ssec:subordination} on subordination. Please consult that section for further discussion and glossed examples. \SYQ{} counts three subordinating suffixes --~\phono{-pti}, \phono{-shpa}, and \phono{-shtin}~-- and one subordinating structure --~\phono{-na-}\lsc{poss}\phono{-kama}. Additionally, in combination with the purposive case suffix, \phono{-paq}, \phono{-na} forms subordinate clauses that indicate the purpose of the action expressed in the main clause (\phono{qawa-na-y-paq} ‘so I can see’) (see §~\ref{sssc:con}).\footnote{An anonymous reviewer points out that all of the case-marked deverbal NPs --~not just \phono{-kama} and \phono{-paq}~-- can form subordinate/adverbial clauses.}

\phono{-pti} is employed when the subjects of the main and subordinate clauses are different (\phono{huk} \phono{qawa-\pb{pti-n-qa}}, \phono{ñuqa-nchik} \phono{qawa-nchik-chu} ‘Although others see, we don’t see’)~(\ref{Glo7:Qawayku}); \phono{shpa} and \phono{-shtin} are employed when the subjects of the two clauses are identical (\phono{tushu\pb{-shpa}} \phono{wasi-ta} \phono{kuti-mu-n} ‘Dancing they return home’)~(\ref{Glo7:Chitchityaku}), (\ref{Glo7:Yantakunata}). \phono{-pti} generally indicates that the event of the subordinated clause began prior to that of the main clause but may also be employed in the case those events are simultaneous (\phono{urkista-qa} \phono{traya-mu\pb{-pti}-n} \phono{tushu-rqa-nchik} ‘When the band arrived, we dansed’). \phono{-shpa} generally indicates that the event of the subordinated clause is simultaneous with that of the main clause (\phono{sapu-qa} \phono{kurrkurrya-\pb{shpa}} \phono{kurri-ya-n} ‘The frog is running going \emph{kurr-kurr!}’)~(\ref{Glo7:Traguwan}) but may also be employed in case the subordinated event precedes the main-clause event~(\ref{Glo7:Familyanchikta}). \phono{-shtin} is employed only when the main and subordinate clause events are simultaneous (\phono{Awa\pb{-shtin}} \phono{miku-chi-ni} \phono{wambra-y-ta} ‘(By) weaving, I feed my children’)~(\ref{Glo7:Yatrakunchik}). \phono{-pti} subordinates are suffixed with allocation suffixes (\phono{tarpu\pb{-pti}-\pb{nchik}} ‘when we plant’)~(\ref{Glo7:Manampaga}); \phono{-shpa} and \phono{-shtin} subordinates do not inflect for person or number (\phononb{*tarpu-\pb{shpa}-\pb{nchik}}; \phono{*tarpu-\pb{shtin}-\pb{yki}}). Subordinate verbs inherit tense, aspect and conditionality specification from the main clause verb (\phono{Ri-shpa qawa-\pb{y-man}} \phono{\pb{karqa}} ‘If I \pb{would have} gone, I \pb{would have} seen’). Depending on the context, \phono{-pti} and \phono{-shpa} can be translated by ‘when’~(\ref{Glo7:Qawayku}), ‘if’~(\ref{Glo7:Kuti}), ‘because’~(\ref{Glo7:Priykupaw}),~(\ref{Glo7:Payqa}) ‘although’~(\ref{Glo7:Qullqita}) or with a gerund~(\ref{Glo7:Chitchityaku}). \phono{-shtin} is translated by a gerund only~(\ref{Glo7:Yantakunata}),~(\ref{Glo7:Yatrakunchik}).

\phono{-na-}\lsc{poss}\phono{-kama} is limitative. It forms subordinate clauses indicating that the event referred to either is simultaneous with~(\ref{Glo7:Mana}) or limits~(\ref{Glo7:Traki}) the event referred to in the main clause (\phono{puñu-\pb{na-y-kama}} ‘while I was sleeping’; \phono{wañu-\pb{na-n-kama}} ‘until she died’).\\

% 1
\gloexe{Glo7:Qawayku}{}{amv}%
{Qawayku\pb{ptin}qa sakristan wañurusa.}%amv que first line
{\morglo{qawa-yku-pti-n-qa}{see-\lsc{excep}-\lsc{subds}-\lsc{3}-\lsc{top}}\morglo{sakristan}{sacristan}\morglo{wañu-ru-sa}{die-\lsc{urgt}-\lsc{npst}}}%morpheme+gloss
\glotran{\pb{When} \pb{he} looked, the care-taker had died.}{}%eng+spa trans
{}{}%rec - time

% 2
\gloexe{Glo7:Chitchityaku}{}{lt}%
{Chitchityaku\pb{shpa} rikullan kabrakunaqa.}%lt que first line
{\morglo{chitchitya-ku-shpa}{say.chit.chit-\lsc{refl}-\lsc{subis}}\morglo{riku-lla-n}{go-\lsc{rstr}-\lsc{3}}\morglo{kabra-kuna-qa}{goat-\lsc{pl}-\lsc{top}}}%morpheme+gloss
\glotran{“\pb{Chit-chitting},” the goats just left.}{}%eng+spa trans
{}{}%rec - time

% 3
\gloexe{Glo7:Yantakunata}{}{amv}%
{Yantakunata qutu\pb{shtin} lliptakunata kañaku\pb{shtin}, hanay~\dots}%amv que first line
{\morglo{yanta-kuna-ta}{firewood-\lsc{pl}-\lsc{acc}}\morglo{qutu-shtin}{gather-\lsc{subavd}}\morglo{llipta-kuna-ta}{ash-\lsc{pl}-\lsc{acc}}\morglo{kaña-ku-shtin}{burn-\lsc{refl}-\lsc{subadv}}\morglo{hanay}{up.hill}}%morpheme+gloss
\glotran{\pb{Gathering} wood, \pb{burning} ash, [we lived] up hill.}{}%eng+spa trans
{}{}%rec - time

% 4
\gloexe{Glo7:Traguwan}{}{amv}%
{Traguwan, kukawan tushuchi\pb{shpa}llam kusichakuni.}%amv que first line
{\morglo{tragu-wan}{drink-\lsc{instr}}\morglo{kuka-wan}{coca-\lsc{instr}}\morglo{tushu-chi-shpa-lla-m}{dance-\lsc{caus}-\lsc{subis}-\lsc{rstr}-\lsc{evd}}\morglo{kusicha-ku-ni}{harvest-\lsc{refl}-\lsc{1}}}%morpheme+gloss
\glotran{With liquor and coca, \pb{making} them dance, I harvest.}{}%eng+spa trans
{}{}%rec - time

% 5
\gloexe{Glo7:Familyanchikta}{}{sp}%
{Familyanchikta wañurichi\pb{shpa}qa lliw partiyan.}%sp que first line
{\morglo{familya-nchik-ta}{family-\lsc{1pl}-\lsc{acc}}\morglo{wañu-ri-chi-shpa-qa}{die-\lsc{incep}-\lsc{caus}-\lsc{subis}-\lsc{top}}\morglo{lliw}{all}\morglo{parti-ya-n}{distribute-\lsc{prog}-\lsc{3}}}%morpheme+gloss
\glotran{\pb{After} they killed our relatives, they distributed everything.}{}%eng+spa trans
{}{}%rec - time

% 6
\gloexe{Glo7:Yatrakunchik}{}{ach}%
{Yatrakunchik imaynapis maski waqaku\pb{shtin}pis~\dots{} asiku\pb{shtin}pis~\dots{} imaynapis.}%ach que first line
{\morglo{yatra-ku-nchik}{live-\lsc{refl}-\lsc{1pl}}\morglo{imayna-pis}{how-\lsc{add}}\morglo{maski}{maski}\morglo{waqa-ku-shtin-pis}{cry-\lsc{refl}-\lsc{subadv}-\lsc{add}}\morglo{asi-ku-shtin-pis}{laugh-\lsc{refl}-\lsc{subis}-\lsc{add}}\morglo{imayna-pis}{how-\lsc{add}}}%morpheme+gloss
\glotran{We live however we can, although \pb{we’re crying}~\dots{} \pb{laughing}~\dots{} however we can.}{}%eng+spa trans
{}{}%rec - time

% 7
\gloexe{Glo7:Manampaga}{}{lt}%
{Manam pagawa\pb{ptiki}qa manam wambraykiqa alliyanqachu.}%lt que first line
{\morglo{mana-m}{no-\lsc{evd}}\morglo{paga-wa-pti-ki-qa}{pay-\lsc{1.obj}-\lsc{subds}-\lsc{2}-\lsc{top}}\morglo{mana-m}{no-\lsc{evd}}\morglo{wambra-yki-qa}{child-\lsc{2}-\lsc{top}}\morglo{alli-ya-nqa-chu}{good-\lsc{inch}-\lsc{3.fut}-\lsc{neg}}}%morpheme+gloss
\glotran{\pb{If} \pb{you} don’t pay \pb{me}, your son isn’t going to get better.}{}%eng+spa trans
{}{}%rec - time

% 8
\gloexe{Glo7:Kuti}{}{amv}%
{Kuti\pb{shpa}qa kutimushaq kimsa tawa watata.}%amv que first line
{\morglo{kuti-shpa-qa}{return-\lsc{subis}-\lsc{top}}\morglo{kuti-mu-shaq}{return-\lsc{cisl}-\lsc{1.fut}}\morglo{kimsa}{three}\morglo{tawa}{four}\morglo{wata-ta}{year-\lsc{acc}}}%morpheme+gloss
\glotran{\pb{If} \pb{I} come back, \pb{I}’ll come back in three or four years.}{}%eng+spa trans
{}{}%rec - time

% 9
\gloexe{Glo7:Priykupaw}{}{amv}%
{Priykupaw puriyan siyrtumpatr warmin mal ka\pb{pti}n.}%amv que first line
{\morglo{priykupaw}{worried}\morglo{puri-ya-n}{walk-\lsc{prog}-\lsc{3}}\morglo{siyrtumpa-tr}{certainly-\lsc{evc}}\morglo{warmi-n}{woman.\lsc{3}}\morglo{mal}{bad}\morglo{ka-pti-n}{be-\lsc{subds}-\lsc{3}}}%morpheme+gloss
\glotran{Certainly, he’d be wandering around worried \pb{because} \pb{his wife} is sick.}{}%eng+spa trans
{}{}%rec - time

% 10
\gloexe{Glo7:Payqa}{}{amv}%
{Payqa rikunñash warmin saqiru\pb{ptin}.}%amv que first line
{\morglo{pay-qa}{he-\lsc{top}}\morglo{ri-ku-n-ña-sh}{go-\lsc{refl}-\lsc{3}-\lsc{disc}-\lsc{evr}}\morglo{warmi-n}{woman-\lsc{3}}\morglo{saqi-ru-pti-n}{leave-\lsc{urgt}-\lsc{subds}-\lsc{3}}}%morpheme+gloss
\glotran{He left \pb{because} his wife abandoned him, they say.}{}%eng+spa trans
{}{}%rec - time

% 11
\gloexe{Glo7:Qullqita}{}{ach}%
{Qullqita gana\pb{shpa}\pb{s} bankuman ima trurakunki}%ach que first line
{\morglo{qullqi-ta}{money-\lsc{acc}}\morglo{gana-shpa-s}{earn-\lsc{subis}-\lsc{add}}\morglo{banku-man}{bank-\lsc{all}}\morglo{ima}{what}\morglo{trura-ku-nki}{put-\lsc{refl}-\lsc{2}}}%morpheme+gloss
\glotran{\pb{Although} you earn money and save it in the bank}{}%eng+spa trans
{}{}%rec - time

% 12 (13*)
\gloexe{Glo7:Mana}{}{amv}%
{Mana vilakuranichu puñu\pb{naykaman}.}%amv que first line
{\morglo{mana}{no}\morglo{vila-ku-ra-ni-chu}{watch.over-\lsc{refl}-\lsc{pst}-\lsc{1}-\lsc{neg}}\morglo{puñu-na-y-kaman}{sleep-\lsc{nmlz}-\lsc{1}-\lsc{lim}}}%morpheme+gloss
\glotran{I didn’t keep watch \pb{while} I was sleeping.}{}%eng+spa trans
{}{}%rec - time

% 13 (14*)
\gloexe{Glo7:Traki}{}{amv}%
{Traki paltanchikpis pushllu\pb{nankama} purinchik.}%amv que first line
{\morglo{traki}{foot}\morglo{palta-nchik-pis}{soul-\lsc{1pl}-\lsc{add}}\morglo{pushllu-na-n-kama}{blister-\lsc{nmlz}-\lsc{3}-\lsc{lim}}\morglo{puri-nchik}{walk-\lsc{1pl}}}%morpheme+gloss
\glotran{We walked \pb{until} blisters formed on the souls of our feet.}{}%eng+spa trans
{}{}%rec - time
