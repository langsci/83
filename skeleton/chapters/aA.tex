% APPENDIX A Analysis of the Southern Yauyos Quechua lexicon
\chapter{Analysis of the Southern Yauyos Quechua lexicon}

What follows is an analysis of lexical differences among the five dialects. This analysis is excerpted from the introduction to the lexicon that accompanies this volume.

The lexicon counts 2537 Quechua words. Most were gleaned from glossed recordings collected in the eleven districts over the course of four years, 2010--2014; additional terms were identified by eliciting cognate or correlate terms for various items in \citet{CerroP94}\index[aut]{Cerrón-Palomino, Rodolfo M.}'s unified dictionary of Southern Quechua as well as his dictionary of Junín-Huanca Quechua (\citet{CerroP76b}). The recordings and annotated transcriptions have been archived by The Language Archive of the Dokumentation Bedrohter Sprachen/Documentation of Endangered Languages~(DoBeS)\index[sub]{DoBeS} archive at the Max Planck Institute~(\url{http://corpus1.mpi.nl/ds/imdi_browser/?openpath=MPI1052935\%23}) and the Archive of the Indigenous Languages of Latin America~(AILLA)\index[sub]{AILLA} at the University of Texas at Austin~(\url{http://www.ailla.utexas.org/site/welcome.html}). All documents --~including the unformatted \mbox{.xml} lexical database~-- can be consulted via those institutions’ web sites. All terms were reviewed with at least two speakers of each dialect: Benedicta Lázaro and Martina Reynoso~(\AH); Mila Chávez, Delfina Chullunkuy, Esther Madueño, Hilda Quispe, and Celia Rojas~(\MV); Iris Barrosa, Gloria Cuevas, Senaida Oré, Hipólita Santos, and Erlinda Vicente,~(\CH); Ninfa Flores and Sofía Vicente~(\LT); and Santa Ayllu, Elvira Huamán, Sofía Huamán, and Maximina P. 

As stated in the Introduction, Yauyos is located on the border between the two large, contiguous zones where languages belonging to the two great branches of the Quechua language family are spoken: the “Quechua~I” (Torero) or “Quechua B” (Parker) languages are spoken in the regions immediately to the north; the “Quechua~II” or “Quechua~A” languages, in the regions immediately to the south. Both grammatically and lexically, the dialects of southern Yauyos share traits with both the \QI{} and \QII{} languages. Critically, however, the dialects which sort with the the \QI{} languages grammatically do not necessarily also sort with them lexically; nor do the dialects which sort with the \QII{} languages grammatically necessarily sort with them lexically. That is, grammatically and lexically, the dialects cleave along distinct lines.

Grammatically, two of the five dialects --~those of Madeán-Viñac and Lincha-Tana~-- sort together, as these, like the \QII{} languages, indicate the first-person subject with \phono{-ni}, the first-person possessor with \phono{-y}, and first-person object with \phono{-wa}. The remaining three --~Azángaro-Huangáscar, Cacra-Hongos, and San Pedro~-- sort together, as these, like the \QI{} languages, indicate the first person subject and possessor with vowel length and the first-person object with \phono{-ma}.%
\footnote{Yauyos counts three additional dialects, spoken in the districts of Alis and Tomas; Huancaya and Vitis; and Laraos, all located in the north of the province. The lexicon, like the grammar, makes abstraction of these dialects.}

Lexically, however, the dialects cleave along different lines, lines defined not by morphology but by geography. Lexically, the two more northern dialects --~the “\QI” \CH{} and the “\QII” \LT~-- sort together while the three more southern dialects --~the “\QI” \AH{} and \SP{} together with the “\QII” \MV~-- sort together. Below, I detail an analysis of the lexicon that I performed using a subset of 2551 terms. The dialects generally agree in the terms they use to name the same referent: I could identify only 37 instances in which the dialects employed words of different roots. In~32 of these instances the dialects cleaved along north-south lines and in 22 of the relevant 28 cases for which correlate terms could be identified from Junín-Huanca Quechua and Ayacucho Quechua (the former a “\QI” language spoken immediately to the north of Yauyos, the second, a “\QII” language spoken very nearby, to the south), the northern dialects employed the term used in Junín-Huanca, while the southern dialects employed the term used in Ayacucho.%
%%%
\footnote{No pair was counted more than once. The lexicon includes both roots and derived terms. Thus both the pairs \phono{sumaq}~(\MV, \AH, \SP) and \phono{tuki}~(\CH, \LT) ‘pretty’ and \phono{sumaq-lla}~(\MV, \AH, \SP) and \phono{tuki-lla}~(\CH, \LT) ‘nicely’ appear in the corpus. Only the root pair, \phono{sumaq}~\textasciitilde~\phono{tuki}, was entered in the catalogue of those cases where dialects differed in root terms employed. There were 116 cases of this type. These were excluded from the count and account given here. Examples are given immediately below. 

\begin{center}
\begin{small}
\begin{tabular}{p{4ex}l}
\lsptoprule
\multicolumn{2}{l}{\phono{qawa-} (\MV, \AH, \SP)~\textasciitilde~\phono{rika-} (\CH, \LT) ‘see’}\\
&~→~\phono{qawa-chi-}~\textasciitilde~\phono{rika-chi-} ‘show’,’make and offering’ \\
\multicolumn{2}{l}{\phono{chakwash} (\MV, \AH, \SP)~\textasciitilde~\phono{paya} (\CH, \LT) ‘old woman’}\\
&~→~\phono{chakwash-ya-}~\textasciitilde~\phono{paya-ya-} ‘become an old woman’ \\
\multicolumn{2}{l}{\phono{qishta} (\MV, \AH, \SP, \LT)~\textasciitilde~\phono{tunta} (\CH, \LT) ‘nest’}\\
&~→~\phono{qishta-cha-}~\textasciitilde~\phono{tunta-cha-} ‘build a nest’\\
\lspbottomrule
\end{tabular}
\end{small}
\end{center}}%

This does not mean that the dialects employed identical terms in all the remaining 2387 cases (subtracting 75 for 36 pairs and one triplet). Far from it. All dialects employed identical terms in only 1603 instances. Included among these are all but 20 of the 522 words in the corpus borrowed from Spanish (examples in~\ref{ExA:1}.\footnote{Virtually any term of Spanish origin in current use in the area may be borrowed into \SYQ. I have included Spanish- origin words in the lexicon just in case they were either~\ref{ExA:1} of extremely high use (\phono{tuma-} ‘take’, ‘drink’ (\Sp~\spanish{tomar} ‘take’, ‘drink’)); \ref{ExA:2} had no corresponding indigenous term (in contemporary usage) (\phono{matansya} ‘massacre’ (\Sp~\spanish{matanza} ‘massacre’)); or~\ref{ExA:3} had altered substantially either in their pronunciation or denotation (\phono{firfanu} ‘orphan’ (\Sp~\spanish{huérfano} ‘orphan’); \phono{baliya-} ‘shoot’ (\Sp~\spanish{bala} ‘bullet’)).} Once terms of Spanish origin are eliminated, we are left with a corpus of 1940 items. All dialects agreed perfectly in their realizations of these items in 1081 cases~(56\%) (examples in~\ref{ExA:2}). The remaining 755 items are accounted for as follows. In 154 cases a Quechua-origin term was realized identically in all dialects in which it was attested but remained unattested in one or more dialects, as in~\ref{ExA:3}. Given the current state of the language --~classified as “moribund” in the 2013 edition of \underline{\smash{Ethnologue}} \citet{ethnologue}\index[aut]{Ethnologue}(\index[aut]{Lewis, M. Paul})~-- nothing can be concluded from these gaps, neither that the dialects originally employed the same term, nor that it was necessarily different. In 630 cases, the dialects employed terms of the same root but with different realizations, as in~\ref{ExA:4}. Included among these are 236 cases where these differences can be attributed to differences in the phonology between Cacra-Hongos and the other four dialects: the realization of \textipa{*[r]} as \textipa{[l]}, for example (151 cases, examples in~\ref{ExA:5}) or \textipa{*/s/} as \textipa{[h]} (45~cases, examples in~\ref{ExA:6}). Also counted among these 745~cases are terms affected by metathesis and other phonological processes (vowel lowering (\textipa{/i/}), velarization~(\textipa{/q/}), depalatization~(\textipa{/sh/}), and gliding~(\textipa{/y/}), among others) (207 cases, examples in~\ref{ExA:7} and~\ref{ExA:8}). Finally, the sample counts terms affected by variation in verbal or nominal morphology (62 cases, examples in~\ref{ExA:9}). Principal among these are instances of words derived with past participles --~formed with \phono{-sha} in the north and \phono{-sa} in the south~-- and others that also differ by virtue of the fronting of \textipa{/sh/} (40 cases, examples in~\ref{ExA:10} and~\ref{ExA:11}).

\todo{labelling changed}
% \begin{enumerate}[label={(\arabic*)}]
\begin{enumerate}
\item\label{ExA:1} Spanish-origin terms identical in all dialects

\begin{small}
\begin{tabular}{l@{~}lll}
\lsptoprule
\phono{tuma-} &(\ALL) &(\Sp~\spanish{tomar}) 	&‘take’	\\
\phono{kida-} &(\ALL) &(\Sp~\spanish{quedar}) 	&‘stay’	\\
\phono{papil} &(\ALL) &(\Sp~\spanish{papel}) 	&‘paper’	\\
\lspbottomrule
\end{tabular}
\end{small}

\newpage
\item\label{ExA:2} Quechua-origin terms identical in all dialects 

\begin{small}
\begin{tabular}{l@{~}ll}
\lsptoprule
\phono{sapi} 	&(\ALL) &‘root’	\\
\phono{sasa} 	&(\ALL) &‘hard’	\\
\phono{yanapa-} &(\ALL) &‘help’	\\
\phono{ishpay} 	&(\ALL) &‘urine’	\\
\phono{ayqi-} 	&(\ALL) &‘escape’	\\
\phono{chaqchu-} &(\ALL) &‘sprinkle, scatter’	\\
\lspbottomrule
\end{tabular}
\end{small}

\item\label{ExA:3} Terms with no Quechua-language correlate in one or more of the dialects 

\begin{small}
\begin{tabular}{l@{~}ll@{~}ll}
\lsptoprule
\multicolumn{2}{l}{Quechua-origin term}		&	\multicolumn{2}{l}{Spanish-origin term}			& Gloss	\\
\midrule
\phono{chaskay} 	&(\MV, \AH, \SP) & \phono{lusiru} (\Sp~\spanish{lucero}) &(\CH, \LT)			& ‘morning star’	\\
\phono{tapsipa-} 	&(\MV, \AH, \SP) & \phono{balansya} (\Sp~\spanish{balancear}) &(\CH, \LT)	& ‘rock’		\\
\phono{uya} 		&(\MV, \AH, \SP) & \phono{kara} (\Sp~\spanish{cara}) &(\CH, \LT) 			& ‘face’		\\
\lspbottomrule
\end{tabular}
\end{small}

\item\label{ExA:4} Terms of the same root but with different realizations in different dialects 

\begin{small}
\begin{tabular}{l@{~}l@{~\textasciitilde~}l@{~}ll}
\lsptoprule
\phono{wa\pb{r}mi}	& (\MV, \AH, \SP)			& \phono{wa\pb{l}mi}	& (\LT, \CH)	& ‘woman’				\\
\phono{\pb{s}apa}	& (\MV, \AH, \SP)			& \phono{\pb{h}apa}		& (\LT, \CH)	& ‘alone’				\\
\phono{a\pb{qs}a}	& (\MV, \AH)				& \phono{a\pb{sq}a}		& (\SP)			& ‘bitter [potato]’		\\
\phono{\pb{q}aracha}& (\MV, \AH, \SP, \CH)	& \phono{\pb{k}aracha}	& (\LT)			& ‘scabies’, ‘mange’	\\
\phono{alli-\pb{paq}}& (\MV, \AH, \SP)		& \phono{alli-\pb{lla}}	& (\LT, \CH)	& ‘slowly’				\\
\phono{kitra-\pb{s}a}& (\MV, \AH, \SP)		& \phono{kitra-\pb{sh}a}& (\LT, \CH)	& ‘open’				\\
\lspbottomrule
\end{tabular}
\end{small}

\item\label{ExA:5} Terms where \textipa{*[r]} is realized as \textipa{[l]} in \CH{} 

\begin{small}
\begin{tabular}{l@{~→~}ll}
\lsptoprule
\phono{\pb{r}aki-}		& [\phono{\pb{l}aki}]		& ‘separate’	\\
\phono{qu\pb{r}u}		& [\phono{qo\pb{l}u}]		& ‘mutilated’	\\
\phono{tru\pb{r}a-}		& [\phono{\^{c}u\pb{l}a}]	& ‘put’			\\
\lspbottomrule
\end{tabular}
\end{small}

\item\label{ExA:6} Terms where \textipa{*/s/} is realized as \textipa{[h]} in \CH{} 

\begin{small}
\begin{tabular}{l@{~→~}ll}
\lsptoprule
\phono{/\pb{s}ara/} 	& [\phono{\pb{h}ala}]	& ‘corn’	\\
\phono{/\pb{s}ama/} 	& [\phono{\pb{h}ama}]	& ‘rest’	\\
\phono{/\pb{s}ati/} 	& [\phono{\pb{h}ati}]	& ‘insert’	\\
\lspbottomrule
\end{tabular}
\end{small}

\newpage
\item\label{ExA:7} Terms affected by metathesis 

\begin{small}
\begin{tabular}{l@{~}l@{~\textasciitilde~}l@{~}ll}
\lsptoprule
\phono{\pb{ch}a\pb{ks}a-} &(\MV, \AH, \CH) & \phono{\pb{ch}a\pb{sk}a-} &(\LT, \SP) & ‘air out’	\\
\phono{\pb{sh}an\pb{t}a-} &(\AH, \CH, \SP) & \phono{\pb{t}an\pb{sh}a-} &(\MV, \LT) & ‘choke’		\\
\phono{\pb{sh}ip\pb{t}i-} &(\MV, \AH, \LT) & \phono{\pb{t}ip\pb{shi}-} &(\CH, \SP) & ‘pinch’		\\
\lspbottomrule
\end{tabular}
\end{small}

\item\label{ExA:8} Terms affected by other phonological processes 

\begin{small}
\begin{tabular}{l@{~}l@{~\textasciitilde~}l@{~}l@{~}ll}
\lsptoprule
\phono{allp\pb{i}}	& (\MV, \AH, \LT, \CH) 	& \phono{allp\pb{a}} &(\SP) &‘dust’, ‘dirt’	& (vowel lowering)	\\
\phono{chill\pb{q}i}& (\MV, \AH, \LT, \SP) 	& \phono{chill\pb{k}i} &(\CH) &‘bud’			& (develarization)	\\
\phono{ma\pb{l}shu}	& (\LT, \CH) 				& \phono{ma\pb{y}shu} &(\MV, \AH, \SP) &‘breakfast’& (gliding)	\\
\lspbottomrule
\end{tabular}
\end{small}

\item\label{ExA:9} Terms affected by variation in verbal or nominal morphology 

\begin{small}
\begin{tabular}{l@{~}l@{~\textasciitilde~}l@{~}ll}
\lsptoprule
\phono{utrku-} &(\MV, \AH, \LT, \SP) & \phono{utr’ku-\pb{cha}-} & (\CH) 		& ‘dig a hole’		\\
\phono{tardi-\pb{ku}} &(\MV{} \AH, \CH, \LT)& \phono{tardi-\pb{ya}-} & (\SP) 		& ‘get late’		\\
\phono{aytri-\pb{na}} &(\MV, \CH) 				 & \phono{aytri-\pb{ku}} & (\AH, \LT) 	& ‘stick for stirring’	\\
\lspbottomrule
\end{tabular}
\end{small}

\item\label{ExA:10} Terms derived with past participles 

\begin{small}
\begin{tabular}{l@{~}l@{~\textasciitilde~}l@{~}ll}
\lsptoprule
\phono{paki-\pb{s}a} 	&(\MV, \AH, \SP) & \phono{paki-\pb{sh}a} &(\CH, \LT) 	&‘broken’		\\
\phono{punki-\pb{s}a} 	&(\MV, \AH, \SP) & \phono{punki-\pb{sh}a} &(\CH, \LT) 	&‘swolen’		\\
\phono{yaku-na-\pb{s}a}	&(\MV, \AH, \SP) & \phono{yaku-na-\pb{sh}a} &(\CH, \LT) &‘thirsty’	\\
\lspbottomrule
\end{tabular}
\end{small}

\item\label{ExA:11} Terms that differ by the exchange s/sh 

\begin{small}
\begin{tabular}{l@{~}l@{~\textasciitilde~}l@{~}ll}
\lsptoprule
\phono{\pb{s}uytu} &(\MV, \AH, \SP) 		&\phono{\pb{sh}uytu} &(\CH, \LT) & ‘oval’, ‘oblong’	\\
\phono{\pb{s}iq\pb{s}i-} &(\MV, \AH, \SP)	&\phono{\pb{sh}iq\pb{sh}i-} &(\CH, \LT) & ‘itch’			\\
\phono{wi\pb{s}wi} &(\MV, \AH, \SP{} \CH) &\phono{wi\pb{sh}wi} &(\LT) & ‘greasy’					\\
\lspbottomrule
\end{tabular}
\end{small}
\end{enumerate}

A clear pattern emerges both with regard to the cases where the dialects employed terms of different roots and those in which they varied in their realizations of the same root term. In~32 of the~37 instances in which root terms differed, the dialects cleaved along north-south lines, with the northern dialects --~\CH{} and \LT\footnote{With the exception of two and a half cases: one where \LT{} sorts with the southern dialects~(‘make an offering’), one where \LT{} recorded no Quechua-origin term~(‘bitter’), and one where Cacra and Hongos split, Cacra alone recording a second term~(‘rain’).}~-- sorting together and the southern dialects --~\MV, \AH, and \SP~-- sorting together, as in~\ref{ExA:12}.

In four of the five remaining instances San Pedro supplied the outstanding term. In~32 of the~37 cases, cognate terms could be identified for Junín and Ayacucho (Yauyos’ “\QI” (northern) and “\QII” (southern) neighbors, respectively). In 23 of the relevant 28 of these 32 cases, the northern dialects --~“\QI” \CH{} and “\QII” \LT~-- employed the term used in Junín, while the southern dialects --~the “\QI” \AH{} and \SP{} and the “\QII” \MV~-- employed the term used in Ayacucho, as in~\ref{ExA:13}.\footnote{In at least two of these 32 cases, the Junín term had a cognate correlate in Jaqaru, an Aymaran language spoken in Tupe, Cacra’s closest neighbor to the north. The terms are \phono{kallwi-} ‘cultivate’ and \phono{liklachiku} ‘underarm’.}

The full list appears in Table~\ref{TabA4}.

\todo{labelling changed}
% \begin{enumerate}[resume,label={(\arabic*)}]
\begin{enumerate}
\item\label{ExA:12} Root terms varying along north-south lines 

\begin{small}
\begin{tabular}{l@{~}ll@{~}ll}
\lsptoprule
\multicolumn{2}{l}{South}		& \multicolumn{2}{l}{North}		& 		\\
\multicolumn{2}{l}{\MV, \AH, \SP}		& \multicolumn{2}{l}{\LT, \CH}		& Gloss		\\
\midrule
\phono{chumpi} & (\MV, \AH, \SP) &\phono{watrakuq} & (\CH, \LT) & ‘sash’		\\
\phono{anu-} & (\MV, \AH, \SP) &\phono{wasqi-} & (\CH, \LT) & ‘wean’		\\
\phono{sumaq} & (\MV, \AH, \SP) &\phono{tuki} & (\CH, \LT) & ‘pretty’	\\
\lspbottomrule
\end{tabular}
\end{small}

\item\label{ExA:13} North/south differences in root terms alligning with Junín and Ayacucho. 

\begin{small}
\begin{tabular}{lllll}
\lsptoprule
South		&	North	&&&\\
\MV, \AH, \SP{} & \LT, \CH{} & Ayacucho	& Junín & Gloss		\\
\midrule
\phono{puyu} & \phono{pukatay} & \phono{puyu} & \phono{pukatay} & ‘cloud’, ‘fog’ \\
\phono{qishTa} & \phono{tunta} & \phono{qisha} & \phono{tunta} &‘nest’	\\
\phono{rakta} & \phono{tita} & \phono{rakta} & \phono{tita} & ‘thick’	\\
\lspbottomrule
\end{tabular}
\end{small}

\item\label{ExA:14} Synonyms employed in southern but not northern dialects 

\begin{small}
\begin{tabular}{l@{~}ll@{~}ll}
\lsptoprule
\multicolumn{2}{l}{Employed}		& \multicolumn{2}{l}{Employed}		& 		\\
\multicolumn{2}{l}{in all}		& \multicolumn{2}{l}{just in the south}		& Gloss		\\
\midrule
\phono{wallwa-} &(\ALL) & \phono{uqlla(n)cha-} &(\MV, \AH, \SP) & ‘carry under the arm’ \\
\phono{patrya-} &(\ALL) & \phono{tuqya-} &(\MV, \AH, \SP) & ‘explode’ \\
\phono{alalaya-} &(\ALL) & \phono{chiriya-} &(\MV, \AH, \SP) &‘be cold’ \\
\lspbottomrule
\end{tabular}
\end{small}
\end{enumerate}

I have taken it as my task here only to present the data; I leave it to other scholars to come to their own conclusions. The raw data are available in the form of an \mbox{.xml} document that can be accessed by all via the DoBeS and AILLA websites.

\begin{landscape}
\renewcommand*\arraystretch{1.3}
\begin{small}
\begin{longtable}{>{\raggedright\let\newline\\\arraybackslash\hspace{0pt}}m{14ex}llll>{\raggedright\let\newline\\\arraybackslash\hspace{0pt}}m{18ex}>{\raggedright\let\newline\\\arraybackslash\hspace{0pt}}m{22ex}}
\caption{Differences among dialects in root terms used to refer to the same referent}\label{TabA4}

\\[2ex]
\lsptoprule
gloss & root\tss{A}	&dialect	& root\tss{B}	& dialect	& Ayacucho root & Junín root		\\
\midrule
\endfirsthead

\multicolumn{7}{c}{\tablename\ \thetable: Continued from previous page.} \\
\lsptoprule
gloss & root\tss{A}	&dialect	& root\tss{B}	& dialect	& Ayacucho root & Junín root		\\
\midrule
\endhead

\lspbottomrule \multicolumn{7}{r}{{\footnotesize Continued on next page~\dots}} \\
\endfoot

\lspbottomrule
\endlastfoot

‘old man’	& \phono{machu} & 	\MV, \AH, \SP{}	& \phono{awkish} & 	\LT, \CH{}	& \phono{machu} & \phono{awkish} \\
‘old woman’	& \phono{chakwash} & 	\MV, \AH, \SP{}	& \phono{paya} & 	\LT, \CH{}	& \phono{chakwash} & \phono{paya} \\
‘nettle’	& \phono{llupa/itana} & 	\MV, \AH, \SP{}	& \phono{chalka} & 	\LT, \CH{}	& \phono{itana} & \phono{itana} \\
‘germinate’	& \phono{shinshi-} & 	\MV, \AH, \SP{}	& \phono{chilQi} & 	\LT, \CH{}	& \phono{NC} & ? \\
‘close eyes, blink’	& \phono{qimchiku-} & 	\MV, \AH, \SP{}	& \phono{chipupa-} & 	\LT, \CH{}	& \phono{chipu-} (close hand) \phono{qimchikatraa-} & \phono{qimlla- / qimchi-} \\
‘sash’	& \phono{chumpi} & 	\MV, \SP{}	& \phono{watraku} & 	\LT, \CH{}	& \phono{chumpi} & \phono{watrakuq} \\
‘sneeze’	& \phono{hachiwsa-} & 	\MV, \AH, \CH, \LT{}	& \phono{haqchu-} & 	\SP{}	& \phono{hachi-} & \phono{haqchiwsa-}, \phono{achiwyaa-} \\
‘cultivate, hoe’	& \phono{hallma-} & 	\MV, \AH, \SP{}	& \phono{kallwa-} & 	\LT, \CH{}	& \phono{hallma-} & \phono{kallwa-} \\
‘scratch’	& \phono{rachka-} & 	\MV, \AH, \SP{}	& \phono{hata-} & 	\LT, \CH{}	& \phono{hata-} & \phono{rachka-} \\
‘add fuel’	& \phono{lawka-} & 	\MV, \AH, \CH, \LT{}	& \phono{huya-} & 	\SP{}	& ? & ? \\
‘sickly’	& \phono{iqu} & 	\MV, \AH, \SP{}	& \phono{latru} & 	\LT, \CH{}	& \phono{iqu} & ? \\
‘thorn, bramle’	& \phono{kichka} & 	\MV, \AH, \SP{}	& \phono{kasha} & 	\LT, \CH{}	& \phono{kichka} & \phono{kasha} \\
‘stick’	& \phono{kaspi} & 	\MV, \AH, \SP{}	& \phono{shukshu} & 	\LT, \CH{}	& \phono{kaspi} & \phono{shukshu} \\
‘splinter’	& \phono{killwi} & 	\MV, \AH{}	& \phono{qawa}/\phono{waqcha} & 	\LT, \CH{}/\SP{}	& \phono{killwi} & \phono{waqcha} (‘log’, ‘timber’) \\
‘make an offering return’	& \phono{qawachi-} & 	\MV, \AH, \LT{}	& \phono{likachi-} & 	\CH{}	& \phono{qawa-} (‘see’) & \phono{lika-} (‘see’) \\
‘underarm’, ‘armpit’	& \phono{wallwachuku} & 	\MV, \AH, \SP{}	& \phono{liklachku} & 	\LT, \CH{}	& \phono{wallwa} & \phono{liklachiku} \\
‘all’	& \phono{lliw} & 	\MV, \AH, \SP{}	& \phono{limpu} & 	\LT, \CH{}	& \phono{lliw} & \phono{lliw} \\
‘avalanche’, ‘mudslide’	& \phono{lluqlla} & 	\MV, \AH, \SP{}	& \phono{tuñiy} & 	\ALL{}	& \phono{tuñi-}\par (‘tumble down’) & \phono{lluqlla} (‘waterfall’) \\
‘coagulate’	& \phono{tika-} & 	\MV, \AH, \CH, \LT{}	& \phono{marki-} & 	\SP{}	& \phono{tikaya-} & \phono{tika-}\par (‘make adobe bricks’) \\
‘knee’	& \phono{muqu} & 	\MV, \AH, \SP{}	& \phono{qunqur} & 	\ALL{}	& \phono{muqu}, \phono{qunqura-} (‘kneel’) & \phono{muqu} (joint) \\
‘comb’ (v.)	& \phono{ñaqcha-} & 	\ALL{}	& \phono{qachaku-} & 	\LT, \CH{}	& \phono{ñaqcha-} & \phono{ñaqcha-} \\
‘cloud’, ‘fog’	& \phono{puyu} & 	\MV, \AH, \SP{}	& \phono{pukutay} & 	\LT, \CH{}	& \phono{puyu} & \phono{pukutay} \\
thorn bush variety	& \phono{ulanki} & 	\MV, \AH, \SP{}	& \phono{qaparara} & 	\LT, \CH{}	& ? & ? \\
‘sick’	& \phono{unqu} & 	\MV, \AH, \SP{}	& \phono{qisha} & 	\CH{}	& \phono{unqu} & \phono{qishya} \\
‘nest’	& \phono{qishTa} & 	\MV, \AH, \SP{}	& \phono{tunta} & 	\LT, \CH{}	& \phono{qisTa} & \phono{qisha} \\
‘thick’	& \phono{rakta} & 	\MV, \AH, \SP{}	& \phono{tita} & 	\LT, \CH{}	& \phono{rakta} & \phono{tita} \\
‘snow’, ‘sleet’	& \phono{riti} & 	\MV, \AH, \SP{}	& \phono{rasu} & 	\LT, \CH{}	& \phono{riti} & \phono{lasu} \\
‘eaten by birds’	& \phono{shuqli} & 	\MV, \AH, \CH, \LT{}	& \phono{wishlu} & 	\SP{}	& ? & ? \\
‘beautiful’	& \phono{sumaq} & 	\MV, \AH, \SP{}	& \phono{tuki} & 	\LT, \CH{}	& \phono{sumaq} & \phono{tuki} \\
‘sheep’	& \phono{uyqa} & 	\MV, \AH, \SP{}	& \phono{usha} & 	\LT, \CH{}	& \phono{NC} & \phono{(uwish)} \\
‘roll’	& \phono{sinku-} & 	\ALL{}	& \phono{trinta-} & 	\LT, \CH{}	& \phono{NC} & \phono{NC} \\
‘explode’	& \phono{tuqya-} & 	\MV, \AH, \SP{}	& \phono{patra-} & 	\ALL{}	& \phono{tuqya-} & \phono{patra-} \\
‘bitter’ [potato]	& \phono{aqsa} & 	\MV, \AH, \SP{}	& \phono{qatqi} & 	\CH{}	& \phono{qatqi} & ? \\
‘rain’	& \phono{para-} & 	\MV, \AH, \SP, \CH{}	& \phono{tamya-} & 	Cacra	& \phono{para-} & \phono{tamya-} \\
\lspbottomrule
\multicolumn{7}{l}{NC=~not cagnate; ?=~not found}\\
\end{longtable}
\end{small}
\end{landscape}
