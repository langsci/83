% CHAPTER 6 ENCLITICS
\chapter{Enclitics}\label{ch:enclitics}
This chapter covers the enclitic\index[sub]{enclitic} suffixes of Southern Yauyos Quechua. In \SYQ, as in other Quechuan languages, enclitics attach to both nouns and verbs as well as to adverbs and negators. Enclitics always follow all inflectional suffixes, verbal and nominal; and, with the exception of restrictive \phono{-lla}, all follow all case suffixes, as well. \SYQ{} counts sixteen enclitics. \phono{-Yá} (emphatic) indicates emphasis. Consistently translated in Spanish by \textit{pues}.\footnote{An anonymous reviewer points out that \it{pues} is used in Andean Spanish “to negotiate common ground, shared knowledge. As such, it is possible that \phono{-ya} is also an interactional or stance marker,” a way a participant in a conversation may negotiate what other participants know or should know.} \phono{-chu} (interrogation, negation, disjunction) indicates absolute and disjunctive questions, negation, and disjunction. \phono{-lla} (restrictive) generally indicates exclusivity or limitation in number; it is generally translated as ‘just’ or ‘only’. \phono{-lla} may express an affective or familiar attitude. \phono{-ña} (discontinuitive) indicates transition, change of state or quality. In affirmative statements, it is generally translated as ‘already’; in negative statements, as ‘no more’ or ‘no longer’; in questions, as ‘yet’. \phono{-pis} (inclusion) indicates the inclusion of an item or event into a series of similar items or events; it is generally translated as ‘too’ or ‘also’ or, when negated, ‘neither’. \phono{-puni} (certainty, precision); it is generally translated ‘necessarily’, ‘definitely’, ‘precisely’. This last is attested only in the \QII{} dialects, where it is infrequently employed. \phono{-qa} (topic marker) indicates the topic of the clause; it is generally left untranslated.\footnote{\phono{-qa} may nevertheless be indicated in Spanish translations by intonation, gesture, and various circumlocutions of speech, as an anonymous reviewer points out.}\\
\phono{-raq} (continuative) indicates continuity of action, state or quality. Translated ‘still’ or, negated, ‘yet’. \phono{-taq} (sequential) indicates the sequence of events. In this capacity, translated ‘then’ or ‘so’. \phono{-taq} also marks content questions. \phono{-mI} (evidential~--~direct experience) indicates that the speaker has personal-experience evidence for the proposition under the scope of the evidential. Usually left untranslated.\\
\phono{-shI} (evidential~--~reportative/quotative) indicates that the speaker has non-perso\-nal-experience evidence for the proposition under the scope of the evidential. \phono{-shI} appears systematically in stories. Often translated as ‘they say.’ \phono{-trI} (evidential~--~conjectural) indicates that the speaker is making a conjecture to the proposition under the scope of the evidential from a set of propositions for which she has either direct or not-direct evidence. Generally translated in Spanish as \spanish{seguro} ‘for sure’, indicating possibility or probability. \phono{-ari} (assertive force) indicates conviction on the part of the speaker. Translated as ‘certainly’ or ‘of course’.\footnote{An anonymous reviewer writes that in other varieties of Quechuan, “\phono{-ari} is interpersonal. It expresses solidarity, affirming what someone else says, thinks or believes to be true.”} \phono{-ik} and \phono{-iki} (evidential modifiers) indicate increasing evidence strength (and increased assertive force or conjectural certainty, in the case of the direct and conjectural modifiers, \phono{-mI} and \phono{-trI}, respectively). Generally translated in Spanish as \spanish{pues} and \spanish{seguro}, respectively. Examples in Table~\ref{Tab30} are fully glossed in the corresponding sections.

% TABLE 30
% \newcommand{\tabexefour}[4]{\Qyell{\phono{#1}}&#2&\Qyell{\textit{#3}}&#4\\}%
\begin{table}[!ht]
\renewcommand*\arraystretch{1.3}
\small\centering
\caption{Enclitic suffixes, with examples}\label{Tab30}
\begin{tabularx}{\textwidth}{p{6ex}@{~}p{13ex}@{~}L@{~}L}
\lsptoprule
\tabexefour{-Yá}{emphasis}{¡Mana-\pb{yá} rupa-chi-nchik-chu! ¡Ari-yá!}{‘We do \pb{\emph{not}} set on fire!’ \mbox{‘Yes, indeed!’}}
\tabexefour{-chu\tss{1}}{interrogation}{¿Iskwila-man trura-shu-rqa-nki-\pb{chu} mama-yki?}{‘\pb{Did} your mother put you in school?’}
\tabexefour{-chu\tss{2}}{negation}{Chay-tri \pb{mana} suya-wa-rqa-\pb{chu}.}{‘That must be why she would\pb{n’t} have waited for me.’}
\tabexefour{-chu\tss{3}}{disjunction}{¿Qari-\pb{chu} ka-nki warmi-\pb{chu} ka-nki?}{‘Are you a man \pb{or} a woman?’}
\tabexefour{-lla}{restriction}{Uma-\pb{lla}-ña traki-\pb{lla}-ña ka-ya-sa.}{‘There was \pb{only} the head \pb{only} the hand.’}
\tabexefour{-ña}{discontuity}{Chay-shi ni-n kundinadaw-\pb{ña}-m wak-qa ka-ya-n.}{‘That one, they say, is \pb{already} condemned.’}
\tabexefour{-pis}{inclusion}{Tukuy tuta tusha-n qaynintin-ta-\pb{pis}.}{‘They dance all night and the next day, \pb{too}.’}
\tabexefour{-puni}{certainty}{Mana-\pb{puni}-m.}{‘By no means’, ‘Not on your life’}
\tabexefour{-qa}{topic}{Mana yatra-q-ni-n-\pb{qa}.}{‘Those of them who didn’t know’}
\tabexefour{-raq}{continuity}{Kama-n-pi puñu-ku-ya-pti-n-\pb{raq} tari-ru-n.}{‘He found him \pb{still} sleeping in his bed.’}
\tabexefour{-taq}{sequence}{hinaptin-ña-\pb{taq}-shi}{‘\pb{then}’ ‘so’}
\tabexefour{-mI}{evidential-direct}{Yanga-ña-\pb{m} qipi-ku-sa puri-ni.}{‘In vain, I walk around carrying it.’}
\tabexefour{-shI}{evidential-reportative}{Qari-n-ta-\pb{sh} wañu-ra-chi-n.}{‘She killed her husband, \pb{they say}.’}
\tabexefour{-trI}{evidential-conjecture}{Awa-ya-n-\pb{tr-iki} kama-ta.}{‘He \pb{must} be weaving a blanket.’}
\tabexefour{-ari}{assertive force}{Chay-\pb{sh-ari} kanan avansa-ru-nqa.}{‘That one \pb{definitely} will advance now, \pb{they say}.’}
\tabexefour{-ikI}{evidential \-modification}{Kay-na-lla-\pb{m-iki} kay urqu-pa-qa yatra-nchik.}{‘Just like this we live on this mountain.’}
\lspbottomrule
\end{tabularx}
\end{table}

\section{Sequence}
Combinations of individual enclitics\index[sub]{enclitic!sequence} generally occur in the order indicated in the table below. In complementary distribution are: \phono{-raq} with \phono{-ña}; the evidentials with each other as well as with \phono{-qa}; \phono{-ari} with \phono{-ikI;} and \phono{-Yá} with \phono{-ikI}.

\begin{center}
\small
\begin{tabular}{*{9}{c}}
\lsptoprule
	&	&	&				&	&	& \phono{-qa}	&	&				\\
	&	&	&				&	&	& \phono{-mI}	&	&				\\
	&	&	& \phono{-Raq}	&	&	& \phono{-shI}	&	& \phono{-ikI}	\\
\phono{-lla} & \phono{-puni} & \phono{-pis} & \phono{-ña} & \phono{-taq} & \phono{-chu} & \phono{-trI} & \phono{-Yá} & \phono{-aRi}\\
\lspbottomrule
\end{tabular}
\end{center}

\section{Individual enclitics}\label{sec:indenc}
In \SYQ, as in other Quechuan languages, the enclitics can be divided into two classes: (a)~those which position the utterance with regard to others salient in the discourse (restrictive/limitative \phono{-lla}, discontinuative \phono{-ña}, additive \phono{-pis}, topic marking \phono{-qa}, continuative \phono{-Raq}, sequential \phono{-taq}, and interrogative/negative/disjunctive \phono{-chu}); and (b)~those that position the speaker with regard to the utterance (emphatic \phono{-YÁ}, certainty marker \phono{-puni}, and the evidentials \phono{-mi}, \phono{-shi}, and \phono{-tri} along with their modifiers \phono{-ik}, \phono{-iki}, and \phono{-aRi}.). §~\ref{ssec:emphatic}--\ref{ssec:emotive} cover all enclitics except the evidentials and their modifiers, in alphabetical order. The evidentials and their modifiers are the subject of §~\ref{ssec:evidence}.

\subsection{Emphatic \phono{-Yá}}\label{ssec:emphatic}\index[sub]{emphatic}
Realized as \phono{-yá} in all environments~(\ref{Glo6:Ari}--\ref{Glo6:Sirbisatatr}) except following an evidential, in which case both the \phono{I} of the evidential and the \phono{Y} of the emphatic are elided and \phono{Yá} is realized as \phono{á}~(\ref{Glo6:Balikushatr}--\ref{Glo6:Unayqa}).\\ 

% 1
\gloexe{Glo6:Ari}{}{amv}%
{¡Ari\pb{yá}!}%amv que first line
{\morglo{ari-yá}{yes-\lsc{emph}}}%morpheme+gloss
\glotran{Yes \pb{indeed}.}{}%eng+spa trans
{}{}%rec - time

% 2
\gloexe{Glo6:Mana}{}{amv}%
{¡Mana-\pb{yá} rupa-chi-nchik-chu!}%amv que first line
{\morglo{mana-yá}{no-\lsc{emph}}\morglo{rupa-chi-nchik-chu}{burn-\lsc{caus}-\lsc{1pl}-\lsc{neg}}}%morpheme+gloss
\glotran{We do \emph{\pb{not}} set on fire!}{}%eng+spa trans
{}{}%rec - time

% 3
\gloexe{Glo6:Pantyunpa}{}{amv}%
{Pantyunpa\pb{yá}. ¡Ima wasiypitr pampamushaq!}%amv que first line
{\morglo{pantyun-pa-\pb{yá}}{cemetery-\lsc{loc}-\lsc{emph}}\morglo{ima}{what}\morglo{wasi-y-pi-tr}{house-\lsc{1}-\lsc{loc}-\lsc{evc}}\morglo{pampa-mu-shaq}{bury-\lsc{cisl}-\lsc{1.fut}}}%morpheme+gloss
\glotran{In the cemetery\pb{!} I doubt I’m going to bury someone in my house.}{}%eng+spa trans
{}{}%rec - time

% 4
\gloexe{Glo6:Imayna}{}{amv}%
{¿Imayna\pb{yá} piru paykuna yatran warmi u qari?}%amv que first line
{\morglo{imayna-yá}{how-\lsc{emph}}\morglo{piru}{but}\morglo{pay-kuna}{they-\lsc{pl}}\morglo{yatra-n}{know-\lsc{3}}\morglo{warmi}{woman}\morglo{u}{or}\morglo{qari}{man}}%morpheme+gloss
\glotran{How \pb{ever} can they know if it will be a woman or a man?}{}%eng+spa trans
{}{}%rec - time

% 5
\gloexe{Glo6:Sirbisatatr}{}{amv}%
{Sirbisatatr mas mastaqa rantikurun. Sirbisatayá.}%amv que first line
{\morglo{sirbisa-ta-tr}{beer-\lsc{acc}-\lsc{evc}}\morglo{mas}{more}\morglo{mas-ta-qa}{more-\lsc{acc}-\lsc{top}}\morglo{ranti-ku-ru-n}{buy-\lsc{refl}-\lsc{urgt}-\lsc{3}}\morglo{sirbisa-ta-yá}{beer-\lsc{acc}-\lsc{emph}}}%morpheme+gloss
\glotran{\spkr~1: “They must have sold a lot more beer.” \spkr~2: “Beer, \pb{all right}!”}{}%eng+spa trans
{}{}%rec - time

% 6
\gloexe{Glo6:Balikushatr}{}{lt}%
{Balikushatr kara. Payta\pb{má} rikarani.}%lt que first line
{\morglo{baliku-sha-tr}{request.a.service-\lsc{prf}-\lsc{evc}}\morglo{ka-ra}{be-\lsc{pst}}\morglo{pay-ta-m-á}{he-\lsc{acc}-\lsc{evd}-\lsc{emph}}\morglo{rika-ra-ni}{see-\lsc{pst}-\lsc{1}}}%morpheme+gloss
\glotran{He must have been requested. I saw him.}{}%eng+spa trans
{}{}%rec - time

% 7
\gloexe{Glo6:Trabahayta}{}{ch}%
{Trabahayta kanan kumunalta trulala\pb{má}.}%ch que first line
{\morglo{trabaha-y-ta}{work-\lsc{inf}-\lsc{acc}}\morglo{kanan}{now}\morglo{kumunal-ta}{community-\lsc{acc}}\morglo{trula-la-m-á}{put-\lsc{pst}-\lsc{evd}-\lsc{emph}}}%morpheme+gloss
\glotran{Now he’s put the community to work.}{}%eng+spa trans
{}{}%rec - time

% 8
\gloexe{Glo6:Unayqa}{}{sp}%
{Unayqa Awkichanka inkantakura\pb{shá} wak altupa yantaman riptiki.}%sp que first line
{\morglo{unay-qa}{before-\lsc{top}}\morglo{Awkichanka}{Awkichanka}\morglo{inkanta-ku-ra-sh-á}{enchant-\lsc{refl}-\lsc{pst}-\lsc{evr}-\lsc{emph}}\morglo{wak}{\lsc{dem.d}}\morglo{altu-pa}{high-\lsc{loc}}\morglo{yanta-man}{firewood-\lsc{all}}\morglo{ri-pti-ki}{go-\lsc{subds}-\lsc{2}}}%morpheme+gloss
\glotran{In olden times, Awkichanka, too, bewitched, \pb{they say}, up hill if you went for firewood.}{}%eng+spa trans
{}{}%rec - time

\subsection{Interrogation, negation, disjunction \phono{-chu}}\label{ssec:innedi}
\phono{-chu} indicates absolute~(\ref{Glo6:Iskwilaman}) and disjunctive questions~(\ref{Glo6:Qari}), (\ref{Glo6:Don}), negation~(\ref{Glo6:Chaytri}), and disjunction~(\ref{Glo6:Kandilaryapa}).\footnote{An anonymous reviewer points out that in Huaylas Q, negative \phono{-tsu} is distinguished from polar question \phono{-ku}. Huaylas is not unique among Quechuan languages in making this distinction.}\\

% 1
\gloexe{Glo6:Iskwilaman}{}{amv}%
{¿Iskwilaman trurashurqanki\pb{chu} mamayki?}%amv que first line
{\morglo{iskwila-man}{school-\lsc{all}}\morglo{trura-shu-rqa-nki-chu}{put-\lsc{2.obj}-\lsc{pst}-\lsc{2}-\lsc{q}}\morglo{mama-yki}{mother-\lsc{3}}}%morpheme+gloss
\glotran{\pb{Did} your mother put you in school?}{}%eng+spa trans
{}{}%rec - time

% 2
\gloexe{Glo6:Qari}{}{amv}%
{¿Qari\pb{chu} kanki warmi\pb{chu} kanki?}%amv que first line
{\morglo{¿qari-chu}{man-\lsc{q}}\morglo{ka-nki}{be-\lsc{2}}\morglo{warmi-chu}{woman-\lsc{q}}\morglo{ka-nki}{be-\lsc{2}}}%morpheme+gloss
\glotran{Are you a man \pb{or} a woman?}{}%eng+spa trans
{}{}%rec - time

% 3
\gloexe{Glo6:Don}{}{amv}%
{¿Don Juan\pb{chu} icha alman\pb{chu} hamuyan?}%amv que first line
{\morglo{Don}{Don}\morglo{Juan-chu}{Juan-\lsc{q}}\morglo{icha}{or}\morglo{alma-n-chu}{soul-\lsc{3}-\lsc{q}}\morglo{hamu-ya-n}{come-\lsc{prog}-\lsc{3}}}%morpheme+gloss
\glotran{Is it Don Juan, \pb{or} is his spirit coming?}{}%eng+spa trans
{}{}%rec - time

% 4
\gloexe{Glo6:Chaytri}{}{amv}%
{Chaytri \pb{mana} suyawarqa\pb{chu}.}%amv que first line
{\morglo{chay-tri}{\lsc{dem.d}-\lsc{evc}}\morglo{mana}{no}\morglo{suya-wa-rqa-chu}{wait-\lsc{1.obj}-\lsc{pst}-\lsc{neg}}}%morpheme+gloss
\glotran{That’s why she would\pb{n’t} have waited for me.}{}%eng+spa trans
{}{}%rec - time

% 5
\gloexe{Glo6:Kandilaryapa}{}{amv}%
{Kandilaryapa\pb{chu} bintisinkupa\pb{chu}.}%amv que first line
{\morglo{kandilarya-pa-chu}{Candelaria-\lsc{loc}-\lsc{disj}}\morglo{binti-sinku-pa-chu}{twenty-five-\lsc{loc}-\lsc{disj}}}%morpheme+gloss
\glotran{\pb{Either} on Candelaria \pb{or} on the twenty-fifth.}{}%eng+spa trans
{}{}%rec - time

\noindent
Where it functions to indicate interrogation\index[sub]{interrogation!\phono{-chu}} or negation\index[sub]{negation!\phono{-chu}}, \phono{-chu} attaches to the sentence fragment that is the focus of the interrogation or negation~(\ref{Glo6:Chaypachu}).\\

% 6
\gloexe{Glo6:Chaypachu}{}{amv}%
{¿Chaypa\pb{chu} tumarqanki?}%amv que first line
{\morglo{chay-pa-chu}{\lsc{dem.d}-\lsc{loc}-\lsc{q}}\morglo{tuma-rqa-nki}{take-\lsc{pst}-\lsc{2}}}%morpheme+gloss
\glotran{Did you take [pictures] \pb{there}?}{}%eng+spa trans
{}{}%rec - time

\noindent
Where it functions to indicate disjunction\index[sub]{disjunction} --~in either disjunctive questions or disjunctive statements~-- \phono{-chu} generally attaches to each of the disjuncts~(\ref{Glo6:Mario}).\\

% 7
\gloexe{Glo6:Mario}{}{amv}%
{Mario\pb{chu} karqa Julián\pb{chu} karqa.}%amv que first line
{\morglo{Mario-chu}{Mario-\lsc{disj}}\morglo{ka-rqa}{be-\lsc{pst}}\morglo{Julián-chu}{Julián-\lsc{disj}}\morglo{ka-rqa}{be-\lsc{pst}}}%morpheme+gloss
\glotran{It was \pb{either} Mario \pb{or} Julián.}{}%eng+spa trans
{}{}%rec - time

\noindent
Questions that anticipate a negative answer are indicated by \phono{mana-chu}~(\ref{Glo6:Manachu}).\\

% 8
\gloexe{Glo6:Manachu}{}{ch}%
{¿\pb{Manachu} kuska linman?}%ch que first line
{\morglo{mana-chu}{no-\lsc{q}}\morglo{kuska}{together}\morglo{li-n-man}{go-\lsc{3}-\lsc{cond}}}%morpheme+gloss
\glotran{\pb{Couldn’t} they go together?}{}%eng+spa trans
{}{}%rec - time

\noindent
\phono{mana-chu} may also “soften” questions~(\ref{Glo6:Paysanu}).\\

% 9
\gloexe{Glo6:Paysanu}{}{amv}%
{Paysanu, ¿\pb{manachu} vakata rantiyta munanki?}%amv que first line
{\morglo{paysanu}{countryman}\morglo{mana-chu}{no-\lsc{q}}\morglo{vaka-ta}{cow-\lsc{acc}}\morglo{ranti-y-ta}{buy-\lsc{inf}-\lsc{acc}}\morglo{muna-nki}{want-\lsc{2}}}%morpheme+gloss
\glotran{My countryman, \pb{do you not} want to buy a cow?}{}%eng+spa trans
{}{}%rec - time

\noindent
It may also be used, like \phono{aw} ‘yes’, in the formation of tag questions~(\ref{Glo6:Lliw}).\\

% 10
\gloexe{Glo6:Lliw}{}{ach}%
{Lliw lliwtriki wañukushun, puchukashun entonces, ¿\pb{manachu}?}%ach que first line
{\morglo{lliw}{all}\morglo{lliw-tr-iki}{all-\lsc{evc}-\lsc{iki}}\morglo{wañu-ku-shun}{die-\lsc{refl}-\lsc{1pl.fut}}\morglo{puchuka-shun}{finish.off-\lsc{1pl.fut}}\morglo{intunsis}{therefore}\morglo{mana-chu}{no-\lsc{q}}}%morpheme+gloss
\glotran{We’ll all have to die, to finish off then, \pb{isn’t that so}?}{}%eng+spa trans
{}{}%rec - time

\noindent
In negative sentences, \phono{-chu} generally co-occurs with \phono{mana} ‘not’~(\ref{Glo6:mana}); \phono{-chu} is also licensed by additive enclitic \phono{-pis}~(\ref{Glo6:Kaspin}), (\ref{Glo6:Manchakushpa}) and \phono{ni} ‘nor’~(\ref{Glo6:Apuraw}), (\ref{Glo6:wayta}).\\

% 11
\gloexe{Glo6:mana}{}{lt}%
{Aa, \pb{mana}yá kan\pb{chu}. \pb{Mana}yá bula kan\pb{chu}.}%lt que first line
{\morglo{aa}{ah}\morglo{mana-yá}{no-\lsc{emph}}\morglo{ka-n-chu}{be-\lsc{3}-\lsc{neg}}\morglo{mana-yá}{no-\lsc{emph}}\morglo{bula}{ball}\morglo{ka-n-chu}{be-\lsc{3}-\lsc{neg}}}%morpheme+gloss
\glotran{Ah, there are\pb{n’t} any. There are\pb{n’t} any balls.}{}%eng+spa trans
{}{}%rec - time

% 12
\gloexe{Glo6:Kaspin}{}{amv}%
{Kaspin\pb{pis} kan\pb{chu}.}%amv que first line
{\morglo{kaspi-n-pis}{stick-\lsc{3}-\lsc{add}}\morglo{ka-n-chu}{be-\lsc{3}-\lsc{neg}}}%morpheme+gloss
\glotran{She does\pb{n’t} have a stick.}{}%eng+spa trans
{}{}%rec - time

% 13
\gloexe{Glo6:Manchakushpa}{}{ach}%
{Manchakushpa tuta\pb{s} puñu:\pb{chu}.}%ach que first line
{\morglo{mancha-ku-shpa}{scare-\lsc{refl}-\lsc{subis}}\morglo{tuta-s}{night-\lsc{add}}\morglo{puñu-:-chu}{sleep-\lsc{1}-\lsc{neg}}}%morpheme+gloss
\glotran{Being scared, I \pb{don’t} sleep at night.}{}%eng+spa trans
{}{}%rec - time

% 14
\gloexe{Glo6:Apuraw}{}{amv}%
{Apuraw wañururqariki. \pb{Ni} apanña\pb{chu}.}%amv que first line
{\morglo{apuraw}{quick}\morglo{wañu-ru-rqa-r-iki}{die-\lsc{urgt}-\lsc{pst}-\lsc{r}-\lsc{iki}}\morglo{ni}{nor}\morglo{apa-n-ña-chu}{bring-\lsc{3}-\lsc{disc}-\lsc{neg}}}%morpheme+gloss
\glotran{He died quickly. They \pb{didn’t even} bring him [to the hospital].}{}%eng+spa trans
{}{}%rec - time

% 15
\gloexe{Glo6:wayta}{}{amv}%
{\pb{Manam} wayta\pb{chu} \pb{ni} pishqu\pb{chu}.}%amv que first line
{\morglo{mana-m}{no-\lsc{evd}}\morglo{wayta-chu}{flower-\lsc{neg}}\morglo{ni}{nor}\morglo{pishqu-chu}{bird-\lsc{neg}}}%morpheme+gloss
\glotran{\pb{Neither} a flower \pb{nor} a bird.}{}%eng+spa trans
{}{}%rec - time

\noindent
In prohibitions, \phono{-chu} co-occurs with \phono{ama} ‘don’t’~(\ref{Glo6:wawqi}).\\

% 16
\gloexe{Glo6:wawqi}{}{ach}%
{“¡\pb{Ama} wawqi:taqa wañuchiy\pb{chu}!” niptinshi wañurachin paywantapis.}%ach que first line
{\morglo{ama}{\lsc{proh}}\morglo{wawqi-:-ta-qa}{brother-\lsc{1}-\lsc{acc}-\lsc{top}}\morglo{wañu-chi-y-chu}{die-\lsc{caus}-\lsc{imp}-\lsc{neg}}\morglo{ni-pti-n-shi}{say-\lsc{subds}-\lsc{3}-\lsc{evr}}\morglo{wañu-ra-chi-n}{die-\lsc{urgt}-\lsc{caus}-\lsc{3}}\morglo{pay-wan-ta-pis}{he-\lsc{instr}-\lsc{acc}-\lsc{add}}}%morpheme+gloss
\glotran{When he said, “\pb{Don’t} kill my brother!” they killed him with him, too.}{}%eng+spa trans
{}{}%rec - time

\noindent
\phono{-chu} does not appear in subordinate clauses, where negation is indicated with a negative particle alone~(\ref{Glo6:qali}), (\ref{Glo6:qatrachakunanpaq}).\footnote{An anonymous reviewer points out that elsewhere in Quechua, the correlates of negative \phono{-chu} typically can appear in subordinate clauses. There are no naturally-occurring examples of this in the Yauyos corpus.}\\

% 17
\gloexe{Glo6:qali}{}{ch}%
{\pb{Mana} qali kaptinqa ñuqanchikpis taqllakta hapishpa qaluwanchik.}%ch que first line
{\morglo{mana}{no}\morglo{qali}{man}\morglo{ka-pti-n-qa}{be-\lsc{subds}-\lsc{3}-\lsc{top}}\morglo{ñuqanchik-pis}{we-\lsc{add}}\morglo{taqlla-kta}{plow-\lsc{acc}}\morglo{hapi-shpa}{grab-\lsc{subis}}\morglo{qaluwa-nchik}{turn.earth-\lsc{1pl}}}%morpheme+gloss
\glotran{When there are \pb{no} \pb{men}, we grab the plow and turn the earth.}{}%eng+spa trans
{}{}%rec - time

% 18
\gloexe{Glo6:qatrachakunanpaq}{}{amv}%
{\pb{Mana} qatrachakunanpaq mandilchanta watachakun.}%amv que first line
{\morglo{mana}{no}\morglo{qatra-cha-ku-na-n-paq}{dirty-\lsc{fact}-\lsc{refl}-\lsc{nmlz}-\lsc{3}-\lsc{purp}}\morglo{mandil-cha-n-ta}{apron-\lsc{dim}-\lsc{3}-\lsc{acc}}\morglo{wata-cha-ku-n}{tie-\lsc{dim}-\lsc{refl}-\lsc{3}}}%morpheme+gloss
\glotran{She’s tying on an apron \pb{so} she \pb{doesn’t} get dirty.}{}%eng+spa trans
{}{}%rec - time

% 19
\gloexe{Glo6:lluqsiptiyki}{}{amv}%
{Manam lluqsiptiyki(qa *\pb{chu}), waqashaqmi.}%amv que first line
{\morglo{mana-m}{no-\lsc{evd}}\morglo{lluqsi-pti-yki-qa}{go.out-\lsc{subds}-\lsc{2}-\lsc{top}}\morglo{chu}{neg}\morglo{waqa-shaq-mi}{cry-\lsc{1.fut}-\lsc{evd}}}%morpheme+gloss
\glotran{\pb{If} you \pb{don’t} go, I’ll cry.}{}%eng+spa trans
{}{}%rec - time

\noindent
In negative sentences, \phono{-chu} never occurs on the same segment as does an evidential enclitic~(\ref{Glo6:lluqsirqanki}).\\

% 20
\gloexe{Glo6:lluqsirqanki}{}{amv}%
{Mana lluqsirqanki(*mi)\pb{chu}.}% que first line
{\morglo{mana}{no}\morglo{lluqsi-rqa-nki-mi-chu}{go.out-\lsc{pst}-\lsc{2}-\lsc{evd}-\lsc{neg}}}%morpheme+gloss
\glotran{You \pb{didn’t} leave.}{}%eng+spa trans
{}{}%rec - time

\noindent
Interrogative \phono{-chu} does not appear in questions using interrogative pronouns~(\ref{Glo6:hamurqa}).\footnote{\phono{¿*Pi-taq} \phono{hamu-n-chu?} \phono{¿*Pi-taq-chu} \phono{hamu-n?} ‘Who is coming?’}\\

% 21
\gloexe{Glo6:hamurqa}{}{amv}%
{*¿Pi hamurqa\pb{chu}?}% que first line
{\morglo{pi}{who}\morglo{hamu-rqa-chu}{come-\lsc{pst}-\lsc{neg}}}%morpheme+gloss
\glotran{\pb{Who} came?}{}%eng+spa trans
{}{}%rec - time

\subsection{Restrictive, limitative \phono{-lla}}
\phono{-lla} indicates exclusivity or limitation in number\index[sub]{restrictive}: the individual~(\ref{Glo6:Iskwilapam}--\ref{Glo6:Kichwa}) or event/event type~(\ref{Glo6:Fwirti}), (\ref{Glo6:lliwtam}) remains limited to itself and is accompanied by no other.\\

% 1
\gloexe{Glo6:Iskwilapam}{}{sp}%
{Iskwilapam niytu:kunaqa wawa:kunaqa rinmi ñuqa\pb{lla}m ka: analfabitu.}%sp que first line
{\morglo{iskwila-pa-m}{school-\lsc{loc}-\lsc{evd}}\morglo{niytu-:-kuna-qa}{nephew-\lsc{1}-\lsc{pl}-\lsc{top}}\morglo{wawa-:-kuna-qa}{baby-\lsc{1}-\lsc{pl}-\lsc{top}}\morglo{ri-n-mi}{go-\lsc{3}-\lsc{evd}}\morglo{ñuqa-lla-m}{I-\lsc{rstr}-\lsc{evd}}\morglo{ka-:}{be-\lsc{1}}\morglo{analfabitu}{illiterate}}%morpheme+gloss
\glotran{My grandchildren are in school. My children went. I’m the \pb{only} illiterate one.}{}%eng+spa trans
{}{}%rec - time

% 2
\gloexe{Glo6:Runapi}{}{amv}%
{Runapi uma\pb{lla}ña traki\pb{lla}ña kayasa.}%amv que first line
{\morglo{runa-pi}{person-\lsc{gen}}\morglo{uma-lla-ña}{head-\lsc{rstr}-\lsc{disc}}\morglo{traki-lla-ña}{foot-\lsc{rstr}-\lsc{disc}}\morglo{ka-ya-sa}{be-\lsc{prog}-\lsc{npst}}}%morpheme+gloss
\glotran{\pb{Just} the head and the hand remained of the person.}{}%eng+spa trans
{}{}%rec - time

% 3
\gloexe{Glo6:Kichwa}{}{ch}%
{Kichwa\pb{lla}ktam limakuya: kaytrawlaq manam kastillanukta lima:chu.}%ch que first line
{\morglo{kichwa-lla-kta-m}{Quechua-\lsc{rstr}-\lsc{acc}-\lsc{evd}}\morglo{lima-ku-ya-:}{speak-\lsc{refl}-\lsc{prog}-\lsc{1}}\morglo{kay-traw-laq}{\lsc{dem.p}-\lsc{loc}-\lsc{cont}}\morglo{mana-m}{no-\lsc{evd}}\morglo{kastillanu-kta}{Spanish-\lsc{acc}}\morglo{lima-:-chu}{speak-\lsc{1}-\lsc{neg}}}%morpheme+gloss
\glotran{I’m talking \pb{just} Quechua. Here, still, we don’t speak Spanish.}{}%eng+spa trans
{}{}%rec - time

% 4
\gloexe{Glo6:Fwirti}{}{ch}%
{Fwirti kashpa\pb{lla}má linchik pustaman.}%ch que first line
{\morglo{fwirti}{strong}\morglo{ka-shpa-lla-m-á}{be-\lsc{subis}-\lsc{rstr}-\lsc{evd}-\lsc{emph}}\morglo{li-nchik}{go-\lsc{1pl}}\morglo{pusta-man}{clinic-\lsc{all}}}%morpheme+gloss
\glotran{\pb{Only} if it’s bad will we go to the health clinic.}{}%eng+spa trans
{}{}%rec - time

% 5
\gloexe{Glo6:lliwtam}{}{ach}%
{Lliw lliwtam rantishpa\pb{lla}ñam kanan kamatapis chay polarkunatapis.}%ach que first line
{\morglo{lliw}{all}\morglo{lliw-ta-m}{all-\lsc{acc}-\lsc{evd}}\morglo{ranti-shpa-lla-ña-m}{buy-\lsc{subis}-\lsc{rstr}-\lsc{disc}-\lsc{evd}}\morglo{kanan}{now}\morglo{kama-ta-pis}{blanket-\lsc{acc}-\lsc{add}}\morglo{chay}{\lsc{dem.d}}\morglo{polar-kuna-ta-pis}{fleece-\lsc{pl}-\lsc{acc}-\lsc{add}}}%morpheme+gloss
\glotran{Now \pb{just} buying everything -- blankets, [polyester] fleece.}{}%eng+spa trans
{}{}%rec - time

\noindent
\phono{-lla} can generally be translated as ‘just’~(\ref{Glo6:Chayna}), (\ref{Glo6:Sirka}) or ‘only’~(\ref{Glo6:Chay}); it sometimes has an ‘exactly’ interpretation~(\ref{Glo6:Iskinanpi}).\\

% 6
\gloexe{Glo6:Chayna}{}{amv}%
{Chayna\pb{lla}m mikuchin~\dots{} pachachin.}%amv que first line
{\morglo{chayna-\pb{lla}-m}{thus-\lsc{rstr}-\lsc{evd}}\morglo{miku-chi-n}{eat-\lsc{caus}-\lsc{3}}\morglo{pacha-chi-n}{dress-\lsc{caus}-\lsc{3}}}%morpheme+gloss
\glotran{\pb{Just} like that, she feeds him, she clothes him.}{}%eng+spa trans
{}{}%rec - time

% 7
\gloexe{Glo6:Sirka}{}{sp}%
{Sirka\pb{lla}tam riya: manam karutachu.}%sp que first line
{\morglo{sirka-lla-ta-m}{close-\lsc{rstr}-\lsc{acc}-\lsc{evd}}\morglo{ri-ya-:}{go-\lsc{prog}-\lsc{1}}\morglo{mana-m}{no-\lsc{evd}}\morglo{karu-ta-chu}{far-\lsc{acc}-\lsc{neg}}}%morpheme+gloss
\glotran{I \pb{just} go close; I don’t go far.}{}%eng+spa trans
{}{}%rec - time

% 8
\gloexe{Glo6:Chay}{}{amv}%
{Chay\pb{lla}tam yatrani. Masta yatranichu.}%amv que first line
{\morglo{chay-lla-ta-m}{\lsc{dem.d}-\lsc{lim}-\lsc{acc}-\lsc{evd}}\morglo{yatra-ni}{know-\lsc{1}}\morglo{mas-ta}{more-\lsc{acc}}\morglo{yatra-ni-chu}{know-\lsc{1}-\lsc{neg}}}%morpheme+gloss
\glotran{I \pb{only} know that. I don’t know more.}{}%eng+spa trans
{}{}%rec - time

% 9
\gloexe{Glo6:Iskinanpi}{}{lt}%
{Iskinanpi sikya tuna\pb{lla}npi wallpay watrakunraq.}%lt que first line
{\morglo{iskina-n-pi}{corner-\lsc{3}-\lsc{loc}}\morglo{sikya}{aqueduct}\morglo{tuna-lla-n-pi}{corner-\lsc{rstr}-\lsc{3}-\lsc{loc}}\morglo{wallpa-y}{chicken-\lsc{1}}\morglo{watra-ku-n-raq}{give.birth-\lsc{refl}-\lsc{3}-\lsc{cont}}}%morpheme+gloss
\glotran{My hen lays eggs in the corner, \pb{right} in the corner of the canal.}{}%eng+spa trans
{}{}%rec - time

\noindent
It is very, very widely employed~(\ref{Glo6:abaskuna}--\ref{Glo6:Chaytam}).\\

% 10
\gloexe{Glo6:abaskuna}{}{amv}%
{Lliwta abaskuna albirhakuna ayvis\pb{lla} rantikuni apani llaqtatam.}%amv que first line
{\morglo{lliw-ta}{all-\lsc{acc}}\morglo{abas-kuna}{broad.beans-\lsc{pl}}\morglo{albirha-kuna}{peas-\lsc{pl}}\morglo{ayvis-lla}{sometimes-\lsc{rstr}}\morglo{ranti-ku-ni}{buy-\lsc{refl}-\lsc{1}}\morglo{apa-ni}{bring-\lsc{1}}\morglo{llaqta-ta-m}{town-\lsc{acc}-\lsc{evd}}}%morpheme+gloss
\glotran{Everything -- broad beans, peas -- \pb{once in while} I sell stuff -- I bring it into town.}{}%eng+spa trans
{}{}%rec - time

% 11
\gloexe{Glo6:kwintuqa}{}{sp}%
{Chayna\pb{lla}m. Chay\pb{lla}m kwintuqa. Mas kanchu manam.}%sp que first line
{\morglo{chayna-lla-m}{thus-\lsc{rstr}-\lsc{evd}}\morglo{chay-lla-m}{\lsc{dem.d}-\lsc{rstr}-\lsc{evd}}\morglo{kwintu-qa}{story-\lsc{top}}\morglo{mas}{more}\morglo{ka-n-chu}{be-\lsc{3}-\lsc{neg}}\morglo{mana-m}{no-\lsc{evd}}}%morpheme+gloss
\glotran{That’s the way it goes. That’s \pb{all} there is to the story. There’s no more.}{}%eng+spa trans
{}{}%rec - time

% 12
\gloexe{Glo6:Chaytam}{}{amv}%
{Chaytam aysashpa\pb{lla} pasachiwaq.}%amv que first line
{\morglo{chay-ta-m}{\lsc{dem.d}-\lsc{acc}-\lsc{evd}}\morglo{aysa-shpa-lla}{pull-\lsc{subis}-\lsc{rstr}}\morglo{pasa-chi-wa-q}{pass-\lsc{caus}-\lsc{1.obj}-\lsc{ag}}}%morpheme+gloss
\glotran{They had me cross the river pulling [me by the hand].}{}%eng+spa trans
{}{}%rec - time

\subsection{Discontinuative \phono{-ña}}
Discontinuitive. \phono{-ña}\index[sub]{discontinuitive} indicates transition --~change of state or quality. In affirmative statements, it can generally be translated as ‘already’~(\ref{Glo6:Kundinadaw}--\ref{Glo6:Paqwayanchik}); in negative statements, as ‘no more’ or ‘no longer’~(\ref{Glo6:Unaytrik}), (\ref{Glo6:Manana}); and in questions, as ‘yet’~(\ref{Glo6:Pasarun}), (\ref{Glo6:Rimaya}).\\

% 1
\gloexe{Glo6:Kundinadaw}{}{amv}%
{Kundinadaw\pb{ña}m wakqa kayan.}%amv que first line
{\morglo{kundinadaw-ña-m}{condemned-\lsc{disc}-\lsc{evd}}\morglo{wak-qa}{\lsc{dem.d}-\lsc{top}}\morglo{ka-ya-n}{be-\lsc{prog}-\lsc{3}}}%morpheme+gloss
\glotran{That one is \pb{already} condemned.}{}%eng+spa trans
{}{}%rec - time

% 2
\gloexe{Glo6:kukaywan}{}{amv}%
{Ñuqaqa kukaywan\pb{ña}m qawaruni.}%amv que first line
{\morglo{ñuqa-qa}{I-\lsc{top}}\morglo{kuka-y-wan-ña-m}{coca-\lsc{1}-\lsc{instr}-\lsc{disc}-\lsc{evd}}\morglo{qawa-ru-ni}{see-\lsc{urgt}-\lsc{1}}}%morpheme+gloss
\glotran{I saw it with my coca \pb{already}.}{}%eng+spa trans
{}{}%rec - time

% 3
\gloexe{Glo6:Paqwayanchik}{}{ch}%
{Paqwayanchik\pb{ña}m talpuyta, ¿aw? Papaktapis talpulalu:\pb{ña}m, kanan halakta, ¿aw?}%ch que first line
{\morglo{paqwa-ya-nchik-ña-m}{finish-\lsc{prog}-\lsc{1pl}-\lsc{disc}-\lsc{evd}}\morglo{talpu-y-ta}{plant-\lsc{inf}-\lsc{acc}}\morglo{aw}{yes}\morglo{papa-kta-pis}{potato-\lsc{acc}-\lsc{add}}\morglo{talpu-la-lu-:-ña-m}{plant-\lsc{unint}-\lsc{urgt}-\lsc{1}-\lsc{disc}-\lsc{evd}}\morglo{kanan}{now}\morglo{hala-kta}{corn-\lsc{acc}}\morglo{aw}{yes}}%morpheme+gloss
\glotran{We’re finishing the planting \pb{already}, no? We’ve \pb{already} planted the potatoes, now the corn, no?}{}%eng+spa trans
{}{}%rec - time

% 4
\gloexe{Glo6:Unaytrik}{}{sp}%
{Unaytrik. Kananqa kan\pb{ña}chu imapis.}%sp que first line
{\morglo{unay-tri-k}{before-\lsc{evc}-\lsc{ik}}\morglo{kanan-qa}{now-\lsc{top}}\morglo{ka-n-ña-chu}{be-\lsc{3}-\lsc{disc}-\lsc{neg}}\morglo{ima-pis}{what-\lsc{add}}}%morpheme+gloss
\glotran{That would be a long time ago. Now there isn’t anything \pb{any more}.}{}%eng+spa trans
{}{}%rec - time

% 5
\gloexe{Glo6:Manana}{}{amv}%
{\pb{Manaña} ni santu ni imapis.}%amv que first line
{\morglo{mana-ña}{no-\lsc{disc}}\morglo{ni}{nor}\morglo{santu}{saint}\morglo{ni}{nor}\morglo{ima-pis}{what-\lsc{add}}}%morpheme+gloss
\glotran{There are \pb{no longer} saints or anything.}{}%eng+spa trans
{}{}%rec - time

% 6
\gloexe{Glo6:Pasarun}{}{amv}%
{¿Pasarun\pb{ñachu}? Tapushun.}%amv que first line
{\morglo{pasa-ru-n-ña-chu}{pass-\lsc{urgt}-\lsc{3}-\lsc{disc}-\lsc{q}}\morglo{tapu-shun}{ask-\lsc{1pl.fut}}}%morpheme+gloss
\glotran{Did she go by \pb{yet}? Let’s ask.}{}%eng+spa trans
{}{}%rec - time

% 7
\gloexe{Glo6:Rimaya}{}{lt}%
{¿Rimaya\pb{nña}\pb{chu} kanan wakpi?}%lt que first line
{\morglo{rima-ya-n-ña-chu}{talk-\lsc{prog}-\lsc{3}-\lsc{disc}-\lsc{q}}\morglo{kanan}{now}\morglo{wak-pi}{\lsc{dem.d}-\lsc{loc}}}%morpheme+gloss
\glotran{Are they talking \pb{yet} there now?}{}%eng+spa trans
{}{}%rec - time

\noindent
It can appear freely but never unaccompanied, redundantly, by \phono{ña}~(\ref{Glo6:tukuchkani}), (\ref{Glo6:riqsiyan}).\\

% 8
\gloexe{Glo6:tukuchkani}{}{amv}%
{“¡\pb{Ñam} tukuchkani\pb{ña}!” ¡Puk! ¡Puk! ¡Puk! sikisapa sapu.}%amv que first line
{\morglo{ña-m}{\lsc{disc}-\lsc{evd}}\morglo{tuku-chka-ni-ña}{finish-\lsc{dur}-\lsc{1}-\lsc{disc}}\morglo{puk}{puk}\morglo{puk}{puk}\morglo{puk}{puk}\morglo{siki-sapa}{behind-\lsc{mult.poss}}\morglo{sapu}{frog}}%morpheme+gloss
\glotran{“I’m \pb{already} finishing up!” Puk! Puk! Puk! said the frog with the behind bigger than usual.}{}%eng+spa trans
{}{}%rec - time

% 9
\gloexe{Glo6:riqsiyan}{}{lt}%
{\pb{Ñam} riqsiyan\pb{ña} hukya yaykun.}%lt que first line
{\morglo{ña-m}{\lsc{disc}-\lsc{evd}}\morglo{riqsi-ya-n-ña}{know-\lsc{prog}-\lsc{3}-\lsc{disc}}\morglo{huk-ya}{one-\lsc{emph}}\morglo{yayku-n}{enter-\lsc{3}}}%morpheme+gloss
\glotran{They’re getting to know it \pb{already} and another comes in.}{}%eng+spa trans
{}{}%rec - time

\subsection{Inclusion \phono{-pis}}
\phono{-pis}\index[sub]{inclusion} indicates the inclusion of an item or event into a series of similar items or events. Translated as ‘and’, ‘too’, ‘also’, and ‘even’~(\ref{Glo6:Turnuchawan}--\ref{Glo6:Mamanwa}) or, when negated, ‘neither’ or ‘not even’~(\ref{Glo6:Imapaqtaq}--\ref{Glo6:Pata}).\\

% 1
\gloexe{Glo6:Turnuchawan}{}{ch}%
{Turnuchawan ñuqakunaqa trabaha: walmi\pb{pis} qali\pb{pis}.}%ch que first line
{\morglo{turnu-cha-wan}{turn-\lsc{dim}-\lsc{instr}}\morglo{ñuqa-kuna-qa}{I-\lsc{pl}-\lsc{top}}\morglo{trabaha-:}{work-\lsc{1}}\morglo{walmi-pis}{woman-\lsc{add}}\morglo{qali-pis}{man-\lsc{add}}}%morpheme+gloss
\glotran{We work in turns, the women \pb{and} the men.}{}%eng+spa trans
{}{}%rec - time

% 2
\gloexe{Glo6:Tukuy}{}{amv}%
{Tukuy tuta tushun qaynintinta\pb{pis}.}%amv que first line
{\morglo{tukuy}{all}\morglo{tuta}{night}\morglo{tushu-n}{dance-\lsc{3}}\morglo{qaynintin-ta-pis}{next.day-\lsc{acc}-\lsc{add}}}%morpheme+gloss
\glotran{They dance all night and the next day, \pb{too}.}{}%eng+spa trans
{}{}%rec - time

% 3
\gloexe{Glo6:subrinu}{}{amv}%
{Pay\pb{pis} chay subrinu wañukuptinñamik payqa tumarun.}%amv que first line
{\morglo{pay-pis}{he-\lsc{add}}\morglo{chay}{\lsc{dem.d}}\morglo{subrinu}{nephew}\morglo{wañu-ku-pti-n-ña-mi-k}{die-\lsc{refl}-\lsc{subds}-\lsc{3}-\lsc{disc}-\lsc{evd-\lsc{ik}}}\morglo{pay-qa}{he-\lsc{top}}\morglo{tuma-ru-n}{take-\lsc{urgt}-\lsc{3}}}%morpheme+gloss
\glotran{He, \pb{too}, when his nephew died, took it [poison].}{}%eng+spa trans
{}{}%rec - time

% 4
\gloexe{Glo6:Salchipullu}{}{amv}%
{Salchipullu rantikuqta\pb{pis} tumarun.}%amv que first line
{\morglo{salchipullu}{fried.chicken}\morglo{ranti-ku-q-ta-pis}{buy-\lsc{refl}-\lsc{ag}-\lsc{acc}-\lsc{add}}\morglo{tuma-ru-n}{take-\lsc{urgt}-\lsc{3}}}%morpheme+gloss
\glotran{She took [pictures] of the people selling fried chicken \pb{also}.}{}%eng+spa trans
{}{}%rec - time

% 5
\gloexe{Glo6:Mamanwa}{}{amv}%
{Maman wañukuptin\pb{pis} manam waqanchu.}%amv que first line
{\morglo{mama-n}{mother-\lsc{3}}\morglo{wañu-ku-pti-n-pis}{die-\lsc{refl}-\lsc{subds}-\lsc{3}-\lsc{add}}\morglo{mana-m}{no-\lsc{evd}}\morglo{waqa-n-chu}{cry-\lsc{3}-\lsc{neg}}}%morpheme+gloss
\glotran{\pb{Even} when his mother died, he didn’t cry.}{}%eng+spa trans
{}{}%rec - time

% 6
\gloexe{Glo6:Imapaqtaq}{}{amv}%
{“¿Imapaqtaq ñuqa waqashaq?” nin. “Warmiypaq\pb{pis} waqarqani\pb{chu}.”}%amv que first line
{\morglo{ima-paq-taq}{what-\lsc{purp}-\lsc{seq}}\morglo{ñuqa}{I}\morglo{waqa-shaq}{cry-\lsc{1.fut}}\morglo{nin}{say-\lsc{3}}\morglo{warmi-y-paq-pis}{woman-\lsc{1}-\lsc{ben}-\lsc{add}}\morglo{waqa-rqa-ni-chu}{cry-\lsc{pst}-\lsc{1}-\lsc{neg}}}%morpheme+gloss
\glotran{“Why am I going to cry?” he said. “I did\pb{n’t} cry for my wife, \pb{either}.”}{}%eng+spa trans
{}{}%rec - time

% 7
\gloexe{Glo6:Paykunaqa}{}{amv}%
{Paykunaqa \pb{manam} qawarqa\pb{pischu}.}%amv que first line
{\morglo{pay-kuna-qa}{he-\lsc{pl}-\lsc{top}}\morglo{mana-m}{no-\lsc{evd}}\morglo{qawa-rqa-pis-chu}{see-\lsc{pst}-\lsc{add}-\lsc{neg}}}%morpheme+gloss
\glotran{\pb{Neither} did they see us.}{}%eng+spa trans
{}{}%rec - time

% 8
\gloexe{Glo6:Pata}{}{amv}%
{Pata saqayta\pb{pis} atipan\pb{chu}.}%amv que first line
{\morglo{pata}{terrace}\morglo{saqa-y-ta-pis}{go.up-\lsc{inf}-\lsc{acc}-\lsc{add}}\morglo{atipa-n-chu}{be.able-\lsc{3}-\lsc{neg}}}%morpheme+gloss
\glotran{They ca\pb{n’t} \pb{even} go up one terrace.}{}%eng+spa trans
{}{}%rec - time

\noindent
\phono{-pis} may --~or, even, may generally~-- imply contrast with some preceding element. Where it scopes over subordinate clauses, it can often be translated ‘although’ or ‘even’~(\ref{Glo6:Uratam}), (\ref{Glo6:Hinaptin}).\\

% 9
\gloexe{Glo6:Uratam}{}{amv}%
{Uratam muna\pb{shpapis}.}%amv que first line
{\morglo{ura-ta-m}{hour-\lsc{acc}-\lsc{evd}}\morglo{muna-shpa-pis}{want-\lsc{subis}-\lsc{add}}}%morpheme+gloss
\glotran{\pb{Although} I want to know the time.}{}%eng+spa trans
{}{}%rec - time

% 10
\gloexe{Glo6:Hinaptin}{}{sp}%
{Hinaptin wasipiña rumiwan takaptin\pb{pis} uyan\pb{chu}.}%sp que first line
{\morglo{hinaptin}{then}\morglo{wasi-pi-ña}{house-\lsc{loc}-\lsc{disc}}\morglo{rumi-wan}{stone-\lsc{instr}}\morglo{taka-pti-n-pis}{hit-\lsc{subds}-\lsc{3}-\lsc{add}}\morglo{uya-n-chu}{be.able-\lsc{3}-\lsc{neg}}}%morpheme+gloss
\glotran{Later, at home, \pb{even when} they hit it with a rock, it couldn’t.}{}%eng+spa trans
{}{}%rec - time

\noindent
Attaching to interrogative-indefinite stems, it forms indefinites and, with \phono{mana}, negative indefinites~(\ref{Glo6:Chaynam}--\ref{Glo6:chambyakuqpaq}).\\

% 11
\gloexe{Glo6:Chaynam}{}{amv}%
{Chaynam \pb{imallatapis} wasiman apamun.}%amv que first line
{\morglo{chayna-m}{thus-\lsc{evd}}\morglo{ima-lla-ta-pis}{what-\lsc{rstr}-\lsc{acc}-\lsc{add}}\morglo{wasi-man}{house-\lsc{all}}\morglo{apa-mu-n}{bring-\lsc{cisl}-\lsc{3}}}%morpheme+gloss
\glotran{That way he brings a little \pb{something} to his house.}{}%eng+spa trans
{}{}%rec - time

% 12
\gloexe{Glo6:tiyndaman}{}{ach}%
{Llapa tiyndaman yaykushpaqa lliw lliwshi \pb{imantapis} apakun.}%ach que first line
{\morglo{llapa}{all}\morglo{tiynda-man}{store-\lsc{all}}\morglo{yayku-shpa-qa}{enter-\lsc{subis}-\lsc{top}}\morglo{lliw}{all}\morglo{lliw-shi}{all-\lsc{evr}}\morglo{ima-n-ta-pis}{what-\lsc{3}-\lsc{acc}-\lsc{add}}\morglo{apa-ku-n}{bring-\lsc{refl}-\lsc{3}}}%morpheme+gloss
\glotran{They entered all the stores and took everything and \pb{anything} they had.}{}%eng+spa trans
{}{}%rec - time

% 13
\gloexe{Glo6:chambyakuqpaq}{}{amv}%
{Alli chambyakuqpaq \pb{manam imapis} faltanmanchu.}%amv que first line
{\morglo{alli}{good}\morglo{chambya-ku-q-paq}{work-\lsc{refl}-\lsc{ag}-\lsc{ben}}\morglo{mana}{no}\morglo{ima-pis}{what-\lsc{add}}\morglo{falta-n-man-chu}{be.missing-\lsc{3}-\lsc{cond}-\lsc{neg}}}%morpheme+gloss
\glotran{\pb{Nothing} can be lacking for a good worker.}{}%eng+spa trans
{}{}%rec - time

\noindent
It is in free variation with \phono{-pas}, and, after a vowel, with \phono{-s}~(\ref{Glo6:Diskansakamuy}--\ref{Glo6:harquruwara}), the latter particularly common in the \ACH{} dialect.\\

% 14
\gloexe{Glo6:Diskansakamuy}{}{lt}%
{“¡Diskansakamuy wasikipa!” niwan kikin\pb{pas} diskansuman ripun.}%lt que first line
{\morglo{diskansa-ka-mu-y}{rest-\lsc{refl}-\lsc{cisl}-\lsc{imp}}\morglo{wasi-ki-pa}{house-\lsc{2}-\lsc{loc}}\morglo{ni-wa-n}{say-\lsc{1.obj}-\lsc{3}}\morglo{kiki-n-pas}{self-\lsc{3}-\lsc{add}}\morglo{diskansu-man}{rest-\lsc{all}}\morglo{ripu-n}{go-\lsc{3}}}%morpheme+gloss
\glotran{“Go rest in your house,” he said to me and he, himself, \pb{too}, went to rest.}{}%eng+spa trans
{}{}%rec - time

% 15
\gloexe{Glo6:Hinaptinqa}{}{sp}%
{Hinaptinqa yutu pawaptinqa chay, “¡Aaaapship ship ship!” Yutu\pb{pas} “¡Wwaaaayyy!”}%sp que first line
{\morglo{hinaptin-qa}{then-\lsc{top}}\morglo{yutu}{partridge}\morglo{pawa-pti-n-qa}{fly-\lsc{subds}-\lsc{3}-\lsc{top}}\morglo{chay}{\lsc{dem.d}}\morglo{aaaapship}{aaaapship}\morglo{ship}{ship}\morglo{ship}{ship}\morglo{yutu-pas}{partridge-\lsc{add}}\morglo{wwaaaayyy}{wwaaaayyy}}%morpheme+gloss
\glotran{Then, when the partridge jumped, he [cried], “Aaaap-ship-ship-ship!” The partridge, \pb{too}, [cried] “Wwaaaayyy!”}{}%eng+spa trans
{}{}%rec - time

% 16
\gloexe{Glo6:harquruwara}{}{lt}%
{Ñuqata\pb{s} harquruwara Kashapataman riranim.}%lt que first line
{\morglo{ñuqa-ta-s}{I-\lsc{acc}-\lsc{add}}\morglo{harqu-ru-wa-ra}{toss.out-\lsc{urgt}-\lsc{1.obj}-\lsc{pst}}\morglo{Kashapata-man}{Kashapata-\lsc{all}}\morglo{ri-ra-ni-m}{go-\lsc{pst}-\lsc{1}-\lsc{evd}}}%morpheme+gloss
\glotran{They threw me out, \pb{too}, and I went to Kashapata.}{}%eng+spa trans
{}{}%rec - time

\subsection{Precision, certainty \phono{-puni}}
\phono{-puni} indicates certainty\index[sub]{certainty} or precision\index[sub]{precision}. It can be translated as ‘necessarily’, ‘definitely’, ‘precisely’. It is attested only in the \AMV{} dialect, where, still, it is not widely employed.\\

% 1
\gloexe{Glo6:Paqarin}{}{amv}%
{Paqarin\pb{puni}m rishaq.~\updag}%amv que first line
{\morglo{paqarin-puni-m}{tomorrow-\lsc{cert}-\lsc{evd}}\morglo{ri-shaq}{go-\lsc{1.fut}}}%morpheme+gloss
\glotran{I’m going to go \pb{precisely} tomorrow.}{}%eng+spa trans
{}{}%rec - time

% 2
\gloexe{Glo6:puni}{}{amv}%
{Mana\pb{puni}m.~\updag}%amv que first line
{\morglo{mana-puni-m}{no-\lsc{cert}-\lsc{evd}}}%morpheme+gloss
\glotran{By no means.}{}%eng+spa trans
{}{}%rec - time

% 3
\gloexe{Glo6:wiqawninchikman}{}{amv}%
{Chay wiqawninchikman\pb{puni} chiri yakuta truranchik.}%amv que first line
{\morglo{chay}{\lsc{dem.d}}\morglo{wiqaw-ni-nchik-man-puni}{waist-\lsc{euph}-\lsc{1pl}-\lsc{all}-\lsc{cert}}\morglo{chiri}{cold}\morglo{yaku-ta}{water-\lsc{acc}}\morglo{trura-nchik}{put-\lsc{1pl}}}%morpheme+gloss
\glotran{We put cold water \pb{right} on our lower backs.}{}%eng+spa trans
{}{}%rec - time

\subsection{Topic-marking \phono{-qa}}\label{ssec:topic}
\phono{-qa}\index[sub]{topic marker} indicates the topic of a clause~(\ref{Glo6:sultiram}--\ref{Glo6:Difindiwanchik}), including in those cases where it attaches to subordinate clauses~(\ref{Glo6:pasiyuman}), (\ref{Glo6:Qipiruptinqa}).\\

% 1
\gloexe{Glo6:sultiram}{}{ch}%
{Madri sultiram kaya: ñuqalla\pb{qa}.}%ch que first line
{\morglo{madri}{mother}\morglo{sultira-m}{alone-\lsc{evd}}\morglo{ka-ya-:}{be-\lsc{prog}-\lsc{1}}\morglo{ñuqa-lla-qa}{I-\lsc{rstr}-\lsc{top}}}%morpheme+gloss
\glotran{\pb{I}’m a single mother.}{}%eng+spa trans
{}{}%rec - time

% 2
\gloexe{Glo6:Ganawniyki}{}{lt}%
{Ganawniyki\pb{qa} achkam miranqa.}%lt que first line
{\morglo{ganaw-ni-yki-qa}{cattle-\lsc{euph}-\lsc{2}-\lsc{top}}\morglo{achka-m}{a.lot-\lsc{evd}}\morglo{mira-nqa}{increase-\lsc{3.fut}}}%morpheme+gloss
\glotran{Your \pb{cattle} are going to multiply a lot.}{}%eng+spa trans
{}{}%rec - time

% 3
\gloexe{Glo6:waqakunki}{}{sp}%
{Qam\pb{qa} waqakunki sumaqllatam. Ñuqa\pb{qa} quyu quyuta waqayani.}%sp que first line
{\morglo{qam-qa}{you-\lsc{top}}\morglo{waqa-ku-nki}{cry-\lsc{refl}-\lsc{2}}\morglo{sumaq-lla-ta-m}{pretty-\lsc{rstr}-\lsc{acc}-\lsc{evd}}\morglo{ñuqa-\pb{qa}}{I-\lsc{top}}\morglo{quyu}{ugly}\morglo{quyu-ta}{ugly-\lsc{acc}}\morglo{waqa-ya-ni}{cry-\lsc{prog}-\lsc{1}}}%morpheme+gloss
\glotran{\pb{You} sing nicely. \pb{I}’m singing awfully.}{}%eng+spa trans
{}{}%rec - time

% 4
\gloexe{Glo6:Yatraqnin}{}{amv}%
{Yatraqnin\pb{qa}; mana yatraqnin\pb{qa} manayá.}%amv que first line
{\morglo{yatra-q-ni-n-qa}{know-\lsc{ag}-\lsc{euph}-\lsc{3}-\lsc{top}}\morglo{mana}{no}\morglo{yatra-q-ni-n-qa}{know	-\lsc{ag}-\lsc{euph}-\lsc{top}}\morglo{mana-yá}{no-\lsc{emph}}}%morpheme+gloss
\glotran{\pb{Those} of them who knew; not \pb{those} of them who didn’t know.}{}%eng+spa trans
{}{}%rec - time

% 5
\gloexe{Glo6:mikunchik}{}{amv}%
{Kanan\pb{qa} mikunchik munasanchik[ta] qullqi kaptin\pb{qa}.}%amv que first line
{\morglo{kanan-qa}{now-\lsc{top}}\morglo{miku-nchik}{eat-\lsc{1pl}}\morglo{muna-sa-nchik[-ta]}{want-\lsc{prf}-\lsc{1}-\lsc{acc}}\morglo{qullqi}{money}\morglo{ka-pti-n-qa}{be-\lsc{subds}-\lsc{3}-\lsc{top}}}%morpheme+gloss
\glotran{\pb{Now} we eat whatever we want when there’s money.}{}%eng+spa trans
{}{}%rec - time

% 6
\gloexe{Glo6:Llaqtaykipa}{}{amv}%
{Llaqtaykipa\pb{qa} ¿tarpunkichu sibadata?}%amv que first line
{\morglo{llaqta-yki-pa-qa}{town-\lsc{2}-\lsc{loc}-\lsc{top}}\morglo{tarpu-nki-chu}{plant-\lsc{2}-\lsc{q}}\morglo{sibada-ta}{barley-\lsc{acc}}}%morpheme+gloss
\glotran{In \pb{your town}, do you plant barley?}{}%eng+spa trans
{}{}%rec - time

% 7
\gloexe{Glo6:puriq}{}{amv}%
{Uray\pb{qa} puriq kani trakillawan trakinchikpis nananankama.}%amv que first line
{\morglo{uray-qa}{down.hill-\lsc{top}}\morglo{puri-q}{walk-\lsc{ag}}\morglo{ka-ni}{be-\lsc{1}}\morglo{traki-lla-wan}{foot-\lsc{rstr}-\lsc{instr}}\morglo{traki-nchik-pis}{foot-\lsc{1pl}-\lsc{add}}\morglo{nana-na-n-kama}{hurt-\lsc{nmlz-}\lsc{3}-\lsc{lim}}}%morpheme+gloss
\glotran{I would walk \pb{down hill} just on foot until our feet hurt.}{}%eng+spa trans
{}{}%rec - time

% 8
\gloexe{Glo6:Difindiwanchik}{}{amv}%
{Difindiwanchik malichukunapaq\pb{qa}.}%amv que first line
{\morglo{difindi-wa-nchik}{defend-\lsc{1.obj}-\lsc{1pl}}\morglo{malichu-kuna-paq-qa}{curse-\lsc{pl}-\lsc{abl}-\lsc{top}}}%morpheme+gloss
\glotran{It protects us against \pb{curses}.}{}%eng+spa trans
{}{}%rec - time

% 9
\gloexe{Glo6:pasiyuman}{}{ch}%
{Lluqsila pasiyuman yaykushpa\pb{qa} manaña puydila\uo{}chu piru.}%ch que first line
{\morglo{lluqsi-la}{go.out-\lsc{pst}}\morglo{pasiyu-man}{walk-\lsc{all}}\morglo{yayku-shpa-qa}{enter-\lsc{subis}-\lsc{top}}\morglo{mana-ña}{no-\lsc{disc}}\morglo{puydi-la-chu}{be.able-\lsc{pst}-\lsc{neg}}\morglo{piru}{but}}%morpheme+gloss
\glotran{They went out for a walk but \pb{when they went in}, they couldn’t.}{}%eng+spa trans
{}{}%rec - time

% 10
\gloexe{Glo6:Qipiruptinqa}{}{sp}%
{Qipiruptinqa~\dots{} chay kundur\pb{qa} qipiptin huk turuta pagaykun.}%sp que first line
{\morglo{qipi-ru-pti-n-qa}{carry-\lsc{urgt}-\lsc{subds}-\lsc{3}-\lsc{top}}\morglo{chay}{\lsc{dem.d}}\morglo{kundur-qa}{condor-\lsc{top}}\morglo{qipi-pti-n}{carry-\lsc{subds}-\lsc{3}}\morglo{huk}{one}\morglo{turu-ta}{bull-\lsc{acc}}\morglo{paga-yku-n}{pay-\lsc{excep}-\lsc{3}}}%morpheme+gloss
\glotran{\pb{When he carried her}, after the condor carried her, she payed him a bull.}{}%eng+spa trans
{}{}%rec - time

\subsection{Continuative \phono{-Raq}}
\phono{-Raq}\index[sub]{continuitive} --~realized in \CH{} as \phono{-laq}~(\ref{Glo6:Kichwallaktam}) and in all other dialects as \phono{-raq}~-- indicates continuity of action, state or quality.\\

% 1
\gloexe{Glo6:Kichwallaktam}{}{ch}%
{Kichwallaktam limakuya: kaytraw\pb{laq} manam kastillanukta lima:chu.}%ch que first line
{\morglo{kichwa-lla-kta-m}{Quechua-\lsc{rstr}-\lsc{acc}-\lsc{evd}}\morglo{lima-ku-ya-:}{talk-\lsc{refl}-\lsc{prog}-\lsc{1}}\morglo{kay-traw-laq}{\lsc{dem.p}-\lsc{loc}-\lsc{cont}}\morglo{mana-m}{no-\lsc{evd}}\morglo{kastillanu-kta}{Spanish-\lsc{acc}}\morglo{lima-:-chu}{talk-\lsc{1}-\lsc{neg}}}%morpheme+gloss
\glotran{I’m just talking Quechua. Here, \pb{still}, we don’t speak Spanish.}{}%eng+spa trans
{}{}%rec - time

\noindent
It can generally be translated ‘still’~(\ref{Glo6:Qamqa}--\ref{Glo6:Kamanpi}) or, negated, ‘yet’~(\ref{Glo6:Runtuwanmi}), (\ref{Glo6:mayqinniypis}).\\

% 2
\gloexe{Glo6:Qamqa}{}{ach}%
{Qamqa flaku\pb{raq}mi. Hawlapam qamtaqa wirayachisayki.}%ach que first line
{\morglo{qam-qa}{you-\lsc{top}}\morglo{flaku-raq-mi}{skinny-\lsc{cont}-\lsc{evd}}\morglo{hawla-pa-m}{cage-\lsc{loc}-\lsc{evd}}\morglo{qam-ta-qa}{you-\lsc{acc}-\lsc{top}}\morglo{wira-ya-chi-sayki}{fat-\lsc{inch}-\lsc{caus}-\lsc{1>2.fut}}}%morpheme+gloss
\glotran{You’re \pb{still} skinny. I’m going to fatten you up in a cage.}{}%eng+spa trans
{}{}%rec - time

% 3
\gloexe{Glo6:Taqsana}{}{amv}%
{Taqsana\pb{raq}tri. Millwata taqsashun.}%amv que first line
{\morglo{taqsa-na-raq-tri}{wash-\lsc{nmlz}-\lsc{cont}-\lsc{evc}}\morglo{millwa-ta}{wool-\lsc{acc}}\morglo{taqsa-shun}{wash-\lsc{1pl.fut}}}%morpheme+gloss
\glotran{It has to be cleaned \pb{still}. We have to clean the wool.}{}%eng+spa trans
{}{}%rec - time

% 4
\gloexe{Glo6:Kamanpi}{}{lt}%
{Kamanpi puñukuyaptin\pb{raq} tarirun.}%lt que first line
{\morglo{kama-n-pi}{bed-\lsc{3}-\lsc{loc}}\morglo{puñu-ku-ya-pti-n-raq}{sleep-\lsc{refl}-\lsc{prog}-\lsc{subds}-\lsc{3}-\lsc{cont}}\morglo{tari-ru-n}{find-\lsc{urgt}-\lsc{3}}}%morpheme+gloss
\glotran{He found him when he was sleeping \pb{still} in his bed.}{}%eng+spa trans
{}{}%rec - time

% 5
\gloexe{Glo6:Runtuwanmi}{}{amv}%
{Runtuwanmi qaquyanmi chaypa \pb{mana}\pb{raq}mi shakashwan.}%amv que first line
{\morglo{runtu-wan-mi}{egg-\lsc{instr}-\lsc{evd}}\morglo{qaqu-ya-n-mi}{massage-\lsc{prog}-\lsc{3}-\lsc{evd}}\morglo{chay-pa}{\lsc{dem.d}-\lsc{loc}}\morglo{mana-raq-mi}{no-\lsc{cont}-\lsc{evd}}\morglo{shakash-wan}{guinea.pig-\lsc{instr}}}%morpheme+gloss
\glotran{He’s massaging with an egg -- \pb{not yet} with the guinea pig.}{}%eng+spa trans
{}{}%rec - time

% 6
\gloexe{Glo6:mayqinniypis}{}{amv}%
{\pb{Mana}m mayqinniypis wañuni\pb{raq}chu.}%amv que first line
{\morglo{mana-m}{no-\lsc{evd}}\morglo{mayqin-ni-y-pis}{which-\lsc{euph}-\lsc{1}-\lsc{add}}\morglo{wañu-ni-raq-chu}{die-\lsc{1}-\lsc{cont}-\lsc{neg}}}%morpheme+gloss
\glotran{\pb{None} of us has died yet.}{}%eng+spa trans
{}{}%rec - time

\noindent
Marking rhetorical questions, it can indicate a kind of despair~(\ref{Glo6:Yawarnintachu}), (\ref{Glo6:gringukunaqa}).\\

% 7
\gloexe{Glo6:Yawarnintachu}{}{ach}%
{¿Yawarnintachu? ¿Imata\pb{raq} hurqura chay dimunyukuna?}%ach que first line
{\morglo{yawar-ni-n-ta-chu}{blood-\lsc{euph}-\lsc{3}-\lsc{acc}-\lsc{q}}\morglo{ima-ta-raq}{what-\lsc{acc}-\lsc{cont}}\morglo{hurqu-ra}{take.out-\lsc{pst}}\morglo{chay}{\lsc{dem.d}}\morglo{dimunyu-kuna}{Devil-\lsc{pl}}}%morpheme+gloss
\glotran{His blood? \pb{What in the world} did the devil suck out of him?}{}%eng+spa trans
{}{}%rec - time

% 8
\gloexe{Glo6:gringukunaqa}{}{ach}%
{Chay gringukunaqa altukunatash rin. ¿Imayna\pb{raq} chay runata wañuchin?}%ach que first line
{\morglo{chay}{\lsc{dem.d}}\morglo{gringu-kuna-qa}{gringo-\lsc{pl}-\lsc{top}}\morglo{altu-kuna-ta-sh}{high-\lsc{pl}-\lsc{acc}-\lsc{evr}}\morglo{ri-n}{go-\lsc{3}}\morglo{imayna-raq}{how-\lsc{cont}}\morglo{chay}{\lsc{dem.d}}\morglo{runa-ta}{\lsc{person}-\lsc{acc}}\morglo{wañu-chi-n}{die-\lsc{caus}-\lsc{3}}}%morpheme+gloss
\glotran{The gringos go to the heights, they say. \pb{How on earth} could they kill those people?}{}%eng+spa trans
{}{}%rec - time

\noindent
With subordinate clauses, it may indicate a prerequisite or a necessary condition for the event to take place, translating in English as ‘first’ or ‘not until’~(\ref{Glo6:ruwashpa}).\\

% 9
\gloexe{Glo6:ruwashpa}{}{amv}%
{Kisuta ruwashpa\pb{raq} trayamuyan.}%amv que first line
{\morglo{kisu-ta}{cheese-\lsc{acc}}\morglo{ruwa-shpa-raq}{make-\lsc{subis}-\lsc{cont}}\morglo{traya-mu-ya-n}{arrive-\lsc{cisl}-\lsc{prog}-\lsc{3}}}%morpheme+gloss
\glotran{\pb{Once} she makes the cheese, she’s coming.}{}%eng+spa trans
{}{}%rec - time

\noindent
\phono{Chay-raq} indicates an imminent future, translating in Andean Spanish \spanish{recién}~(\ref{Glo6:tapayan}). Employed as a coordinator, it implies a contrast between the coordinated elements (see~§~\ref{sec:coord}).\\

% 10
\gloexe{Glo6:tapayan}{}{amv}%
{Chay\pb{raq}mi tapayan. Qallaykuyani chay\pb{raq}.}%amv que first line
{\morglo{chay-raq-mi}{\lsc{dem.d}-\lsc{cont}-\lsc{evd}}\morglo{tapa-ya-n}{cover-\lsc{prog}-\lsc{3}}\morglo{qalla-yku-ya-ni}{begin-\lsc{excep}-\lsc{prog}-\lsc{1}}\morglo{chay-raq}{\lsc{dem.d}-\lsc{cont}}}%morpheme+gloss
\glotran{He’s \pb{just now going to} cap it. I’m \pb{just now} going to start.}{}%eng+spa trans
{}{}%rec - time

\subsection{Sequential \phono{-taq}}
\phono{-taq}\index[sub]{sequential} indicates the sequence of events~(\ref{Glo6:Tardiqa}).\\

% 1
\gloexe{Glo6:Tardiqa}{}{amv}%
{Tardiqa yapa listu suyan; yapa\pb{taq}shi trayarun.}%amv que first line
{\morglo{tardi-qa}{afternoon-\lsc{top}}\morglo{yapa}{again}\morglo{listu}{ready}\morglo{suya-n}{wait-\lsc{3}}\morglo{yapa-taq-shi}{again-\lsc{seq}-\lsc{evr}}\morglo{traya-ru-n}{arrive-\lsc{urgt}-\lsc{3}}}%morpheme+gloss
\glotran{In the afternoon, \pb{again}, ready, he waits. \pb{Then, again}, [the zombie] arrived.}{}%eng+spa trans
{}{}%rec - time

\noindent
Adelaar~(p.c.)\index[aut]{Adelaar, Willem F. H.} points out that in Ayacucho Quechua \phono{-ña-taq} is a fixed combination. It appears that may be the case here too~(\ref{Glo6:pikarushpa}--\ref{Glo6:makiywan}). In these examples \phono{-taq} seems to continue to indicate a sequence of events.\\

% 2
\gloexe{Glo6:pikarushpa}{}{amv}%
{Lliwta pikarushpa, kaymanñataq quturini trurani wakmanña\pb{taq}.}%amv que first line
{\morglo{lliw-ta}{all-\lsc{acc}}\morglo{pika-ru-shpa}{pick-\lsc{urgt}-\lsc{subds}}\morglo{kay-man-ña-taq}{\lsc{dem.d}-\lsc{all}-\lsc{disc}-\lsc{seq}}\morglo{qutu-ri-ni}{gather-\lsc{incep}-\lsc{1}}\morglo{trura-ni}{put-\lsc{1}}\morglo{wak-man-ña-taq}{\lsc{dem.p}-\lsc{all}-\lsc{disc}-\lsc{seq}}}%morpheme+gloss
\glotran{When I have all these sorted, \pb{then} I gather everything here and \pb{then} store it there.}{}%eng+spa trans
{}{}%rec - time

% 3
\gloexe{Glo6:takllawanmi}{}{ch}%
{Qaliqa takllawanmi halun. Qipantaña\pb{taq} kulpakta maqanchik pikuwan.}%ch que first line
{\morglo{qali-qa}{man-\lsc{top}}\morglo{taklla-wan-mi}{plow-\lsc{instr}-\lsc{evd}}\morglo{halu-n}{turn.earth-\lsc{3}}\morglo{qipa-n-ta-ña-taq}{behind-\lsc{3}-\lsc{acc}-\lsc{disc}-\lsc{seq}}\morglo{kulpa-kta}{clod-\lsc{acc}}\morglo{maqa-nchik}{hit-\lsc{1pl}}\morglo{piku-wan}{pick-\lsc{instr}}}%morpheme+gloss
\glotran{Men turn over the earth with a foot plow. Behind them, \pb{then}, we break up the clods with a pick.}{}%eng+spa trans
{}{}%rec - time

% 4
\gloexe{Glo6:makiywan}{}{amv}%
{Ñuqapa makiywan aytrichiyanmi. Kanan trakillaña\pb{taq}. Huknin makiwanña\pb{taq} kananmi.}%amv que first line
{\morglo{ñuqa-pa}{I-\lsc{gen}}\morglo{maki-y-wan}{hand-\lsc{1}-\lsc{instr}}\morglo{aytri-chi-ya-n-mi}{stir-\lsc{caus}-\lsc{prog}-\lsc{3}-\lsc{evd}}\morglo{kanan}{now}\morglo{traki-lla-ña-taq}{foot-\lsc{rstr}-\lsc{disc}-\lsc{seq}}\morglo{huk-ni-n}{one-\lsc{euph}-\lsc{3}}\morglo{maki-wan-ña-taq}{hand-\lsc{instr}-\lsc{disc}-\lsc{seq}}\morglo{kanan-mi}{now-\lsc{evd}}}%morpheme+gloss
\glotran{He’s stirring it with my hand. Now, the foot. Now with the other hand.}{}%eng+spa trans
{}{}%rec - time

\noindent
In a question introduced by an interrogative (\phono{pi-}, \phono{ima-}~\dots) \phono{-taq} attaches to the interrogative in case it is the only word in the phrase or, in case the phrase includes two or more words, to the final word in the phrase (\ref{Glo6:Ishpaykuruwan}--\ref{Glo6:Imanashaq}).\\

% 5
\gloexe{Glo6:Ishpaykuruwan}{}{amv}%
{¡Ishpaykuruwan! ¿Imapaq\pb{taq} ishpan?}%amv que first line
{\morglo{ishpa-yku-ru-wa-n}{urinate-\lsc{excep}-\lsc{urgt}-\lsc{1.obj}-\lsc{3}}\morglo{ima-paq-taq}{what-\lsc{purp}-\lsc{seq}}\morglo{ishpa-n}{urinate-\lsc{3}}}%morpheme+gloss
\glotran{It urinated on me! \pb{Why} does it urinate?}{}%eng+spa trans
{}{}%rec - time

% 6
\gloexe{Glo6:rikuq}{}{amv}%
{¿Ima rikuq\pb{taq} karqa sapatillayki?}%amv que first line
{\morglo{ima}{what}\morglo{rikuq-taq}{color-\lsc{seq}}\morglo{ka-rqa}{be-\lsc{pst}}\morglo{sapatilla-yki}{shoe-\lsc{2}}}%morpheme+gloss
\glotran{\pb{What color} were your shoes?}{}%eng+spa trans
{}{}%rec - time

% 7
\gloexe{Glo6:Imanashaq}{}{lt}%
{¿Imanashaq\pb{taq}? Diosllatañatriki.}%lt que first line
{\morglo{ima-na-shaq-taq}{what-\lsc{vrbz}-\lsc{1.fut}-\lsc{seq}}\morglo{Dios-lla-ta-ña-tr-iki}{God-\lsc{rstr}-\lsc{acc}-\lsc{disc}-\lsc{evc}-\lsc{iki}}}%morpheme+gloss
\glotran{\pb{What am I going to do}? It’s for God already.}{}%eng+spa trans
{}{}%rec - time

\noindent
In this capacity, \phono{-taq} may be the most transparent of the enclitics attaching to \phono{q}-phrases. In a clause with a conditional or in a subordinate clause, \phono{-taq} can indicate a warning~(\ref{Glo6:Kurasunniyman}).\\

% 8
\gloexe{Glo6:Kurasunniyman}{}{amv}%
{Kurasunniyman shakashta trurayan. Ñuqa niyani “¡Kaniruwaptinña\pb{taq}!”}%amv que first line
{\morglo{kurasun-ni-y-man}{heart-\lsc{euph}-\lsc{1}-\lsc{all}}\morglo{shakash-ta}{guinea.pig-\lsc{acc}}\morglo{trura-ya-n}{put-\lsc{prog}-\lsc{3}}\morglo{ñuqa}{I}\morglo{ni-ya-ni}{say-\lsc{prog}-\lsc{1}}\morglo{kani-ru-wa-pti-n-ña-taq}{bite-\lsc{urgt}-\lsc{1.obj}-\lsc{subds}-\lsc{3}-\lsc{disc}-\lsc{seq}}}%morpheme+gloss
\glotran{He’s putting the guinea pig over my heart. I’m saying, “\pb{Be careful} it doesn’t bite me!”}{}%eng+spa trans
{}{}%rec - time

\noindent
\phono{-taq} also functions as a conjunction~(\ref{Glo6:puchkawan}) (see~§~\ref{sec:coord}).\\

% 9
\gloexe{Glo6:puchkawan}{}{amv}%
{Warmiña\pb{taq} puchkawan qariña\pb{taq} tihiduwan.}%amv que first line
{\morglo{warmi-ña-taq}{women-\lsc{disc}-\lsc{seq}}\morglo{puchka-wan}{spinning-\lsc{instr}}\morglo{qari-ña-taq}{man-\lsc{disc}-\lsc{seq}}\morglo{tihidu-wan}{weaving-\lsc{instr}}}%morpheme+gloss
\glotran{Women with spinning \pb{and} men with weaving.}{}%eng+spa trans
{}{}%rec - time

\subsection{Emotive \phono{-ya}}\label{ssec:emotive}
\phono{-ya} indicates regret or resignation\index[sub]{emotive}. It can be translated ‘alas’ or ‘regretfully’ or with a sigh. Not very widely employed.\\

% 1
\gloexe{Glo6:Hinashpaqa}{}{amv}%
{Hinashpaqa\pb{ya}, “Wañurachishaqña wakchachaytaqa dimasllam sufriyan.”}%amv que first line
{\morglo{hinashpa-qa-ya}{then-\lsc{top}-\lsc{emo}}\morglo{wañu-ra-chi-shaq-ña}{die-\lsc{urgt}-\lsc{caus}-\lsc{1.fut}-\lsc{disc}}\morglo{wakcha-cha-y-ta-qa}{lamb-\lsc{dim}-\lsc{1}-\lsc{acc}-\lsc{top}}\morglo{dimas-lla-m}{too.much-\lsc{rstr}-\lsc{evd}}\morglo{sufri-ya-n}{suffer-\lsc{prog}-\lsc{3}}}%morpheme+gloss
\glotran{Then, \pb{alas}, “I’m going to kill my little lamb already -- he’s suffering too much,” [I said].}{}%eng+spa trans
{}{}%rec - time

% 2
\gloexe{Glo6:runakunaqa}{}{amv}%
{Unay runakunaqa yatrayan masta, masta\pb{ya}, lliwta~\dots{} aaaa.}%amv que first line
{\morglo{unay}{before}\morglo{runa-kuna-qa}{person-\lsc{pl}-\lsc{top}}\morglo{yatra-ya-n}{know-\lsc{prog}-\lsc{3}}\morglo{mas-ta}{more-\lsc{acc}}\morglo{mas-ta-ya}{more-\lsc{acc}-\lsc{emo}}\morglo{lliw-ta}{all-\lsc{acc}}\morglo{aaaa}{ahhh}}%morpheme+gloss
\glotran{In the old days, people knew more, more, everything, \pb{ahhh}.}{}%eng+spa trans
{}{}%rec - time

\subsection{Evidence}\label{ssec:evidence}
Evidentials\index[sub]{evidentials} indicate the type of the speaker’s source of information. \SYQ, like most\footnote{Note, though, that Huallaga Q counts four evidentials, (\phono{-mi}, \phono{-shi}, \phono{-chi}, snd \phono{-chaq}) (Weber 1989:76). South Conchucos Q counts six, (\phono{-mi}, \phono{-shi}, \phono{-chi}, \phono{-cha:}, and \phono{-cher}); Sihuas, too, counts six (Hintz and Hintz 2014).} other Quechuan languages, counts three evidential suffixes: direct \phono{-mi}~(\ref{Glo6:Taytacha}--\ref{Glo6:puntraw}), reportative \phono{-shi}~(\ref{Glo6:Radyukunapa}--\ref{Glo6:Qarinta}), and conjectural \phono{-tri}~(\ref{Glo6:trayarachiptiki}--\ref{Glo6:Wasiy}) (\ie~the speaker has her own evidence for P (generally visual); the speaker learned P from someone else; or the speaker infers P based on some other evidence). Following a short vowel, these are realized as \phono{-m}, \phono{sh}, and \phono{-tr}, respectively~(\ref{Glo6:puntraw}), (\ref{Glo6:Qarinta}), (\ref{Glo6:Wasiy}).\\

% 1
\gloexe{Glo6:Taytacha}{}{amv}%
{Taytacha José irransakurqa chaypa\pb{m}.}%amv que first line
{\morglo{tayta-cha}{father-\lsc{dim}}\morglo{José}{José}\morglo{irransa-ku-rqa}{herranza-\lsc{refl}-\lsc{pst}}\morglo{chay-pa-m}{\lsc{dem.d}-\lsc{loc}-\lsc{evd}}}%morpheme+gloss
\glotran{My grandfather José held herranzas \pb{there}.}{}%eng+spa trans
{}{}%rec - time

% 2
\gloexe{Glo6:Trurawarqaya}{}{amv}%
{Trurawarqaya huk ratu. Manayá puchukachiwarqachu. Trurawarqa\pb{m}.}%amv que first line
{\morglo{trura-wa-rqa-yá}{put-\lsc{1.obj}-\lsc{pst}-\lsc{emph}}\morglo{huk}{one}\morglo{ratu}{moment}\morglo{mana-yá}{no-\lsc{emph}}\morglo{puchuka-chi-wa-rqa-chu}{finish-\lsc{caus}-\lsc{1.obj}-\lsc{pst}-\lsc{neg}}\morglo{trura-wa-rqa-m}{put-\lsc{1.obj}-\lsc{pst}-\lsc{evd}}}%morpheme+gloss
\glotran{They put me in [school] a short while. They didn’t have me finish, but they did \pb{put me in}.}{}%eng+spa trans
{}{}%rec - time

% 3
\gloexe{Glo6:puntraw}{}{ach}%
{Qayna puntraw qanin puntrawlla\pb{m} trayamura:.}%ach que first line
{\morglo{qayna}{previous}\morglo{puntraw}{day}\morglo{qanin}{day.before.yesterday}\morglo{puntraw-lla-m}{day-\lsc{rstr}-\lsc{evd}}\morglo{traya-mu-ra-:}{arrive-\lsc{cisl}-\lsc{pst}-\lsc{1}}}%morpheme+gloss
\glotran{I arrived yesterday, \pb{just the day} before yesterday.}{}%eng+spa trans
{}{}%rec - time

% 4
\gloexe{Glo6:Radyukunapa}{}{sp}%
{Radyukunapa rimayta rimayan. Lluqsiyamun\pb{shi} tirrurista. Tirrurista rikariyamun\pb{shi}.}%sp que first line
{\morglo{radyu-kuna-pa}{radio-\lsc{pl}-\lsc{loc}}\morglo{rima-y-ta}{talk-\lsc{inf}-\lsc{acc}}\morglo{rima-ya-n}{talk-\lsc{prog}-\lsc{3}}\morglo{lluqsi-ya-mu-n-shi}{go.out-\lsc{prog}-\lsc{cisl}-\lsc{3}-\lsc{evr}}\morglo{tirrurista}{terrorist}\morglo{tirrurista}{terrorist}\morglo{rikari-ya-mu-n-shi}{appear-\lsc{prog}-\lsc{cisl}-\lsc{3}-\lsc{evr}}}%morpheme+gloss
\glotran{On the radio they talk for the sake of talking. Terrorists \pb{are coming out, they say}. Terrorists \pb{are appearing, they say}.}{}%eng+spa trans
{}{}%rec - time

% 5
\gloexe{Glo6:uchukllapa}{}{amv}%
{Chay uchukllapa pashñataq uywakuptinñataq\pb{shi} maqtaqa aparqa mikunanta.}%amv que first line
{\morglo{chay}{\lsc{dem.d}}\morglo{uchuk-lla-pa}{small-\lsc{rstr}-\lsc{loc}}\morglo{pashña-taq}{girl-\lsc{acc}}\morglo{uywa-ku-pti-n-ña-taq-shi}{raise-\lsc{refl}-\lsc{subds}-\lsc{3}-\lsc{disc}-\lsc{seq}-\lsc{evr}}\morglo{maqta-qa}{young.man-\lsc{top}}\morglo{apa-rqa}{bring-\lsc{pst}}\morglo{miku-na-n-ta}{eat-\lsc{nmlz}-\lsc{3}-\lsc{acc}}}%morpheme+gloss
\glotran{When \pb{he raised} the girl in that cave, the man brought her his food, \pb{they say}.}{}%eng+spa trans
{}{}%rec - time

% 6
\gloexe{Glo6:Qarinta}{}{amv}%
{Qarinta\pb{sh} wañurachin mashanta\pb{sh} wañurachin.}%amv que first line
{\morglo{qari-n-ta-sh}{man-\lsc{3}-\lsc{acc}-\lsc{evr}}\morglo{wañu-ra-chi-n}{die-\lsc{urgt}-\lsc{caus}-\lsc{3}}\morglo{masha-n-ta-sh}{son.in.law-\lsc{3}-\lsc{acc}-\lsc{evr}}\morglo{wañu-ra-chi-n}{die-\lsc{urgt}-\lsc{caus}-\lsc{3}}}%morpheme+gloss
\glotran{She killed her \pb{husband, they say}; she killed her \pb{son-in-law, they say}.}{}%eng+spa trans
{}{}%rec - time

% 7
\gloexe{Glo6:trayarachiptiki}{}{amv}%
{Qiñwalman trayarachiptiki wañukunman\pb{tri}.}%amv que first line
{\morglo{qiñwal-man}{quingual.grove-\lsc{all}}\morglo{traya-ra-chi-pti-ki}{arrive-\lsc{urgt}-\lsc{caus}-\lsc{subds}-\lsc{2}}\morglo{wañu-ku-n-man-tri}{die-\lsc{refl}-\lsc{3}-\lsc{cond}-\lsc{evc}}}%morpheme+gloss
\glotran{If you make her go all the way to the quingual grove, she might die.}{}%eng+spa trans
{}{}%rec - time

% 8
\gloexe{Glo6:Suwawan}{}{lt}%
{Suwawan\pb{tri}. Durasnuy kara mansanay kara qanin puntraw.}%lt que first line
{\morglo{suwa-wa-n-tri}{rob-\lsc{1.obj}-\lsc{3}-\lsc{evr}}\morglo{durasnu-y}{peach-\lsc{1}}\morglo{ka-ra}{be-\lsc{pst}}\morglo{mansana-y}{apple-\lsc{1}}\morglo{ka-ra}{be-\lsc{pst}}\morglo{qanin}{previous}\morglo{puntraw}{day}}%morpheme+gloss
\glotran{They \pb{may have robbed} me. The day before yesterday I had peaches and apples.}{}%eng+spa trans
{}{}%rec - time

% 9
\gloexe{Glo6:Wasiy}{}{amv}%
{Wasiy rahasa kayan. Saqaykurunqa\pb{tr}.}%amv que first line
{\morglo{wasi-y}{house-\lsc{1}}\morglo{raha-sa}{crack-\lsc{prf}}\morglo{ka-ya-n}{be-\lsc{prog}-\lsc{3}}\morglo{saqa-yku-ru-nqa-tr}{go.down-\lsc{excep}-\lsc{urgt}-\lsc{3.fut}-\lsc{evc}}}%morpheme+gloss
\glotran{My house is cracked. \pb{It’s going to fall down}.}{}%eng+spa trans
{}{}%rec - time

The evidential system of \SYQ{} is unusual among Quechuan languages, however, in that it overlays the three-way distinction standard to Quechua with a second three-way distinction. The set of evidentials in \SYQ{} thus counts nine members: \phono{-mI}, \phono{-m-ik}, and \phono{-m-iki}; \phono{-shI}, \phono{-sh-ik}, and \phono{-sh-iki}; and \phono{-trI}, \phono{-tr-ik}, and \phono{-tr-iki}. The \phono{-I}, \phono{-ik}, and \phono{-iki} forms are not allomorphs: they receive different interpretations, generally indicating increasing degrees of evidence strength or, in the case of modalized verbs, increasing modal force. §~\ref{ssec:evidence} describes this system in some detail. For further formal analysis, see \citet{Shimelman12}.\index[aut]{Shimelman, Aviva}

In addition to indicating the speaker’s information type, evidentials also function to indicate focus or comment and to complete copular predicates (for further discussion and examples, see §~\ref{sec:emphasis} and~\ref{sec:equative} on emphasis and equatives).

Evidentials are subject to the following distributional restrictions. They never attach to the topic or subject; these are, rather, marked with \phono{-qa}. In content questions, the evidential attaches to the question word or to the last word of the questioned phrase~(\ref{Glo6:chay}) (see~§~\ref{sec:interr} on interrogation).\\

% 10
\gloexe{Glo6:chay}{}{amv}%
{¿May\pb{mi} chay warmi?}%amv que first line
{\morglo{may-mi}{where-\lsc{evd}}\morglo{chay}{\lsc{dem.d}}\morglo{warmi}{woman}}%morpheme+gloss
\glotran{\pb{Where} is that woman?}{}%eng+spa trans
{}{}%rec - time

\noindent
Evidentials do not appear in commands or injunctions~(\ref{Glo6:Ruwaruchun}); finally, only one evidential may occur per clause~(\ref{Glo6:Vakay}).\\

% 11
\gloexe{Glo6:Ruwaruchun}{}{amv}%
{¡Ruwaruchun*mi/shi/tri!}% que first line
{\morglo{ruwa-ru-chun-*mi/shi/tri}{make-\lsc{urgt}-\lsc{injunc}-\lsc{evd}-\lsc{evr}-\lsc{evc}}}%morpheme+gloss
\glotran{\pb{Let} him do it!}{}%eng+spa trans
{}{}%rec - time

% 12
\gloexe{Glo6:Vakay}{}{amv}%
{¡Vakay wira wira\pb{m}, matraypi puñushpa, allin pastuta mikushpa\pb{m}.}% que first line
{\morglo{vaka-y}{cow-\lsc{1}}\morglo{wira}{fat}\morglo{wira-m}{fat-\lsc{evd}}\morglo{matray-pi}{cave-\lsc{loc}}\morglo{puñu-shpa}{sleep-\lsc{subis}}\morglo{allin}{good}\morglo{pastu-ta}{pasture.grass-\lsc{acc}}\morglo{miku-shpa-m}{eat-\lsc{refl}-\lsc{evd}}}%morpheme+gloss
\glotran{My cow is really fat, sleeping in a cave and eating good pasture grass.}{}%eng+spa trans
{}{}%rec - time

All three evidentials are interpreted as assertions. The first, \phono{-mI}, is generally left untranslated in Spanish; the second, \phono{-shI}, is often rendered \phono{dice} ‘they say’; the third is reflected in a change in verb tense or mode (see~§~\ref{ssec:conjectural}). The difference between the three is a matter, first, of whether or not evidence is from personal experience, and, second, whether that evidence supports the proposition, \phono{p}, immediately under the scope of the evidential or another set of propositions, \phono{P’}, that are evidence for \phono{p}, as represented in Table \ref{Tab31}.

% Table 31
\begin{table}
\small\centering
\caption{Evidential schema: “evidence from” by “evidence for”}\label{Tab31}
\begin{tabular}{lll}
\lsptoprule
	& Supports scope 			& Supports \phono{P’}		\\
	& proposition \phono{p} 	& evidence for \phono{p}		\\
\midrule
Direct 	&\lsc{direct} &\lsc{conjectural}\\
(personal experience) evidence 			& \phono{-mI} & \phono{-trI}		\\[1ex]
Reportative 	&\lsc{reportative}&\lsc{conjectural}\\
(non-personal experience) evidence 	& \phono{-shI} & \phono{-trI}	\\
\lspbottomrule
\end{tabular}
\end{table}

So, employing \phono{-mI}(\phono{p}), the speaker asserts predicate \phono{p} and represents that she has personal-experience evidence for \phono{p}; employing \phono{-shI}(\phono{p}), the speaker asserts \phono{p} and refers the hearer to another source for evidence for \phono{p}; and employing \phono{-trI}(\phono{p}), the speaker asserts \phono{p} and represents that \phono{p} is a conjecture from \phono{P’}, propositions for which she has either \phono{-mI}-type or \phono{-shI}-type evidence or both. That is, although \SYQ{} counts three evidential suffixes, it counts only two evidence types, direct and reportative; these two are jointly exhaustive. §~\ref{ssec:direct}--\ref{ssec:conjectural} cover \phono{-mI}, \phono{-shI}, and \phono{trI}, in turn. §~\ref{ssec:evidmodifi} covers the evidential modifiers, \phono{-ari} and \phono{-ik/iki}.

\subsubsection{Direct \phono{-mI}}\label{ssec:direct}
\phono{-mI}\index[sub]{evidentials!direct} indicates that the speaker speaks from direct experience. Unlike \phono{-shI} and \phono{-trI}, it is generally left untranslated. Note that in the examples below, with the exception of~(\ref{Glo6:Vakaqa}), the speaker’s knowledge is \emph{not} the product of visual experience.\\

% 1 (3)
\gloexe{Glo6:Vakaqa}{}{amv}%
{Vakaqa kaypa waqrayuq\pb{mi}ki kayan.}%amv que first line
{\morglo{vaka-qa}{cow-\lsc{top}}\morglo{kay-pa}{\lsc{dem.p}-\lsc{loc}}\morglo{waqra-yuq-m-iki}{horn-\lsc{poss}-\lsc{evd}-\lsc{iki}}\morglo{ka-ya-n}{be-\lsc{prog}-\lsc{3}}}%morpheme+gloss
\glotran{The cows here \pb{have horns}.}{}%eng+spa trans
{}{}%rec - time

% 2 (1)
\gloexe{Glo6:pakarayan}{}{amv}%
{Piñiy\pb{mi} pakarayan wasiypa wak ichuypa ukunpa.}%amv que first line
{\morglo{piñi-y-mi}{necklace-\lsc{1}-\lsc{evd}}\morglo{paka-ra-ya-n}{hide-\lsc{unint}-\lsc{intens}-\lsc{3}}\morglo{wasi-y-pa}{house-\lsc{1}-\lsc{loc}}\morglo{wak}{\lsc{dem.d}}\morglo{ichuy-pa}{straw-\lsc{gen}}\morglo{uku-n-pa}{inside-\lsc{3}-\lsc{loc}}}%morpheme+gloss
\glotran{\pb{My necklace} is hidden in my house under the straw.}{}%eng+spa trans
{}{}%rec - time

% 3 (2)
\gloexe{Glo6:Chaywan}{}{amv}%
{Chaywan\pb{mi} pwirtata ruwayani. Mana\pb{m} achkataq ruwanichu.}%amv que first line
{\morglo{chay-wan-mi}{\lsc{dem.d}-\lsc{instr}-\lsc{evd}}\morglo{pwirta-ta}{door-\lsc{acc}}\morglo{ruwa-ya-ni}{make-\lsc{prog}-\lsc{1}}\morglo{mana-m}{no-\lsc{evd}}\morglo{achka-taq}{a.lot-\lsc{acc}}\morglo{ruwa-ni-chu}{make.\lsc{1}-\lsc{neg}}}%morpheme+gloss
\glotran{I make doors with this. I don’t make a lot.}{}%eng+spa trans
{}{}%rec - time

% 4
\gloexe{Glo6:Karrupis}{}{ach}%
{Karrupis ashnakuyan\pb{mi}.}%ach que first line
{\morglo{karru-pis}{car-\lsc{add}}\morglo{ashna-ku-ya-n-mi}{smell-\lsc{refl}-\lsc{prog}-\lsc{3}-\lsc{evd}}}%morpheme+gloss
\glotran{The buses, too, \pb{stink}.}{}%eng+spa trans
{}{}%rec - time

% 5
\gloexe{Glo6:Qunirirachishunki}{}{ach}%
{Qunirirachishunki. Kaliyntamanchik\pb{mi}.}%ach que first line
{\morglo{quni-ri-ra-chi-shu-nki}{warm-\lsc{incep}-\lsc{caus}-\lsc{2.obj}-\lsc{2}}\morglo{kaliynta-ma-nchik-mi}{warm-\lsc{1.obj}-\lsc{1pl}-\lsc{evd}}}%morpheme+gloss
\glotran{It warms you up. \pb{It warms us up}.}{}%eng+spa trans
{}{}%rec - time

\subsubsection{Reportative \phono{-shI}}
\phono{-shI}\index[sub]{evidentials!reportative} indicates that the speaker’s evidence does not come from personal experience~(\ref{Glo6:Awkichanka}--\ref{Glo6:Lata}).\\

% 1
\gloexe{Glo6:Awkichanka}{}{amv}%
{Awkichanka urqupaqa inkantu\pb{sh} -- karru\pb{sh} chinkarurqa qutrapa.}%amv que first line
{\morglo{Awkichanka}{Awkichanka}\morglo{urqu-pa-qa}{hill-\lsc{loc}-\lsc{top}}\morglo{inkantu-sh}{spirit-\lsc{evr}}\morglo{karru-sh}{car-\lsc{evr}}\morglo{chinka-ru-rqa}{lose-\lsc{urgt}-\lsc{pst}}\morglo{qutra-pa}{lake-\lsc{loc}}}%morpheme+gloss
\glotran{In the hill Okichanka, there is \pb{a spirit, they say} -- a car was lost in a reservoir.}{}%eng+spa trans
{}{}%rec - time

% 2
\gloexe{Glo6:Mashwaqa}{}{ch}%
{Mashwaqa prustatapaq\pb{shi} allin.}%ch que first line
{\morglo{mashwa-qa}{mashua-\lsc{top}}\morglo{prustata-paq-shi}{prostate-\lsc{ben}-\lsc{evr}}\morglo{allin}{good}}%morpheme+gloss
\glotran{Mashua is good for the \pb{prostate}, \pb{they say}.}{}%eng+spa trans
{}{}%rec - time

% 3
\gloexe{Glo6:Chaypash}{}{amv}%
{Chaypa\pb{sh} runtuta mikuchishunki.}%amv que first line
{\morglo{chay-pa-sh}{\lsc{dem.d}-\lsc{loc}-\lsc{evr}}\morglo{runtu-ta}{egg-\lsc{acc}}\morglo{miku-chi-shu-nki}{eat-\lsc{caus}-\lsc{2.obj}-\lsc{2}}}%morpheme+gloss
\glotran{They’ll feed you eggs \pb{there}, \pb{they say}.}{}%eng+spa trans
{}{}%rec - time

% 4
\gloexe{Glo6:Lata}{}{ach}%
{Lata-wan yanu-shpa-taq-\pb{shi} runa-ta-pis miku-ru-ra.}%ach que first line
{\morglo{lata-wan}{can-\lsc{instr}}\morglo{yanu-shpa-taq-shi}{cook-\lsc{subis}-\lsc{seq}-\lsc{evr}}\morglo{runa-ta-pis}{person-\lsc{acc}-\lsc{add}}\morglo{miku-ru-ra}{eat-\lsc{urgt}-\lsc{pst}}}%morpheme+gloss
\glotran{They [the Shining Path] even \pb{cooked} people in metal pots and ate them, \pb{they say}.}{}%eng+spa trans
{}{}%rec - time

\noindent
It is used systematically in stories~(\ref{Glo6:Unay}), (\ref{Glo6:Chaypaqshi}).\\

% 5
\gloexe{Glo6:Unay}{}{sp}%
{Unay\pb{shi} kara huk asnu.}%sp que first line
{\morglo{unay-shi}{before-\lsc{evr}}\morglo{ka-ra}{be-\lsc{pst}}\morglo{huk}{one}\morglo{asnu}{donkey}}%morpheme+gloss
\glotran{\pb{Once upon a time, they say} there was a mule.}{}%eng+spa trans
{}{}%rec - time

% 6
\gloexe{Glo6:Chaypaqshi}{}{lt}%
{Chaypaq\pb{shi} kutirun maman kaqta papanin kaqta.}%lt que first line
{\morglo{chay-paq-shi}{\lsc{dem.d}-\lsc{abl}-\lsc{evr}}\morglo{kuti-ru-n}{return-\lsc{urgt}-\lsc{3}}\morglo{mama-n}{mother-\lsc{3}}\morglo{ka-q-ta}{be-\lsc{ag}-\lsc{acc}}\morglo{papa-ni-n}{father-\lsc{euph}-\lsc{3}}\morglo{ka-q-ta}{be-\lsc{ag}-\lsc{acc}}}%morpheme+gloss
\glotran{He returned \pb{from there, they say}, to his mother’s place, to his father’s place.}{}%eng+spa trans
{}{}%rec - time

\subsubsection{Conjectural \phono{-trI}}\label{ssec:conjectural}
\phono{-trI}\index[sub]{evidentials!conjectural} indicates that the speaker does not have evidence for the proposition directly under the scope of the evidential, but is, rather, conjecturing to that proposition from others for which she does have evidence~(\ref{Glo6:Awayan}--\ref{Glo6:hamuyan}).\\

% 1
\gloexe{Glo6:Awayan}{}{amv}%
{Awayan\pb{tr}iki kamata.}%amv que first line
{\morglo{awa-ya-n-tr-iki}{weave-\lsc{prog}-\lsc{evr}-\lsc{iki}}\morglo{kama-ta}{blanket-\lsc{acc}}}%morpheme+gloss
\glotran{\pb{He must be weaving} a blanket.}{}%eng+spa trans
{}{}%rec - time

% 2
\gloexe{Glo6:kayachuwan}{}{amv}%
{Wañuypaqpis kayachuwan\pb{tr}iki.}%amv que first line
{\morglo{wañu-y-paq-pis}{die-\lsc{inf}-\lsc{abl}-\lsc{add}}\morglo{ka-ya-chuwan-tr-iki}{be-\lsc{prog}-\lsc{1pl.cond}-\lsc{evc}-\lsc{iki}}}%morpheme+gloss
\glotran{\pb{We could be} also about to die.}{}%eng+spa trans
{}{}%rec - time

% 3
\gloexe{Glo6:Kukachankunata}{}{amv}%
{Kukachankunata aparuptiyqa tiyaparuwanqa\pb{tr}ik.}%amv que first line
{\morglo{kuka-cha-n-kuna-ta}{coca-\lsc{dim}-\lsc{3}-\lsc{pl}-\lsc{acc}}\morglo{apa-ru-pti-y-qa}{bring-\lsc{urgt}-\lsc{subds}-\lsc{1}-\lsc{top}}\morglo{tiya-pa-ru-wa-nqa-tr-ik}{sit-\lsc{ben}-\lsc{urgt}-\lsc{1.obj}-\lsc{evc}-\lsc{ik}}}%morpheme+gloss
\glotran{If I bring them their coca, \pb{they’ll accompany me sitting}.}{}%eng+spa trans
{}{}%rec - time

% 4
\gloexe{Glo6:Chayman}{}{ach}%
{Chayman\pb{tr}ik ayarikura.}%ach que first line
{\morglo{chay-man-tr-ik}{\lsc{dem.d}-\lsc{all}-\lsc{evc}-\lsc{ik}}\morglo{aya-ri-ku-ra}{cadaver-\lsc{incep}-\lsc{refl}-\lsc{pst}}}%morpheme+gloss
\glotran{She \pb{must} have become a cadaver.}{}%eng+spa trans
{}{}%rec - time

% 5
\gloexe{Glo6:Upyachinman}{}{ch}%
{Upyachinman\pb{tri}.}%ch que first line
{\morglo{upya-chi-ma-n-tri}{drink-\lsc{caus}-\lsc{1.obj}-\lsc{3}-\lsc{evc}}}%morpheme+gloss
\glotran{She \pb{might} make me drink.}{}%eng+spa trans
{}{}%rec - time

% 6
\gloexe{Glo6:rikuyan}{}{ach}%
{Yakuña\pb{tr} rikuyan pampantaqa.}%ach que first line
{\morglo{yaku-ña-tr}{water-\lsc{disc}-\lsc{evc}}\morglo{ri-ku-ya-n}{go-\lsc{refl}-\lsc{prog}-\lsc{3}}\morglo{pampa-n-ta-qa}{ground-\lsc{3}-\lsc{acc}-\lsc{top}}}%morpheme+gloss
\glotran{\pb{Water should} already be running along the ground.}{}%eng+spa trans
{}{}%rec - time

% 7
\gloexe{Glo6:Allintaqa}{}{sp}%
{Allintaqa. Kapas\pb{tr}iki palabrata kichwapa apakunqa kananpis.}%sp que first line
{\morglo{allin-ta-qa}{good-\lsc{acc}-\lsc{top}}\morglo{kapas-tr-iki}{possible-\lsc{evc}-\lsc{iki}}\morglo{palabra-ta}{word-\lsc{acc}}\morglo{kichwa-pa}{Quechua-\lsc{gen}}\morglo{apa-ku-nqa}{\lsc{bring}-\lsc{refl}-\lsc{3.fut}}\morglo{kanan-pis}{now-\lsc{add}}}%morpheme+gloss
\glotran{Good. \pb{Maybe} they’ll bring Quechua now, too.}{}%eng+spa trans
{}{}%rec - time

% 8
\gloexe{Glo6:hamuyan}{}{amv}%
{Ayvis kumpañaw hamuyan -- wañuypaqpis kayachuwantriki.}%
{\morglo{ayvis}{sometimes}\morglo{kumpañaw}{accompanied}\morglo{hamu-ya-n}{come-\lsc{prog}-\lsc{3}}\morglo{wañu-y-paq-pis}{die-\lsc{1}-\lsc{purp}-\lsc{add}}\morglo{ka-ya-chuwan-tr-iki}{be-\lsc{prog}-\lsc{1pl}.\lsc{cond}-\lsc{evc}-\lsc{iki}}}%morpheme+gloss
\glotran{Sometimes someone comes accompanied -- we might be also about to die.}{}%eng+spa trans
{}{}%rec - time

\subsubsection{Evidential modification}\label{ssec:evidmodifi}
\SYQ{} counts four evidential modifiers\index[sub]{evidentials!modification}, \phono{-ari} and the set \uo, \phono{-ik} and \phono{-iki}. §~\ref{par:assertive} and~\ref{par:evistre} cover \phono{-ari} and \phono{-\uo/-ik/iki}, respectively. The latter largely repeats~\citet{Shimelman12}.

\paragraph{Assertive force \phono{-aRi}}\label{par:assertive}
\phono{-aRi}\index[sub]{evidentials!assertive force} --~realized \phono{-ali} in \CH{}~(\ref{Glo6:Wayrakuyan}) and \phono{-ari} in all other dialects~-- indicates conviction on the part of the speaker.\footnote{The Quechuas of (at least) Ancash-Huailas \citet[151]{Parker76gram},\index[aut]{Parker, Gary J.} Cajamarca-Canaris \citet[158]{Quesada76}\index[aut]{Quesada Castillo, Félix} and Junin-Huanca \citet[238--9]{CerroP76a}\index[aut]{Cerrón-Palomino, Rodolfo M.} have suffixes \phono{-rI}, \phono{-rI} and \phono{-ari}, respectively, which, like the \SYQ{} \phono{-k} succeed evidentials and are most often translated \spanish{pues} ‘then’. It seems unlikely that the \lsc{ahq}, \lsc{ccq} and \lsc{jhq} forms correspond to the \phono{-k} or \phono{-ki} of \SYQ. First, unlike \phono{-ik} or \phono{-iki}, \phono{-rI} and \phono{-ari} may appear independent of any evidential and they may function as general emphatics. Second, \SYQ, too, has a suffix \phono{-ari} which, like \phono{-rI} and \phono{-ari}, functions as a general emphatic, also translating as \spanish{pues}. Third, the \SYQ{} \phono{-ari} is in complementary distribution with \phono{-k} and \phono{-ki}. Finally, unlike the \lsc{ahq}, \lsc{ccq} and \lsc{jhq} forms, the \SYQ{} \phono{-ari} cannot appear independently of the evidentials \phono{-mI} or \phono{-shI} or else of \phono{-y}, and, further, always forms an independent word with these.} \\

% 1
\gloexe{Glo6:Wayrakuyan}{}{amv}%
{Wayrakuyan\pb{mari}.}%amv que first line
{\morglo{wayra-ku-ya-n-m-ari}{wind-\lsc{refl}-\lsc{prog}-\lsc{3}-\lsc{evd}-\lsc{ari}}}%morpheme+gloss
\glotran{\pb{It’s windy}.}{}%eng+spa trans
{}{}%rec - time

\noindent
It can often be translated as ‘surely’ or ‘certainly’ or ‘of course’. \phono{-aRi} generally occurs only in combination with \phono{-mI}~(\ref{Glo6:llapa}), (\ref{Glo6:firmachiwan}), \phono{-shI}~(\ref{Glo6:shali}), (\ref{Glo6:shari}) and \phono{-Yá}~(\ref{Glo6:Qillakuyanki}--\ref{Glo6:Yatraqninqa}).\\

% 2
\gloexe{Glo6:llapa}{}{amv}%
{Mana\pb{mari} llapa ruwayaqhina kayani.}%amv que first line
{\morglo{mana-m-ari}{no-\lsc{evd}-\lsc{ari}}\morglo{llapa}{all}\morglo{ruwa-ya-q-hina}{make-\lsc{prog}-\lsc{ag}-\lsc{comp}}\morglo{ka-ya-ni}{be-\lsc{prog}-\lsc{1}}}%morpheme+gloss
\glotran{\pb{No, of course}, it seems like I’m making it all up.}{}%eng+spa trans
{}{}%rec - time

% 3
\gloexe{Glo6:firmachiwan}{}{lt}%
{Ñuqa[ta]s firmachiwan\pb{mari}. Piru mana\pb{shari} chay wawi warmiytapis firmachinraqchu.}%lt que first line
{\morglo{ñuqa[-ta]-s}{I-\lsc{acc}-\lsc{add}}\morglo{firma-chi-wa-n-m-ari}{sign-\lsc{caus}-\lsc{1.obj}-\lsc{3}-\lsc{evd}-\lsc{ari}}\morglo{piru}{but}\morglo{mana-sh-ari}{no-\lsc{evr}-\lsc{ari}}\morglo{chay}{\lsc{dem.d}}\morglo{wawi}{baby}\morglo{warmi-y-ta-pis}{woman-\lsc{1}-\lsc{acc}-\lsc{add}}\morglo{firma-chi-n-raq-chu}{sign-\lsc{caus}-\lsc{3}-\lsc{cont}-\lsc{neg}}}%morpheme+gloss
\glotran{\pb{They made me sign}, too. But they \pb{didn’t} make my daughter sign yet, \pb{they say}.}{}%eng+spa trans
{}{}%rec - time

% 4
\gloexe{Glo6:shali}{}{ch}%
{Viñacpaq\pb{shali}.}%ch que first line
{\morglo{Viñac-paq-\pb{sh-ali}}{Viñac-\lsc{abl}-\lsc{evr}-\lsc{ari}}}%morpheme+gloss
\glotran{\pb{From Viñac, she says, then}.}{}%eng+spa trans
{}{}%rec - time

% 5
\gloexe{Glo6:shari}{}{amv}%
{Ripun\pb{shari} umaqa kunkanman.}%amv que first line
{\morglo{ripu-n-sh-ari}{go-\lsc{3}-\lsc{evr}-\lsc{ari}}\morglo{uma-qa}{head-\lsc{top}}\morglo{kunka-n-man}{neck-\lsc{3}-\lsc{all}}}%morpheme+gloss
\glotran{The head \pb{went} [flying back] towards his neck, \pb{they say}.}{}%eng+spa trans
{}{}%rec - time

% 6
\gloexe{Glo6:Qillakuyanki}{}{lt}%
{¡Kurriy! Qillakuyanki\pb{trari}.}%lt que first line
{\morglo{kurri-y}{run-\lsc{imp}}\morglo{qilla-ku-ya-nki-tr-ari}{lazy-\lsc{refl}-\lsc{prog}-\lsc{2}-\lsc{evc}-\lsc{ari}}}%morpheme+gloss
\glotran{Run!~\dots{} \pb{You must be being lazy}.}{}%eng+spa trans
{}{}%rec - time

% 7
\gloexe{Glo6:Kidakushun}{}{ach}%
{Kidakushun kaypa\pb{yari}.}%ach que first line
{\morglo{kida-ku-shun}{stay-\lsc{refl}-\lsc{1pl.fut}}\morglo{kay-pa-y-ari}{\lsc{dem.p}-\lsc{loc}-\lsc{emph}-\lsc{ari}}}%morpheme+gloss
\glotran{We’re going to stay \pb{here}.}{}%eng+spa trans
{}{}%rec - time

% 8 (10)
\gloexe{Glo6:Yatraqninqa}{}{amv}%
{Yatraqninqa mana yatraqninqa mana\pb{yari}.}%amv que first line
{\morglo{yatra-q-ni-n-qa}{know-\lsc{ag}-\lsc{euph}-\lsc{3}-\lsc{top}}\morglo{mana}{no}\morglo{yatra-q-ni-n-qa}{know-\lsc{ag}-\lsc{euph}-\lsc{3}-\lsc{top}}\morglo{mana-y-ari}{no-\lsc{emph}-\lsc{ari}}}%morpheme+gloss
\glotran{The ones who knew how. The ones who didn’t know how, \pb{no, of course}.}{}%eng+spa trans
{}{}%rec - time

\noindent
It is far less often employed than \phono{-ik} and \phono{-iki.} It is, however, prevalent in the LT dialect\phono, which supplied the single instance of \phono{tr-ari} in the corpus~(\ref{Glo6:itana}).\\

% 9 (11)
\gloexe{Glo6:itana}{}{amv}%
{Chay wayra itana piru rimidyum Hilda. ¡Piru wachikun\pb{yari}!}%amv que first line
{\morglo{chay}{\lsc{dem.d}}\morglo{wayra}{wind}\morglo{itana}{thorn}\morglo{piru}{but}\morglo{rimidyu-m}{remedy-\lsc{evd}}\morglo{Hilda}{Hilda}\morglo{piru}{but}\morglo{wachi-ku-n-y-ari}{sting-\lsc{refl}-\lsc{3}-\lsc{emph}-\lsc{ari}}}%morpheme+gloss
\glotran{The wind thorns are medicinal, Hilda. But \pb{do they ever sting}!}{}%eng+spa trans
{}{}%rec - time

\paragraph{Evidence strength \phono{-ik} and \phono{-iki}}\label{par:evistre}\index[sub]{evidentials!evidence strength}
\SYQ{} is unusual\footnote{Ayacucho Q also makes use of \phono{-ki}.} in that each of its three evidentials counts three variants, formed by the suffixation of \phono{-\uo}, \phono{-ik} or \phono{-iki}. The resulting nine forms are direct \phono{-mI-\uo}, \phono{-m-ik} and\phono{-m-iki}~(\ref{Glo6:trayamunchu}--\ref{Glo6:Wanuchinakun}); reportative \phono{-shI-\uo}, \phono{-sh-ik} and \phono{-sh-iki}~(\ref{Glo6:susyukuna}--\ref{Glo6:nisha}); and conjectural \phono{-trI-\uo}, \phono{-tr-ik} and\phono{-tr-iki}~(\ref{Glo6:Imapaqraq}--\ref{Glo6:Alkansachin}).\footnote{In Lincha, \phono{-iki} may modify both \phono{-mI} and \phono{-shI} but not \phono{-trI}; in Tana, \phono{-iki} may modify all three evidentials.}\\

% 1
\gloexe{Glo6:trayamunchu}{}{ach}%
{Manam trayamunchu mana\pb{mik} rikarinchu.}%ach que first line
{\morglo{mana-m}{no-\lsc{evd}}\morglo{traya-mu-n-chu}{arrive-\lsc{cisl}-\lsc{3}-\lsc{neg}}\morglo{mana-m-ik}{no-\lsc{evd}-\lsc{ik}}\morglo{rikari-n-chu}{appear-\lsc{3}-\lsc{neg}}}%morpheme+gloss
\glotran{He \pb{hasn’t} arrived. He \pb{hasn’t} showed up.}{}%eng+spa trans
{}{}%rec - time

% 2
\gloexe{Glo6:rishaq}{}{lt}%
{Limatam rishaq. Limapaqa buskaq kan\pb{miki}. Sutintapis rimayan\pb{miki}. ¿Ichu manachu?}%lt que first line
{\morglo{Lima-ta-m}{Lima-\lsc{acc}-\lsc{evd}}\morglo{ri-shaq}{go-\lsc{1.fut}}\morglo{Lima-pa-qa}{Lima-\lsc{loc}-\lsc{top}}\morglo{buska-q}{look.for-\lsc{ag}}\morglo{ka-n-m-iki}{be-\lsc{3}-\lsc{evd}-\lsc{iki}}\morglo{suti-n-ta-pis}{name-\lsc{3}-\lsc{acc}-\lsc{add}}\morglo{rima-ya-n-m-iki}{talk-\lsc{prog}-\lsc{3}-\lsc{evd}-\lsc{iki}}\morglo{ichu}{or}\morglo{mana-chu}{no-\lsc{q}}}%morpheme+gloss
\glotran{I’m going to go to Lima. In Lima, \pb{there are} people who read cards, \pb{then}. They’re \pb{saying} his name, \pb{then}, yes or no?}{}%eng+spa trans
{}{}%rec - time

% 3
\gloexe{Glo6:Wanuchinakun}{}{sp}%
{Wañuchinakun ima\pb{miki} chaytaqa muna:chu.}%sp que first line
{\morglo{wañu-chi-naku-n}{die-\lsc{caus}-\lsc{recip}-\lsc{3}}\morglo{ima-m-iki}{what-\lsc{evd}-\lsc{iki}}\morglo{chay-ta-qa}{\lsc{dem.d}-\lsc{acc}-\lsc{top}}\morglo{muna-:-chu}{want-\lsc{1}-\lsc{neg}}}%morpheme+gloss
\glotran{They kill each other and \pb{what-not, then}. I don’t want that.}{}%eng+spa trans
{}{}%rec - time

% 4
\gloexe{Glo6:susyukuna}{}{amv}%
{Chay\pb{shik} chay susyukuna ruwapakurqa chay nichuchanta wañushpa chayman pampakunanpaq.}%amv que first line
{\morglo{chay-sh-ik}{\lsc{dem.d}-\lsc{evr}-\lsc{ik}}\morglo{chay}{\lsc{dem.d}}\morglo{susyu-kuna}{associates-\lsc{pl}}\morglo{ruwa-paku-rqa}{make-\lsc{jtacc}-\lsc{pst}}\morglo{chay}{\lsc{dem.d}}\morglo{nichu-cha-n-ta}{crypt-\lsc{dim}-\lsc{3}-\lsc{acc}}\morglo{wañu-shpa}{die-\lsc{subis}}\morglo{chay-man}{\lsc{dem.d}-\lsc{all}}\morglo{pampa-ku-na-n-paq}{bury-\lsc{refl}-\lsc{nmlz}-\lsc{3}-\lsc{purp}}}%morpheme+gloss
\glotran{\pb{That’s why, they say}, before, the members made each other the small crypts, to bury them when they died.}{}%eng+spa trans
{}{}%rec - time

% 5
\gloexe{Glo6:Llutanshiki}{}{lt}%
{Llutanshiki. Llutan runa\pb{shik} kan.}%lt que first line
{\morglo{llutan-sh-iki}{ugly-\lsc{evr}-\lsc{iki}}\morglo{llutan}{ugly}\morglo{runa-sh-ik}{person-\lsc{evr}-\lsc{ik}}\morglo{ka-n}{be-\lsc{3}}}%morpheme+gloss
\glotran{\pb{They’re messed up, they say}. There are messed up \pb{people, they say}.}{}%eng+spa trans
{}{}%rec - time

% 6
\gloexe{Glo6:nisha}{}{ch}%
{“¡Mátalo!” nisha\pb{shiki}.}%ch que first line
{\morglo{mátalo}{{}[Spanish]}\morglo{ni-sha-sh-iki}{say-\lsc{npst}-\lsc{evr}-\lsc{iki}}}%morpheme+gloss
\glotran{“Kill him!” \pb{she’s said, they say}.}{}%eng+spa trans
{}{}%rec - time

% 7
\gloexe{Glo6:Imapaqraq}{}{ach}%
{¿Imapaqraq chayta ruwara paytaqa? Yanqaña\pb{trik} chayta wañuchira.}%ach que first line
{\morglo{ima-paq-raq}{what-\lsc{purp}-\lsc{cont}}\morglo{chay-ta}{\lsc{dem.d}-\lsc{acc}}\morglo{ruwa-ra}{make-\lsc{pst}}\morglo{pay-ta-qa}{he-\lsc{acc}-\lsc{top}}\morglo{yanqa-ña-tr-ik}{lie-\lsc{disc}-\lsc{evc}-\lsc{ik}}\morglo{chay-ta}{\lsc{dem.d}-\lsc{acc}}\morglo{wañu-chi-ra}{die-\lsc{caus}-\lsc{pst}}}%morpheme+gloss
\glotran{What did they do that to him for? They \pb{must have} killed him \pb{just for the sake of it}.}{}%eng+spa trans
{}{}%rec - time

% 8
\gloexe{Glo6:Ablan}{}{sp}%
{Ablan\pb{shiki}. “Tragu, vino”, nishpa\pb{triki} ablayamun.}%sp que first line
{\morglo{abla-n-sh-iki}{talk-\lsc{3}-\lsc{evr}-\lsc{iki}}\morglo{tragu}{drink}\morglo{vino}{wine}\morglo{ni-shpa-tr-iki}{say-\lsc{subis}-\lsc{evc}-\lsc{iki}}\morglo{abla-ya-mu-n}{talk-\lsc{prog}-\lsc{cisl}-\lsc{3}}}%morpheme+gloss
\glotran{\pb{They talk, they say, for sure}. “Pay me liquor, wine,” \pb{they must be saying}, talking.}{}%eng+spa trans
{}{}%rec - time

% 9
\gloexe{Glo6:Alkansachin}{}{amv}%
{Alkansachin warkawan\pb{tri}. Kabrapis kasusam, piru. Riqsiyan\pb{triki} runantaqa.}%amv que first line
{\morglo{alkansa-chi-n}{reach-\lsc{caus}-\lsc{3}}\morglo{warka-wan-tri}{sling-\lsc{instr}-\lsc{evc}}\morglo{kabra-pis}{goat-\lsc{add}}\morglo{kasu-sa-m}{attention-\lsc{prf}-\lsc{evd}}\morglo{piru}{but}\morglo{riqsi-ya-n-tr-iki}{know-\lsc{prog}-\lsc{3}-\lsc{evc}-\lsc{iki}}\morglo{runa-n-ta-qa}{person-\lsc{3}-\lsc{acc}-\lsc{top}}}%morpheme+gloss
\glotran{She \pb{must make [the stones] reach} with the sling, \pb{for sure}. The goats obey her. They \pb{must know} their master, \pb{for sure}.}{}%eng+spa trans
{}{}%rec - time

Evidentials obligatorily take evidentional modifier (hereafter “\lsc{em}”) arguments; \lsc{em}s are enclitics and attach exclusively to evidentials. So, for example, \phono{*mishi-m} [cat-\lsc{evd}] and \phono{*mishi-ki} (cat-\lsc{iki}) are both ungrammatical. The corresponding grammatical forms would be \phono{mishi-m-\pb{\uo}} [cat-\lsc{evd}-\uo] and \phono{*mishi\pb{-mi}-ki} (cat-\lsc{evd}-\lsc{iki}), respectively. With all three sets of evidentials, the \phono{-ik} form is associated with some variety of increase over the \phono{-\uo} form; the \phono{-iki} form, with greater increase still. With all three evidentials, \phono{-ik} and \phono{-iki} --~except in those cases in which they take scope over universal-deontic-modal or future-tense verbs~-- indicate an increase in strength of evidence. With the direct \phono{-mI}, \phono{-ik} and \phono{-iki} generally also affect the interpretation of strength of assertion; with the conjectural \phono{-trI}, the interpretation of certainty of conjecture. In the case of universal-deontic modal and future-tense verbs, with both \phono{-mI} and \phono{trI}, \phono{-ik} and \phono{-iki} indicate increasingly strong obligation and increasingly imminent/certain futures, respectively.

\subsubsection{Evidentials in questions}
In questions, the evidentials\index[sub]{evidentials!questions} generally indicate that the speaker expects a response with the same evidential (\ie,~an answer based on direct evidence, reportative evidence or conjecture, in the cases of \phono{-mI}, \phono{-shI}, and \phono{-trI}, respectively)~(\ref{Glo6:Amador}--\ref{Glo6:Kutiramunman}).\\

% 1
\gloexe{Glo6:Amador}{}{ach}%
{¿Amador Garaychu? ¿\pb{Imam} sutin kara?}%ach que first line
{\morglo{Amador}{Amador}\morglo{Garay-chu}{Garay-\lsc{q}}\morglo{ima-m}{what-\lsc{evd}}\morglo{suti-n}{name-\lsc{3}}\morglo{ka-ra}{be-\lsc{pst}}}%morpheme+gloss
\glotran{Amador Garay? \pb{What} was his name?}{}%eng+spa trans
{}{}%rec - time

% 2
\gloexe{Glo6:Maypish}{}{ch}%
{¿\pb{Maypish} wasinta lulayan?}%ch que first line
{\morglo{may-pi-sh}{where-\lsc{loc}-\lsc{evr}}\morglo{wasi-n-ta}{house-\lsc{3}-\lsc{acc}}\morglo{lula-ya-n}{make-\lsc{prog}-\lsc{3}}}%morpheme+gloss
\glotran{\pb{Where did she} say she’s making her house?}{}%eng+spa trans
{}{}%rec - time

% 3
\gloexe{Glo6:Kutiramunman}{}{ach}%
{¿Kutiramunman\pb{chutr}? ¿\pb{Imatrik} pasan?}%ach que first line
{\morglo{kuti-ra-mu-n-man-chu-tr}{return-\lsc{urgt}-\lsc{cisl}-\lsc{q}-\lsc{evc}}\morglo{ima-tr-ik}{what-\lsc{evc}-\lsc{ik}}\morglo{pasan}{pass-\lsc{3}}}%morpheme+gloss
\glotran{\pb{Could} he come back? \pb{What would have} happened?}{}%eng+spa trans
{}{}%rec - time

\noindent
The use of \phono{-trI} in a question may, additionally, indicate that the speaker doesn’t actually expect any response at all~(\ref{Glo6:Kawsan}).\\

% 4
\gloexe{Glo6:Kawsan}{}{ach}%
{¿Kawsan\pb{chutr} mana\pb{chutr}? No se sabe.}%ach que first line
{\morglo{kawsa-n-chu-tr}{live-\lsc{3}-\lsc{q}-\lsc{evc}}\morglo{mana-chu-tr?}{no-\lsc{q}-\lsc{evc}}\morglo{No se sabe.}{{}[Spanish]}}%morpheme+gloss
\glotran{\pb{Would} he be alive or dead? We don’t know.}{}%eng+spa trans
{}{}%rec - time

\noindent
And the use of \phono{-shI} may indicate not that the speaker is expecting an answer based on reported evidence, but that the speaker is reporting the question~(\ref{Glo6:5}).
