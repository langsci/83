%APPENDIX B Further analysis of evidential modifiers
\chapter{Further analysis of evidential modifiers}

This appendix presents a further analysis of the interpretation of propositions under the scope of the various permutations of the direct and the conjectural evidentials  -- \phono{-mI} and \phono{-trI} -- in combination with the three evidential modifiers --  \phono{-\uo{}},  \phono{-ik}, and  \phono{-iki}.

\section{The EM's and the interpretation of propositions under direct \phono{-mI}}
In the case of the direct \phono{-mI}, all three forms, \phono{-mI-\uo{}}, \phono{-m-ik}, and \phono{m-iki}, indicate that the speaker has evidence from personal experience for the proposition immediately under the scope of the evidential. The \phono{-ik} and \phono{-iki} forms then indicate increases in the strength of that evidence, generally that it is increasingly immediate or definitive. For example, consultants explain, with \phono{wa\~nu-rqa-\uo{}} [die-\lsc{pst}-3] `died', a speaker might use \phono{-mI-\uo{}} if she had seen the corpse, while she would use \phono{-m-iki} if she had actually been present when the person died. Or with \phono{para-ya-n} [rain-\lsc{prog}-3] `it's raining', a speaker might use \phono{-mI-\uo{}} if she were observing the rain from inside through a window, while she would use \phono{-m-iki} if she were actually standing under the rain. (1) and (2) give naturally-occurring \phono{-m-iki} examples. In (1) the speaker reports her girlhood experience working as a shepherdess in the \phono{puna} (high, cold, wet pasture grounds). What would run out on her was her matches. In (2) the speaker reports her experience with the Shining Path\index[sub]{Shining Path}, an armed Maoist group that terrorized the region in the 1980's with its robberies, kidnappings and public executions. The fight she refers to is the battle between the Shining Path and the government \emph{Sinchis} (commandos). In both examples, the speakers are reporting events they experienced with painful immediacy and with regard to which there are no more authoritative sources than themselves.

\gloexe{}{}{amv}%
{Ariy\'a urqupaqa puchukapakunchik\pb{miki}.}%amv que first line
{\morglo{ari-y\'a}{yes-\lsc{emph}}\morglo{urqu-pa-qa}{hill-\lsc{loc}-\lsc{top}}\morglo{puchuka-paku-nchik-m-iki}{finish-\lsc{mutben}-\lsc{1pl}-\lsc{evd}-\lsc{iki}}}%morpheme+gloss
\glotran{`Yes, in the hills \pb{we ran out}.'}{}%eng+spa trans
{Vinac\_SH\_Puna\_Breasts}{01:34--01:37}%rec - time

\gloexe{}{}{amv}%
{Huk visislla piliyara chaypaq chinkakura\~na\pb{miki}.}%amv que first line
{\morglo{huk}{one}\morglo{visis-lla}{times-\lsc{rstr}}\morglo{piliya-ra}{fight-\lsc{pst}}\morglo{chay-paq}{\lsc{dem.d}-\lsc{abl}}\morglo{chinka-ku-ra-\~na-m-iki}{lose-\lsc{refl}-\lsc{pst}-\lsc{disc}-\lsc{evd}-\lsc{iki}}}%morpheme+gloss
\glotran{`They fought just once and then \pb{they disappeared}.'}{}%eng+spa trans
{}{}%rec - time

In addition to indicating increases in evidence strength, \phono{-ik} and \phono{-iki}, in combination with \phono{-mI}, generally correspond to increases in strength of assertion. A \phono{-m-ik} assertion is interpreted as stronger than a \phono{-mI-\uo{}} assertion; a \phono{-m-iki} assertion as stronger still. In Spanish, \phono{-mI-\uo{}} generally has no reflex in translation. More than anything else, it serves to mark comment or focus (see \S~\ref{sec:emphasis}) or else to stand in for the copular verb \phono{ka}, defective in the third-person present tense (see \S~\ref{ssec:copu}). In contrast, \phono{-m-iki} does have a reflex in Spanish: it translates with an emphatic, either `\spanish{pues}' `then' or `\spanish{s\'i}' `yes'. So, \phono{quni-\pb{m-}\pb{\uo{}}} [warm-\lsc{evd}-\uo{}] receives the Spanish translation `\spanish{es caliente}' `it's warm'; in contrast, \phono{quni-\pb{m-iki}} [warm-\lsc{evd}-\lsc{ki}] receives the translations, `\spanish{es caliente, pues}' `it's warm, then' or `\spanish{s\'i, es caliente}' `yes, it's warm'. Example (3) is taken from a story. An old lady has sent two boys for wood -- ``so I can cook you a nice supper,'' she said. Two doves appear at the wood pile to warn the boys. \phono{Miku-shunki\pb{-m-iki}} `she's going to eat you', they warn. Using the \phono{-iki} form, the birds make the strongest assertion they can. They need to convince the boys that they are indeed in trouble -- their very lives are in danger.

\gloexe{}{}{amv}%
{Kananqa wirayaykachishunki mikushunki\pb{miki}.}%amv que first line
{\morglo{kanan-qa}{now-\lsc{top}}\morglo{wira-ya-yka-chi-shunki}{fat-\lsc{inch}-\lsc{excep}-\lsc{caus}-\lsc{3>2.fut}}\morglo{miku-shunki-{mi-ki}.}{eat-\lsc{3>2.fut}-\lsc{evd}-\lsc{iki}}}%morpheme+gloss
\glotran{`Now she's going to fatten you up and \pb{eat you}!'}{}%eng+spa trans
{Villaflor\_VA\_Dove\_Dreams}{02:02--02:06}%rec - time

In those cases in which \phono{-mI} takes scope over universal-deontic-modal or future-tense verbs, \phono{-k} and \phono{-ki} do not generally indicate an increase in evidence strength; rather, they indicate increasingly strong obligations and increasingly immediate futures, respectively. So, for example, under the scope of \phono{-mI-\uo{}}, \phono{yanapa-na-y} [help-\lsc{nmlz}-1] receives a weak universal deontic interpretation, `I ought to help'. In contrast, under the scope of \phono{-m-ik} or \phono{-m-iki}, the same phrase receives increasingly strong universal interpretations, on the order of `I have to help' and `I must help', respectively. Under the scope of \phono{-mI-\uo{}}, the phrase is understood as something like a strong suggestion, while under \phono{-m-iki}, it is understood as a more urgent obligation. That is, here, \phono{-ik} and \phono{-iki} seem to do something like increase the degree of modal force, turning a weak universal modal into a strong one. This is the case, too, where \phono{-mI} takes scope over future-tense verbs. For example, explain consultants, in the case of the future-tense \phono{ri-shaq} [go-1.\lsc{fut}] `I will go', a speaker might use\phono{-mI-\uo{}} if she were going to go at some unspecified, possibly very distant future time. In contrast, she might use \phono{-m-ik} if her going were imminent, and \phono{-m-iki} if she were already on her way. The speaker of (4), for example, urgently needed to water her garden and had been on her way to do just that when she got caught up in the conversation. When she uttered (4) she was, in fact, already in motion.

\gloexe{}{}{amv}%
{Ri\pb{shaq} yakuta\pb{miki} qawa\pb{shaq}.}%amv que first line
{\morglo{ri-shaq}{go-\lsc{1.fut}}\morglo{yaku-ta-mi-ki}{water-\lsc{acc}-\lsc{evd}-\lsc{iki}}\morglo{qawa-mu-shaq}{look-\lsc{cisl}-\lsc{1.fut}}}%morpheme+gloss
\glotran{`I'm \pb{going to} go. I'm \pb{going to} take care of the water now.'}{}%eng+spa trans
{Tana\_IM\_Orchard}{}%rec - time

\section{The EM's and the interpretation of propositions under conjectural \phono{-trI}}
In the case of the conjectural \phono{-trI}, all three forms, \phono{-trI-\uo{}}, \phono{-tri-k}, and \phono{-tri-ki}, indicate that the speaker has either direct or reportative evidence for a set of propositions, \phono{P}, and that the speaker is conjecturing from \phono{P} to \phono{p}, the proposition immediately under the scope of the evidential. The \phono{-ik} and \phono{-iki} forms then indicate increases in the strength of the speaker's evidence and generally correspond to increases in certainty of conjecture.

In case a verb under its scope is not already modalized or not already specified for modal force or conversational base by virtue of its morphology, \phono{-trI} assigns the values [universal] and [epistemic], for force and base, respectively. So, for example, the progressive present-tense \phono{kama-ta} \phono{awa-ya-n} [blanket-\lsc{acc} weave-\lsc{prog}-3] `is weaving a blanket' and the simple past-tense \phono{wa\~nu-rqa-\uo{}} [die-\lsc{pst}-3] `died', both unmodalized and therefore necessarily not specified for either modal force or conversational base, receive universal epistemic interpretations under the scope of \phono{-trI}: `he would/must be weaving a blanket' and `he would/must have died', respectively. Speakers bilingual in Yauyos and Spanish consistently translate and simple-present- and simple-past-tense verbs under the scope of \phono{-trI} with the future and future perfect, respectively. The \phono{awa-ya-n} `is weaving' and \phono{wa\~nu-rqa-\uo{}} `died' of the examples immediately above are translated `\spanish{estar\'a tejiendo}' and `\spanish{habr\'a muerto}', respectively. In English, `would' and `must' will have to do the job.

Present-tense conditional verbs in \SYQ{} may receive at least existential ability, circumstantial, deontic, epistemic and teleological interpretations. Past-tense conditional verbs may, in addition to these, also receive universal deontic and epistemic interpretations. That is, present-tense conditionals are specified for modal force [existential], but not modal base, while past-tense conditionals are specified for neither force nor base. \phono{-trI} restricts the interpretation of conditionals, generally excluding all but epistemic readings. In the case of past-tense conditionals, it generally excludes all but universal readings, as well. For example, although the present-tense conditional of (5), \phono{saya-ru-chuwan} `we could stand around', is normally five-ways ambiguous, under the scope of \phono{-trI}, only the existential epistemic reading available: `it could happen that we stand around'. Similarly, although the past-tense conditional of (6), \phono{miku-ra-ma-n-man} \phono{ka-rqa-\uo{}} `could/would/should/might have eaten me', is normally seven-ways ambiguous, under the scope of \phono{-trI}, only the universal epistemic reading is available: `the Devil would necessarily have eaten me'. The context for (1) -- a discussion of women and alcohol -- supports the epistemic reading. The speaker, a woman who in her eighty-odd years had never taken alcohol, was speculating on what would happen if women were to drink. Her conclusion: it's possible we would stand around naked, going crazy. 

\gloexe{}{}{amv}%
{Qalapis sayaruchuwan-\pb{tri} lukuyarishpaqa.}%amv que first line
{\morglo{qala-pis}{naked-\lsc{add}}\morglo{saya-ru-chuwan-tri}{stand-\lsc{urgt}-\lsc{1pl.cond}-\lsc{evc}}\morglo{luku-ya-ri-shpa-qa}{crazy-\lsc{inch}-\lsc{incep}-\lsc{subis}-\lsc{top}}}%morpheme+gloss
\glotran{`We could also stand around naked, going crazy.'}{}%eng+spa trans
{Colcas\_LM\_ShiningPath}{35:32--35:40}%rec - time

\gloexe{}{}{amv}%
{Mana chay kaptinqa mikuramanman\pb{tri} karqa chay dimunyukuna.}%amv que first line
{\morglo{mana}{no}\morglo{chay}{\lsc{dem.d}}\morglo{ka-pti-n-qa}{be-\lsc{subds}-\lsc{3}-\lsc{top}}\morglo{miku-ra-ma-n-man-\pb{tri}}{eat-\lsc{urgt}-\lsc{1.obj}-\lsc{3}-\lsc{cond}-\lsc{evc}}\morglo{ka-rqa}{be-\lsc{pst}}\morglo{chay}{\lsc{dem.d}}\morglo{dimunyu-kuna}{devil-\lsc{pl}}}%morpheme+gloss
\glotran{`If not for that, the Devil \pb{might have} eaten me.'}{}%eng+spa trans
{Tapalla\_RS\_Spin}{07:57--08:12}%rec - time

If it is the case, as \citet{Copley09} argue,\index[aut]{Copley, Bridget} and \citet{Matthewson05}\index[aut]{Matthewson, Lisa}\index[aut]{Rullmann, Hotze}\index[aut]{Davis, Henry} that the future tense is a modal specified for both force, [universal], and base, [metaphysical] or [circumstantial], \phono{-trI} should have no effect on the interpretation of mode in the case of future-tense verbs. This is indeed the case. For example, both the \phono{tiya-pa-ru-wa-nga} of (7) and \phono{ashna-ku-lla-shaq} of (8) receive exactly the interpretations they would have were they not under the scope of \phono{-trI:} `they will accompany me sitting' and `I'm going to stink', respectively. This does not mean that \phono{-trI-\uo/ik/iki} has no effect on the interpretation of future-tense verbs, however. Although it leaves \lsc{tam} interpretation unaffected, \phono{-trI} continues to indicate that the proposition under its scope is a conjecture. And \phono{-ik} and \phono{-iki}, as they do in conjunction with \phono{-mI}, indicate increasingly immediate or certain futures. So, although the \lsc{tam} interpretations of (3)'s \phono{tiya-pa-ru-wa-nga} `will accompany me sitting' and (4)'s \phono{ashna-ku-lla-shaq} `I'm going to stink' are unchanged under the scope of \phono{-trI}, the \phono{-ik} of the first and the \phono{-iki} of the second signal immediate and certain futures, respectively. In (7), that future was about an hour away: it was 6 o'clock and the coca-consuming accompaniers were expected at 7:00 for a healing ceremony. The context for (8), too, was a healing ceremony. The speaker was referring to the upcoming part of the ceremony in which she would have to wash with putrid urine -- certain to make anyone stink!

\gloexe{}{}{amv}%
{Kukachankunata aparuptiyqa tiyaparuwanqa\pb{trik}.}%amv que first line
{\morglo{kuka-cha-n-kuna-ta}{coca-\lsc{dim}-\lsc{3}-\lsc{pl}-\lsc{acc}}\morglo{apa-ru-pti-y-qa}{bring-\lsc{urgt}-\lsc{subds}-\lsc{1}-\lsc{top}}\morglo{tiya-pa-ru-wa-nqa-tri-k}{sit-\lsc{ben}-\lsc{urgt}-\lsc{1.obj}-\lsc{3.fut}-\lsc{evc}-\lsc{ik}}}%morpheme+gloss
\glotran{`When I bring them their coca, \pb{they will accompany me sitting}.'}{}%eng+spa trans
{Vinac\_JC\_Cure}{00:27--00:32}%rec - time

\gloexe{}{}{amv}%
{\textexclamdown{}Ashnakullashaq\pb{triki}!}%amv que first line
{\morglo{ashna-ku-lla-shaq-tri-ki}{smell-\lsc{refl}-\lsc{rstr}-\lsc{1.fut}-\lsc{evc}-\lsc{iki}}}%morpheme+gloss
\glotran{`\pb{I'm going to stink}!'}{}%eng+spa trans
{Vinac\_HQ\_Healer}{46:20--46:24}%rec - time

In those cases in which \phono{-ik} and \phono{-iki} modify \phono{-trI}, they generally correspond to increases in certainty of conjecture: a \phono{-tr-ik} conjecture is interpreted as more certain than a \phono{-trI\uo{}} conjecture; and a \phono{-tr-iki} conjecture is interpreted as more certain still. Recall that under the scope of \phono{-trI}, present-tense conditional verbs generally receive existential epistemic interpretations while past-tense-conditional as well as simple-present- and simple-past-tense verbs generally receive universal epistemic interpretations. In the case of the first, \phono{-k} and \phono{-ki} yield increasingly strong possibility readings; in the case of the second, third and fourth, increasingly strong necessity readings. So, under the scope of \phono{trI-\uo{}}, the present-tense conditional \phono{wa\~nu-ru-n-man} [die-\lsc{urgt}-3-\lsc{cond}] `could die' receives something like a weak possibility reading; under \phono{-tr-iki}, in contrast, the same phrase receives something like a strong possibility reading. Consultants explain that the \phono{-\uo{}} form might be used in a situation where the person was sick but it remained to be seen whether he would die; the \phono{-iki} form, in contrast, might be used in a situation where the person was gravely ill and far more likely to die. Similarly, under the scope of \phono{-trI-\uo{}}, the simple past tense \phono{wa\~nu-rqa-\uo{}} [die-\lsc{pst}-3] `died' would receive something like a weak necessity reading: it is highly probable but not completely certain that the person died. In contrast, under the scope of \phono{-tr-iki}, the same phrase would receive something like a strong necessity reading: it is very highly probable, indeed, virtually certain, that the person died. Consultants explain that a speaker might use \phono{-\uo{}} form if she knew, say, that the person, who had been very sick, still had not returned two months after having been transported down the mountain to a hospital in Lima. In contrast, that same speaker might use the \phono{-iki} form if she had, additionally, say, heard funeral bells ringing and seen two of person's daughters crying in the church. (9) and (10) give naturally-occurring examples. In (9), the speaker\tss{i} makes a present-tense conditional \phono{-trI-\uo{}} conjecture: She\tss{j} could possibly be with a soul (\ie, accompanied by the spirit of a recently deceased relative). The speaker made this conjecture after hearing the report of a single piece of evidence -- that a calf had spooked when she\tss{j} came near. Surely, whether or not a person is walking around with the spirit of a recently dead relative hovering somewhere close by is a hard thing to judge, even with an accumulation of evidence. In this case, only the weak \phono{-\uo{}} form is licensed. In (10), in contrast, the speaker makes a simple-present-tense \phono{-tr-iki} conjecture: A certain calf (a friend's) must be being weaned. The speaker, having spent all but a half dozen of her 70-odd years raising goats, sheep, cows and alpacas, would not just be making an educated guess as to whether a calf was being weaned. She knows the signs. In this situation, the strong \phono{-iki} form is licensed. 

\gloexe{}{}{amv}%
{Almayuqpis kayanman\pb{tri}.}%amv que first line
{\morglo{alma-yuq-pis}{soul-\lsc{poss}-\lsc{add}}\morglo{ka-ya-n-man-tri}{be-\lsc{prog}-\lsc{3}-\lsc{cond}-\lsc{evc}}}%morpheme+gloss
\glotran{`She \pb{might be} accompanied by a soul.'}{}%eng+spa trans
{Yuracsayhua\_UY\_Bull\_Riddles\_Souls}{13:49--13:55}%rec - time

\gloexe{}{}{amv}%
{Anuyan\~na\pb{triki}.}%amv que first line
{\morglo{anu-ya-n-\~na-tr-iki}{wean-\lsc{prog}-\lsc{3}-\lsc{disc}-\lsc{evd}-\lsc{iki}}}%morpheme+gloss
\glotran{`She \pb{must} be weaning him already, for sure}{}%eng+spa trans
{Vinac\_DC\_Milking}{00:37--00:40}%rec - time

In sum, Yauyos' three evidentials, \phono{-mI}, \phono{shI}, and \phono{-trI}, each has three variants, formed by the affixation of three evidential modifiers, \phono{-\uo}, \phono{-ik}, and \phono{-iki}. The EM's are ordered on a cline of strength, with the \phono{-ik} and \phono{-iki} forms generally indicating progressively stronger evidence. With the direct \phono{-mI}, this then generally corresponds to progressively stronger assertions; with the conjectural \phono{-trI}, to progressively more certain conjectures. In the case of verbs receiving universal-deontic-modal or future-tense interpretations, \phono{-k} and \phono{-ki} indicate stronger obligations and more imminent futures, respectively. \phono{-trI} has the prior effect of changing the modal interpretation of the verbs under its scope. In case a verb under its scope is not already already specified for modal force or conversational base by virtue of its morphology, \phono{-trI} assigns the default values [universal] and [epistemic] for force and base, respectively.

\section{A sociolinguistic note}
In a dialogue, \phono{-\uo}($\varphi$) will often be answered with \phono{-ik}($\varphi$) or \phono{-iki}($\varphi$), where $\varphi$ is a propostition-evidential pair. Thus, \phono{Karu-\pb{m}-\pb{\uo{}}} `it's far' may be answered with \phono{Aw}, \phono{karu-\pb{mi}-\pb{ki}} `Yes, you got it/that's right/you bet you/ummhunn/, it's far'. In (11), the first speaker makes a \phono{-trI-\pb{\uo{}}} conjecture, `They must have left drunk'. The second answers with \phono{-tr-i\pb{k}}, echoing the judgement of the first, `Indeed, they must have gotten drunk'.

\gloexe{}{}{amv}%
{\spkr~1: `Chay kidamuq runakuna shinka\~na\pb{tr} lluqsimurqa.' \spkr~2: `Shinkarun\pb{tri-k}.'}%amv que first line
{\morglo{chay}{\lsc{dem.d}}\morglo{kida-mu-q}{stay-\lsc{cisl}-\lsc{ag}}\morglo{runa-kuna}{person-\lsc{pl}}\morglo{shinka-\~na-tr}{drunk-\lsc{disc}-\lsc{evc}}\morglo{lluqsi-mu-rqa}{exit-\lsc{cisl}-\lsc{pst}}\morglo{shinka-ru-n-tri-k}{get.drunk-\lsc{urgt}-\lsc{3}-\lsc{evc}-\lsc{ik}}}%morpheme+gloss
\glotran{\spkr~1: `Those people who stayed must have come out drunk already.' \spkr~2: `\pb{Indeed}, they must have gotten drunk.'}%eng trans
{`Esas personas habr\'an salido borrachas ya'. `Se habr\'an emborrachado, seguro'.}%spa trans
{Vinac\_VV\_TodosMuertos}{15:38--15:43}%rec - time
